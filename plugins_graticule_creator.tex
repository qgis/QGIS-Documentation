% vim: set textwidth=78 autoindent:

\subsection{Graticule Creator Plugin}

% when the revision of a section has been finalized, 
% comment out the following line:
% \updatedisclaimer

The graticule creator allows to create a ``grid'' of points or polygons to cover our area of interest.
All units must be entered in decimal degrees.
The output is a shapefile which can be projected on the fly to match your other data.

\begin{figure}[ht]
\begin{center}
  \caption{Create a graticule layer \nixcaption}\label{fig:graticule}\smallskip
  \includegraphics[clip=true, width=10cm]{grid_maker_dialog}
\end{center}
\end{figure}

Here is an example how to create a graticule:

\begin{enumerate}
\item Start QGIS, load the Graticule Creator Plugin in the Plugin Manager (see Section 
\ref{sec:load_core_plugin}) and click on the \toolbtntwo{grid_maker}{Graticule Creator} 
icon which appears in the QGIS toolbar menu.
\item Choose the type of graticule you wish to create: point or polygon.
\item Enter the latitude and longitude for the lower left and upper right corners of the graticule.
\item Enter the interval to be used in constructing the grid. You can enter different values for the X and Y directions (longitude, latitude)
\item Choose the name and location of the shapefile to be created.
\item Click \button{OK} to create the graticule and add it to the map canvas.
\end{enumerate} 


