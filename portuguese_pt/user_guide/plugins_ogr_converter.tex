%  !TeX  root  =  user_guide.tex

\section{OGR Converter Plugin}

% when the revision of a section has been finalized, 
% comment out the following line:
%\updatedisclaimer

The OGR Layer Converter plugin adds the ability to convert vector data from one OGR-supported 
vector format to another.\footnotetext{Supported formats may vary according to the installed 
GDAL/OGR package.}
The plugin is very simple to run, and only requires a few parameters to be 
specified before running:

\begin{itemize}[label=--]
\item \textbf{Source Format/Datset/Layer}: Enter OGR format and path to the vector file to be converted
\item \textbf{Target Format/Datset/Layer}: Enter OGR format and path to the vector output file
\end{itemize}

\begin{figure}[ht]
 \centering
   \includegraphics[clip=true, width=7.5cm]{ogr_converter_dialog} 
   \caption{OGR Layer Converter Plugin \nixcaption}\label{fig:ogr_converter_dialog}
\end{figure}

\minisec{Using the Plugin}

\begin{enumerate}
  \item Start QGIS, load the OGR converter plugin in the Plugin Manager (see Section 
  \ref{sec:load_core_plugin}) and click on the \toolbtntwo{ogr_converter}{OGR Layer Converter} 
  icon which appears in the QGIS toolbar menu. The OGR Layer Converter plugin dialog appears as shown in Figure \ref{fig:ogr_converter_dialog}.
  \item Select the OGR-supported format (e.g., \selectstring{ESRI Shapefile}{\ldots}) and the path to the vector input file (e.g., \filename{alaska.shp}) in the Source area.
  \item Select the OGR-supported format (e.g., \selectstring{GML}{\ldots}) and define a path and the vector output filename (e.g., \filename{alaska.gml}) in the Target area.
  \item Click \button{Ok}.
\end{enumerate}

\FloatBarrier
