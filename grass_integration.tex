% vim: set textwidth=78 autoindent:

\section{GRASS GIS Integration}\label{sec:grass}\index{GRASS}

% when the revision of a section has been finalized, 
% comment out the following line:
%\updatedisclaimer

The GRASS plugin provides access to GRASS GIS~\cite{GRASSweb} databases and 
functionalities. This includes visualization of GRASS raster and vector 
layers, digitizing vector layers, editing vector attributes, creating new 
vector layers and analysing GRASS 2D and 3D data with more than 300 GRASS 
modules.

In this Section we'll introduce the plugin functionalities and give some 
examples on managing and working with GRASS data. Following main features 
are provided with the toolbar menu, when you start the GRASS plugin, as 
described in Section~\ref{sec:starting_grass}:
 
\begin{itemize}
\item \toolbtntwo{grass_open_mapset}{Open mapset}
\item \toolbtntwo{grass_new_mapset}{New mapset}
\item \toolbtntwo{grass_close_mapset}{Close mapset}
\item \toolbtntwo{grass_add_vector}{Add GRASS vector layer}
\item \toolbtntwo{grass_add_raster}{Add GRASS raster layer}
\item \toolbtntwo{grass_new_vector_layer}{Create new GRASS vector}
\item \toolbtntwo{grass_edit}{Edit GRASS vector layer}
\item \toolbtntwo{grass_tools}{Open GRASS tools}
%\item \toolbtntwo{grass_shell}{Open GRASS Shell}
\item \toolbtntwo{grass_region}{Display current GRASS region} 
\item \toolbtntwo{grass_region_edit}{Edit current GRASS region}
\end{itemize}

\subsection{Starting the GRASS plugin}\label{sec:starting_grass}
\index{GRASS!starting QGIS}

To use GRASS functionalities and/or visualize GRASS vector and raster layers 
in QGIS, you must select and load the GRASS plugin with the Plugin Manager. 
Therefore click the menu \mainmenuopt{Plugins} > \mainmenuopt{Manage Plugins}, 
select \dropmenuopt{GRASS} and click \button{OK}. 

You can now start loading raster and vector layers from an existing GRASS 
\filename{LOCATION} (see Section \ref{sec:load_grassdata}). Or you create a 
new GRASS \filename{LOCATION} with QGIS (see Section \ref{sec:create_loc}) 
and import some raster and vector data (see Section \ref{sec:import_loc_data}) 
for further analysis with the GRASS Toolbox (see Section 
\ref{subsec:grass_toolbox}).

\subsection{Loading GRASS raster and vector layers}\label{sec:load_grassdata}\index{GRASS!loading data}

With the GRASS plugin, you can load vector or raster layers using the
appropriate button on the toolbar menu. As an example we use the QGIS alaska
dataset (see Section \ref{label_sampledata}). It includes a small sample 
GRASS \filename{LOCATION} with 3 vector layers and 1 raster elevation map.

\begin{enumerate}
  \item Create a new folder \filename{grassdata}, download the QGIS alaska
  dataset \filename{qgis\_sample\_data.zip} from
  \url{http://download.osgeo.org/qgis/data/} and unzip the file into
  \filename{grassdata}. 
  \item Start QGIS.
  \item If not already done in a previous QGIS session, load the GRASS plugin
  clicking on \mainmenuopt{Plugins} > \mainmenuopt{Manage Plugins} and
  selecting \dropmenuopt{GRASS}. The GRASS toolbar appears on the toolbar menu.
  \item In the GRASS toolbar, click the \toolbtntwo{grass_open_mapset}{Open
  mapset} icon to bring up the \filename{MAPSET} wizard.
  \item For \filename{Gisdbase} browse and select or enter the path to the
  newly created folder \filename{grassdata}.
  \item You should now be able to select the \filename{LOCATION alaska}
  and the MAPSET \filename{demo}. 
  \item Click \button{OK}. Notice that some previously disabled tools in the 
  GRASS toolbar are now enabled.
  \item Click on \toolbtntwo{grass_add_raster}{Add GRASS raster layer},
  choose the map name \filename{gtopo30} and click \button{OK}. The elevation
  layer will be visualized.
  \item Click on \toolbtntwo{grass_add_vector}{Add GRASS vector layer},
  choose the map name \filename{alaska} and click \button{OK}. The alaska
  boundary vector layer will be overlayed on top of the gtopo30 map. You can
  now adapt the layer properties as described in chapter \ref{sec:vectorprops},
  e.g. change opacity, fill and outline color.
  \item Also load the other two vector layers \filename{rivers} and
  \filename{airports} and adapt their properties.
\end{enumerate}

As you see, it is very simple to load GRASS raster and vector layers in QGIS. 
See following Sections for editing GRASS data and creating a new 
\filename{LOCATION}. More sample GRASS \filename{LOCATIONs} are available at 
the GRASS website at \url{http://grass.osgeo.org/download/data.php}.

\begin{Tip}\caption{\textsc{GRASS Data Loading}}
\qgistip{If you have problems loading data or QGIS terminates abnormally,
check to make sure you have loaded the GRASS plugin properly as described in
Section \ref{sec:starting_grass}.
}
\end{Tip} 

\subsection{GRASS LOCATION and MAPSET}\label{sec:about_loc}

GRASS data are stored in a directory referred to as GISDBASE. This directory 
often called \filename{grassdata}, must be created before you start working 
with the GRASS plugin in QGIS. Within this directory, the GRASS GIS data 
are organized by projects stored in subdirectories called \filename{LOCATION}. 
Each \filename{LOCATION} is defined by its coordinate system, map projection 
and geographical boundaries. Each \filename{LOCATION} can have several 
\filename{MAPSETs} (subdirectories of the \filename{LOCATION}) that are used 
to subdivide the project into different topics, subregions, or as workspaces 
for individual team members (Neteler \& Mitasova 2008 
\cite{neteler_mitasova08}). In order to analyze vector and raster layers with 
GRASS modules, you must import them into a GRASS \filename{LOCATION}.
\footnote{This is not strictly true - with the GRASS modules 
\filename{r.external} and \filename{v.external} you can create read-only links 
to external GDAL/OGR-supported data sets without importing them. But because 
this is not the usual way for beginners to work with GRASS, this functionality 
will not be described here.}

\begin{figure}[ht]
\begin{center}
\caption{GRASS data in the alaska LOCATION (adapted from Neteler \& 
Mitasova 2008 \cite{neteler_mitasova08})}\label{fig:grass_location}\smallskip
\includegraphics[clip=true]{grass_location}
\end{center}  
\end{figure}

\subsubsection{Creating a new GRASS LOCATION}\label{sec:create_loc}

As an an example you find the instructions how the sample GRASS
\filename{LOCATION alaska}, which is projected in Albers Equal Area
projection with unit feet was created for the QGIS sample dataset. This
sample GRASS \filename{LOCATION alaska} will be used for all examples and
exercises in the following GRASS GIS related chapters. It is useful to
download and install the dataset on your computer \ref{label_sampledata}).

\begin{figure}[ht]
\begin{center}
\caption{Creating a new GRASS LOCATION or a new MAPSET in QGIS \nixcaption}
\label{fig:create_grass_location}\smallskip
\includegraphics[clip=true, width=10cm]{create_grass_location}
\end{center}  
\end{figure}

\begin{enumerate}
  \item Start QGIS and make sure the GRASS plugin is loaded
  \item Visualize the \filename{alaska.shp} Shapefile (see Section
  \ref{sec:load_shapefile}) from the QGIS alaska dataset~\ref{label_sampledata}.
  \item In the GRASS toolbar, click on the \toolbtntwo{grass_open_mapset}{Open
    mapset} icon to bring up the \filename{MAPSET} wizard.
  \item Select an existing GRASS database (GISDBASE) folder 
  \filename{grassdata} or create one for the new \filename{LOCATION} using a 
  file manager on your computer. Then click \button{Next}. 
  \item We can use this wizard to create a new \filename{MAPSET} within an 
  existing \filename{LOCATION} (see Section~\ref{sec:add_mapset}) or to create 
  a new \filename{LOCATION} altogether. Click on the radio button
  \radiobuttonon{Create new location} (see Figure \ref{fig:create_grass_location}).
  \item Enter a name for the \filename{LOCATION} - we used alaska and click 
  \button{Next} 
  \item Define the projection by clicking on the radio button
  \radiobuttonon{Projection} to enable the projection list 
  \item We are using Albers Equal Area Alaska (feet) projection. Since we
  happen to know that it is represented by the EPSG ID 2964, we enter it in
  the search box. (Note: If you want to repeat this process for another 
  \filename{LOCATION} and projection and haven't memorized the EPSG ID, 
  click on the
  \toolbtntwo{mIconProjectionEnabled}{projector} icon in the lower right-hand
  corner of the status bar (see Section \ref{label_projstart})).
  \item Click \button{Find} to select the projection
  \item Click \button{Next} 
  \item To define the default region, we have to enter the \filename{LOCATION} 
  bounds in north, south, east, and west direction. Here we simply click on 
  the button \button{Set current QGIS extent}, to apply the extend of the 
  loaded layer \filename{alaska.shp} as the GRASS default region extend.
  \item Click \button{Next} 
  \item We also need to define a \filename{MAPSET} within our new 
  \filename{LOCATION}. You can name it whatever you like - we used demo.
  \footnote{When creating a new \filename{LOCATION}, GRASS automatically 
  creates a special \filename{MAPSET} called \filename{PERMANENT} designed to 
  store the core data for the project, its default spatial extend and 
  coordinate system definitions (Neteler \& Mitasova 2008 
  \cite{neteler_mitasova08}).}
  \item Check out the summary to make sure it's correct and click
  \button{Finish} 
  \item The new \filename{LOCATION alaska} and two \filename{MAPSETs demo}
  and \filename{PERMANENT} are created. The currently opened working set is
  \filename{MAPSET demo}, as you defined.
  \item Notice that some of the tools in the GRASS toolbar that were 
  disabled are now enabled.
\end{enumerate}

If that seemed like a lot of steps, it's really not all that bad and a very 
quick way to create a \filename{LOCATION}. The \filename{LOCATION alaska} is 
now ready for data import (see Section \ref{sec:import_loc_data}).
You can also use the already existing vector and raster data in the sample 
GRASS \filename{LOCATION alaska} included in the QGIS alaska dataset 
\ref{label_sampledata} and move on to Section \ref{label_vectmodel}.

\subsubsection{Adding a new MAPSET}\label{sec:add_mapset}

A user has only write access to a GRASS \filename{MAPSET} he created. This 
means, besides access to his own \filename{MAPSET}, each user can also read 
maps in other user's \filename{MAPSETs}, but he can modify or remove only 
the maps in his own \filename{MAPSET}. All \filename{MAPSETs} include a 
\filename{WIND} file that stores the current boundary coordinate values and 
the currently selected raster resolution (Neteler \& Mitasova 2008 
\cite{neteler_mitasova08}, see Section \ref{sec:grass_region}). 

\begin{enumerate}
  \item Start QGIS and make sure the GRASS plugin is loaded
  \item In the GRASS toolbar, click on the 
  \toolbtntwo{grass_new_mapset}{New mapset} icon to bring up the 
  \filename{MAPSET} wizard.
  \item Select the GRASS database (GISDBASE) folder \filename{grassdata} 
  with the \filename{LOCATION alaska}, where we want to add a further 
  \filename{MAPSET}, called test.
  \item Click \button{Next}. 
  \item We can use this wizard to create a new \filename{MAPSET} within an 
  existing \filename{LOCATION} or to create a new \filename{LOCATION} 
  altogether. Click on the radio button \radiobuttonon{Select location} 
  (see Figure \ref{fig:create_grass_location}) and click \button{Next}.
  \item Enter the name \filename{text} for the new \filename{MAPSET}. Below 
  in the wizard you see a list of existing \filename{MAPSETs} and its owners.
  \item Click \button{Next}, check out the summary to make sure it's all 
  correct and click \button{Finish} 
\end{enumerate}

\subsection{Importing data into a GRASS LOCATION}\label{sec:import_loc_data}

This Section gives an example how to import raster and vector data into the 
\filename{alaska} GRASS \filename{LOCATION} provided by the QGIS alaska 
dataset. Therefore we use a landcover raster map \filename{landcover.img} 
and a vector GML File \filename{lakes.gml} from the QGIS alaska 
dataset \ref{label_sampledata}.

\begin{enumerate}
  \item Start QGIS and make sure the GRASS plugin is loaded.
  \item In the GRASS toolbar, click the \toolbtntwo{grass_open_mapset}{Open 
  MAPSET} icon to bring up the \filename{MAPSET} wizard.
  \item Select as GRASS database the folder \filename{grassdata} in the QGIS 
  alaska dataset, as \filename{LOCATION alaska}, as \filename{MAPSET} 
  \filename{demo} and click \button{OK}.
  \item Now click the \toolbtntwo{grass_tools}{Open GRASS tools} icon. The 
  GRASS Toolbox (see Section \ref{subsec:grass_toolbox}) dialog appears.
  \item To import the raster map \filename{landcover.img}, click the module 
  \filename{r.in.gdal} in the \tab{Modules Tree} tab. This GRASS module 
  allows to import GDAL supported raster files into a GRASS 
  \filename{LOCATION}. The module dialog for \filename{r.in.gdal} appears.
  \item Browse to the folder \filename{raster} in the QGIS alaska dataset 
  and select the file \filename{landcover.img}.
  \item As raster output name define \filename{landcover\_grass} and click 
  \button{Run}. In the \tab{Output} tab you see the currently running GRASS 
  command \filename{r.in.gdal -o input=/path/to/landcover.img 
  output=landcover\_grass}.
  \item When it says \textbf{Succesfully finished} click \button{View output}. 
  The \filename{landcover\_grass} raster layer is now imported into GRASS and 
  will be visualized in the QGIS canvas.
  \item To import the vector GML file \filename{lakes.gml}, click the module 
  \filename{v.in.ogr} in the \tab{Modules Tree} tab. This GRASS module allows 
  to import OGR supported vector files into a GRASS \filename{LOCATION}. The 
  module dialog for \filename{v.in.ogr} appears.
  \item Browse to the folder \filename{gml} in the QGIS alaska 
  dataset and select the file \filename{lakes.gml} as OGR file.
  \item As vector output name define \filename{lakes\_grass} and click 
  \button{Run}. You don't have to care about the other options in this 
  example. In the \tab{Output} tab you see the currently running GRASS 
  command \filename{v.in.ogr -o dsn=/path/to/lakes.gml output=lakes\_grass}.
  \item When it says \textbf{Succesfully finished} click \button{View output}. 
  The \filename{lakes\_grass} vector layer is now imported into GRASS and will 
  be visualized in the QGIS canvas. 
\end{enumerate}


\subsection{The GRASS vector data model}\label{label_vectmodel}\index{GRASS!vector data
model}

It is important to understand the GRASS vector data model prior to
digitizing.\index{GRASS!digitizing} In general, GRASS uses a topological
vector model.\index{GRASS!topology} This means that areas are not represented
as closed polygons, but by one or more boundaries. A boundary between two
adjacent areas is digitized only once, and it is shared by both areas.
Boundaries must be connected without gaps. An area is identified (labeled) 
by the centroid of the area.

Besides boundaries and centroids, a vector map can also contain
points and lines. All these geometry elements can be mixed
in one vector and will be represented in different so called 'layers' inside
one GRASS vector map. So in GRASS a layer is not a vector or raster map but a
level inside a vector layer. This is important to distinguish carefully.
\footnote{Although it
is possible to mix geometry elements, it is unusual and even in GRASS only
used in special cases such as vector network analysis. Normally you should
prefere to store different geometry elements in different layers.}

It is possible to store more 'layers' in one vector dataset. For example,
fields, forests and lakes can be stored in one vector. Adjacent
forest and lake can share the same boundary, but they have separate attribute
tables. It is also possible to attach attributes to boundaries. For example,
the boundary between lake and forest is a road, so it can have a different 
attribute table.
 
The 'layer' of the feature is defined by 'layer' inside GRASS. 'Layer' is the 
number which defines if there are more than one layer inside the dataset, e.g. 
if the geometry is forest or lake. For now, it can be only a number, in the 
future GRASS will also support names as fields in the user interface.

Attributes can be stored inside the GRASS \filename{LOCATION} as DBase or 
SQLITE3 or in external database tables, for example PostgreSQL, MySQL, 
Oracle, etc.\index{GRASS!attribute storage}

Attributes in database tables are linked to geometry elements using
a 'category' value.\index{GRASS!attribute linkage} 'Category' (key, ID) is an
integer attached to geometry primitives, and it is used as the link to one
key column in the database table.

\begin{Tip}\caption{\textsc{Learning the GRASS Vector Model}}
\qgistip{
The best way to learn the GRASS vector model and its capabilities is to 
download one of the many GRASS tutorials where the vector model is described
more deeply. See \url{http://grass.osgeo.org/gdp/manuals.php} for more
information, books and tutorials in several languages.
}
\end{Tip} 

\subsection{Creating a new GRASS vector layer}\label{sec:creating_new_grass_vectors}\index{GRASS!Creating new vectors|see{editing!creating a new layer}}

To create a new GRASS vector layer with the GRASS plugin click the 
\toolbtntwo{grass_new_vector_layer}{Create new GRASS vector} toolbar icon. 
Enter a name in the text box and you can start digitizing point, line or 
polygone geometries, following the procedure described in Section 
\ref{grass_digitising}. 

In GRASS it is possible to organize all sort of geometry types (point, line 
and area) in one layer, because GRASS uses a topological vector model, so you 
don't need to select the geometry type when creating a new GRASS vector. This 
is different from Shapefile creation with QGIS, because Shapefiles use the 
Simple Feature vector model (see Section \ref{sec:create shape}).

\begin{Tip}\caption{\textsc{Creating an attribute table for a new GRASS vector layer}}
\qgistip{
If you want to assign attributes to your digitized geometry features, make sure to create an attribute table with columns before you start digitizing (see Figure \ref{fig:grass_digitizing_table}).
}
\end{Tip} 

\subsection{Digitizing and editing a GRASS vector layer}\index{GRASS!digitizing tools}\label{grass_digitising}

The digitizing tools for GRASS vector layers are accessed using the
\toolbtntwo{grass_edit}{Edit GRASS vector layer} icon on the toolbar. Make 
sure you have loaded a GRASS vector and it is the selected layer in the legend 
before clicking on the edit tool. Figure \ref{fig:grass_digitizing_category} 
shows the GRASS edit dialog that is displayed when you click on the edit tool. 
The tools and settings are discussed in the following sections.

\begin{Tip}\caption{\textsc{Digitizing polygones in GRASS}}
\qgistip{
If you want to create a polygone in GRASS, you first digitize the boundary of 
the polygone, setting the mode to \usertext{No category}. Then you add a 
centroid (label point) into the closed boundary, setting the mode to 
\usertext{Next not used}. The reason is, that a topological vector model links 
attribute information of a polygon always to the centroid and not to the 
boundary.
}
\end{Tip} 

\minisec{Toolbar}\label{label_grasstoolbar}

In Figure \ref{fig:grass_digitizing_toolbar} you see the GRASS digitizing
toolbar icons provided by the GRASS plugin. Table \ref{tab:grass_tools}
explains the available functionalities.

\begin{figure}[h]
   \begin{center}
   \caption{GRASS Digitizing Toolbar \nixcaption}\label{fig:grass_digitizing_toolbar} 
   \includegraphics[clip=true,width=12cm]{grass_digitizing_toolbar}
\end{center}  
\end{figure}

\begin{table}[h]\index{GRASS!digitizing tools}
\centering
\caption{GRASS Digitizing Tools}\label{tab:grass_tools}\medskip
 \begin{tabular}{|l|l|p{5in}|}
 \hline \textbf{Icon} & \textbf{Tool} & \textbf{Purpose} \\
\hline \includegraphics[width=0.7cm]{grass_new_point} & New Point & Digitize
new point \\
\hline \includegraphics[width=0.7cm]{grass_new_line} & New Line & Digitize
new line (finish by selecting new tool) \\
\hline \includegraphics[width=0.7cm]{grass_new_boundary} & New Boundary &
Digitize new boundary (finish by selecting new tool)\\
\hline \includegraphics[width=0.7cm]{grass_new_centroid} & New Centroid &
Digitize new centroid (label existing area)\\
\hline \includegraphics[width=0.7cm]{grass_move_vertex} & Move vertex & Move
one vertex of existing line or boundary and identify new position\\
\hline \includegraphics[width=0.7cm]{grass_add_vertex} & Add vertex & Add a
new vertex to existing line\\
\hline \includegraphics[width=0.7cm]{grass_delete_vertex} & Delete vertex &
Delete vertex from existing line (confirm selected vertex by another click)\\
\hline \includegraphics[width=0.7cm]{grass_move_line} & Move element & Move
selected boundary, line, point or centroid and click on new position\\
\hline \includegraphics[width=0.7cm]{grass_split_line} & Split line & Split
an existing line to 2 parts\\
\hline \includegraphics[width=0.7cm]{grass_delete_line} & Delete element &
Delete existing boundary, line, point or centroid (confirm selected element by
another click)\\
\hline \includegraphics[width=0.7cm]{grass_edit_attributes} & Edit attributes
& Edit attributes of selected element (note that one element can represent
more features, see above)\\
\hline \includegraphics[width=0.7cm]{grass_close_edit} & Close & Close
session and save current status (rebuilds topology afterwards)\\
\hline
\end{tabular}
\end{table}

\minisec{Category Tab}\index{GRASS!category settings}

The \tab{Category} tab allows you to define the way in which the category 
values will be assigned to a new geometry element.

\begin{figure}[h]
 \begin{center}
  \caption{GRASS Digitizing Category Tab \nixcaption}\label{fig:grass_digitizing_category}
  \includegraphics[clip=true,width=10cm]{grass_digitizing_category}
 \end{center}
\end{figure}

\begin{itemize}
\item \textbf{Mode}: what category value shall be applied to new geometry 
elements.
\begin{itemize}
\item Next not used - apply next not yet used category value to geometry
element.
\item Manual entry - manually define the category value for the geometry
element in the 'Category'-entry field.
\item No category - Do not apply a category value to the geometry element.
This is e.g. used for area boundaries, because the category values are
connected via the centroid.
\end{itemize}
\item \textbf{Category} - A number (ID) is attached to each digitized geometry
element. It is used to connect each geometry element with its attributes.
\item \textbf{Field (layer)} - Each geometry element can be connected with
several attribute tables using different GRASS geometry layers. Default layer
number is 1. 
\end{itemize}

\begin{Tip}\caption{\textsc{Creating an additional GRASS 'layer' with QGIS}}
\qgistip{If you would like to add more layers to your dataset, just add a new
number in the 'Field (layer)' entry box and press return. In the Table tab
you can create your new table connected to your new layer.
}
\end{Tip}

\minisec{Settings Tab}\label{label_settingtab}\index{GRASS!snapping
tolerance}

The \tab{Settings} tab allows you to set the snapping in screen pixels. The
threshold defines at what distance new points or line ends are snapped to
existing nodes. This helps to prevent gaps or dangles between boundaries. The
default is set to 10 pixels.

\begin{figure}[h]
 \begin{center}
 \caption{GRASS Digitizing Settings Tab \nixcaption}\label{fig:grass_digitizing_settings}
 \includegraphics[clip=true,width=8cm]{grass_digitizing_settings}
 \end{center}
\end{figure}

\minisec{Symbology Tab}\index{GRASS!symbology settings}

The \tab{Symbology} tab allows you to view and set symbology and color
settings for various geometry types and their topological status (e.g. closed
/ opened boundary).

\begin{figure}[h]
 \begin{center}
 \caption{GRASS Digitizing Symbolog Tab \nixcaption}\label{fig:grass_digitizing_symbology}
 \includegraphics[clip=true,width=8cm]{grass_digitizing_symbology}
 \end{center}
\end{figure}

\minisec{Table Tab} \index{GRASS!table editing}

The \tab{Table} tab provides information about the database table for
a given 'layer'. Here you can add new columns to an existing attribute table,
or create a new database table for a new GRASS vector layer (see Section 
\ref{sec:creating_new_grass_vectors}).

\begin{figure}[h]
 \begin{center}
 \caption{GRASS Digitizing Table Tab \nixcaption}\label{fig:grass_digitizing_table}
 \includegraphics[clip=true,width=10cm]{grass_digitizing_table}
 \end{center}
\end{figure}

\begin{Tip}\caption{\textsc{GRASS Edit Permissions}}\index{GRASS!edit
permissions}
\qgistip{You must be the owner of the GRASS \filename{MAPSET} you want to 
edit. It is impossible to edit data layers in a \filename{MAPSET} that is not 
yours, even if you have write permissions.
}
\end{Tip} 

\subsection{The GRASS region tool}\label{sec:grass_region}\index{GRASS!region}

The region definition (setting a spatial working window) in GRASS is important 
for working with raster layers. Vector analysis is per default not limited
to any defined region definitions. All newly-created rasters will have the
spatial extension and resolution of the currently defined GRASS region,
regardless of their original extension and resolution. The current GRASS
region is stored in the \filename{\$LOCATION/\$MAPSET/WIND} file, and it 
defines north, south, east and west bounds, number of columns and rows, 
horizontal and vertical spatial resolution.

It is possible to switch on/off the visualization of the GRASS region in the
QGIS canvas using the \toolbtntwo{grass_region}{Display current GRASS region}
button. \index{GRASS!region!display}.

With the \toolbtntwo{grass_region_edit}{Edit current GRASS region} icon you 
can open a dialog to change the current region and the symbology of the GRASS 
region rectangle in the QGIS canvas. Type in the new region bounds and 
resolution and click \button{OK}. It also allows to select a new region 
interactively with your mouse on the QGIS canvas. Therefore click with the 
left mouse button in the QGIS canvas, open a rectangle, close it using the 
left mouse button again and click \button{OK}.\index{GRASS!region!editing}
The GRASS module \filename{g.region} provide a lot more parameters to define 
an appropriate region extend and resolution for your raster analysis. You can 
use these parameters with the GRASS Toolbox, described in Section 
\ref{subsec:grass_toolbox}.

\subsection{The GRASS toolbox}\label{subsec:grass_toolbox}\index{GRASS!toolbox}

The \toolbtntwo{grass_tools}{Open GRASS Tools} box provides GRASS module 
functionalities to work with data inside a selected GRASS \filename{LOCATION} 
and \filename{MAPSET}. To use the GRASS toolbox you need to open a 
\filename{LOCATION} and \filename{MAPSET} where you have write-permission 
(usually granted, if you created the \filename{MAPSET}). This is necessary, 
because new raster or vector layers created during analysis need to be written 
to the currently selected \filename{LOCATION} and \filename{MAPSET}.

\subsubsection{Working with GRASS modules}\index{GRASS!toolbox}

\begin{figure}[h]
\centering
\caption{GRASS Toolbox and searchable Modules List \nixcaption}\label{fig:grass_modules}
   \subfigure[Modules Tree] {\label{subfig:grass_module_tree}\includegraphics[clip=true, width=0.4\textwidth]{grass_toolbox_moduletree}}\goodgap
   \subfigure[Searchable Modules List] {\label{subfig:grass_module_list}\includegraphics[clip=true, width=0.4\textwidth]{grass_toolbox_modulelist}}
\end{figure}

The GRASS Shell inside the GRASS Toolbox provides access to almost all (more 
than 300) GRASS modules in command line modus. To offer a more user
friendly working environment, about 200 of the available GRASS modules and 
functionalities are also provided by graphical dialogs. These dialogs are 
grouped in thematic blocks, but are searchable as well. You find a complete 
list of GRASS modules available in QGIS version \CURRENT
in appendix \ref{appdx_grass_toolbox_modules}. It is also possible to 
customize the GRASS Toolbox content. It is described in Section 
\ref{sec:toolbox-customizing}.

As shown in Figure \ref{fig:grass_modules}, you can look for the appropriate 
GRASS module using the thematically grouped \tab{Modules Tree} or the 
searchable \tab{Modules List} tab. 

Clicking on a grapical module icon a new tab will be added to the toolbox 
dialog providing three new sub-tabs \tab{Options}, \tab{Output} and 
\tab{Manual}. In Figure \ref{fig:grass_module_dialog} you see an example 
for the GRASS module \filename{v.buffer}.

\begin{figure}[h]
\centering
\caption{GRASS Toolbox Module Dialogs \nixcaption}\label{fig:grass_module_dialog}
   \subfigure[Module Options] {\label{subfig:grass_module_option}\includegraphics[clip=true, width=0.3\textwidth]{grass_module_option}}\goodgap
   \subfigure[Modules Output] {\label{subfig:grass_module_output}\includegraphics[clip=true, width=0.3\textwidth]{grass_module_output}}\goodgap
   \subfigure[Module Manual] {\label{subfig:grass_module_manual}\includegraphics[clip=true, width=0.3\textwidth]{grass_module_manual}}
\end{figure}

\minisec{Options}

The \tab{Options} tab provides a simplified module dialog where you can 
usually select a raster or vector layer visualized in the QGIS canvas and 
enter further module specific parameters to run the module. The provided 
module parameters are often not complete to keep the dialog clear. If you want 
to use further module parameters and flags, you need to start the GRASS Shell 
and run the module in the command line.

\minisec{Output}

The \tab{Output} tab provides information about the output status of the 
module. When you click the \button{Run} button, the module switches to the 
\tab{Output} tab and you see information about the analysis process. If all 
works well, you will finally see a \usertext{Successfully finished} message.

\minisec{Manual}

The \tab{Manual} tab shows the HTML help page of the GRASS module. You can 
use it to check further module parameters and flags or to get a deeper 
knowledge about the purpose of the module. At the end of each module 
manual page you see further links to the \filename{Main Help index}, the 
\filename{Thematic index} and the \filename{Full index}. These links provide 
the same information as if you use the module \filename{g.manual} 

\begin{Tip}\caption{\textsc{Display results immediately}}\index{GRASS!display results}
\qgistip{If you want to display your calculation results immediately in your 
map canvas, you can use the 'View Output' button at the bottom of the 
module tab.
}
\end{Tip} 

\subsubsection{Working with the GRASS LOCATION browser} \index{GRASS!toolbox!Browser}

Another useful feature inside the GRASS Toolbox is the GRASS 
\filename{LOCATION} browser. In Figure~\ref{fig:grass_mapset_browser} you 
can see the current working \filename{LOCATION} with its \filename{MAPSETs}.

In the left browser windows you can browse through all \filename{MAPSETs} 
inside the current \filename{LOCATION}. The right browser window shows some 
meta information for selected raster or vector layers, e.g. resolution, 
bounding box, data source, connected attribute table for vector data and a 
command history.

\begin{figure}[h]
 \begin{center}
 \caption{GRASS LOCATION browser \nixcaption}\label{fig:grass_mapset_browser}
 \includegraphics[clip=true,width=10cm]{grass_mapset_browser}
 \end{center}
\end{figure}

The toolbar inside the \tab{Browser} tab offers following tools to manage 
the selected \filename{LOCATION}:

\begin{itemize}
\item \toolboxtwo{grass_add_map}{Add selected map to canvas}
\item \toolboxtwo{grass_copy_map}{Copy selected map}
\item \toolboxtwo{grass_rename_map}{Rename selected map}
\item \toolboxtwo{grass_delete_map}{Delete selected map}
\item \toolboxtwo{grass_set_region}{Set current region to selected map}
\item \toolboxtwo{grass_refresh}{Refresh browser window}
\end{itemize}

The \toolboxtwo{grass_rename_map}{Rename selected map} and 
\toolboxtwo{grass_delete_map}{Delete selected map} only work with maps inside 
your currently selected \filename{MAPSET}. All other tools also work with 
raster and vector layers in another \filename{MAPSET}.

\subsubsection{Customizing the GRASS Toolbox} \index{GRASS!toolbox!customize}
\label{sec:toolbox-customizing}

Nearly all GRASS modules can be added to the GRASS toolbox. A XML 
interface is provided to parse the pretty simple XML files which configures 
the modules appearance and parameters inside the toolbox.

A sample XML file for generating the module \usertext{v.buffer} (v.buffer.qgm) 
looks like this:
\begin{verbatim}
<?xml version="1.0" encoding="UTF-8"?>
<!DOCTYPE qgisgrassmodule SYSTEM "http://mrcc.com/qgisgrassmodule.dtd">

<qgisgrassmodule label="Vector buffer" module="v.buffer">
        <option key="input" typeoption="type" layeroption="layer" />
        <option key="buffer"/>
        <option key="output" />
</qgisgrassmodule>
\end{verbatim}

The parser reads this definition and creates a new tab inside the toolbox 
when you select the module. A more detailed description for adding new 
modules, changing the modules group, etc. can be found on the QGIS wiki at \\
\url{http://wiki.qgis.org/qgiswiki/Adding\_New\_Tools\_to\_the\_GRASS\_Toolbox}.

