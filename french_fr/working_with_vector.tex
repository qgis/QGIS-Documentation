%  !TeX  root  =  user_guide.tex

\chapter{Les données vectorielles}\label{label_workingvector}
%\index{vector layers|(}
\index{couches vectorielles}

\qg gère un grand nombre de formats vecteurs, dont ceux gérés par l'extension 
de conversion de données de la bibliothèque OGR, comme les formats shapefile 
ESRI,\index{shapefiles}\index{ESRI!shapefiles}\index{fichiers SHP} \map MIF 
(format d'échange)\index{fichiers MIF}\index{MapInfo!fichiers MIF} et  \map TAB 
(format natif).\index{fichiers TAB}\index{MapInfo!fichiers TAB}
Vous trouverez la liste des formats vectoriels supportés par OGR dans l'Annexe \ref{appdx_ogr}.

\qg gère également les couches \pg \index{PostGIS}\index{PostgreSQL!PostGIS} des 
bases de données \psq grâce à l'extension fournisseur de données \psq. La gestion 
d'autres types de données (par exemple les données texte délimitées) se fait 
grâce à d'autres extensions prestataires de données.\index{texte délimité}

\qg utilise la bibliothèque OGR pour lite et écrire les formats de données 
vecteurs \footnote{La gestion des vecteurs GRASS et de PostgreSQL est fournie 
par des extensions d'accès aux données natives dans QGIS.}, dont les formats shapefile
ESRI,\index{shapefiles}\index{ESRI!shapefiles}\index{fichiers SHP} \map MIF
(format d'échange)\index{fichiers MIF}\index{MapInfo!fichiers MIF} et  \map TAB
(format natif).\index{fichiers TAB}\index{MapInfo!fichiers TAB} et beaucoup 
d'autres. Au moment de la rédaction de ce manuel, 60 formats vecteurs étaient 
gérés par la bibliothèque OGR \cite{OGRWeb}. La liste complète est disponible 
sur \url{http://www.gdal.org/ogr/ogr_formats.html}.

\textbf{Note} : certains formats listés peuvent ne pas fonctionner dans QGIS 
pour différentes raisons. Par exemple, certains nécessitent des bibliothèques 
commerciales externes ou l'installation de GDAL/OGR sur votre système n'a pas 
été compilée avec la gestion du format que vous désirez utiliser. Seuls les 
formats qui ont été testés apparaitront dans la liste des types de fichiers 
lors du chargement d'un vecteur dans QGIS. Les autres formats non testés 
peuvent être chargés en sélectionnant *.*. 

Travailler avec les données vectorielles de GRASS est décrit dans la section 
\ref{sec:grass}.

%This section describes how to work with two common formats: ESRI shapefiles and \pg layers. Many of the features available in \qg work the same regardless of the vector data source.
%This is by design and includes the identify, select, labeling and attributes functions.
Cette section décrit comment travailler avec les formats les plus communs : 
les shapefiles ESRI, les couches \pg et SpatiaLite. Beaucoup des fonctionnalités 
de \qg marchent, de par sa conception, de la même manière quel que soit le format 
vecteur des données sources. Il s'agit des fonctionnalités d'identification, de 
sélection, d'étiquetage et de gestion des attributs.

%\section{ESRI Shapefiles}
%\index{vector layers!ESRI shapefiles}
%\index{shapefiles}
%\index{ESRI!shapefiles}
%\index{SHP files}
\section{Shapefiles ESRI}
\index{couches vectorielles!shapefiles ESRI}
\index{shapefiles}
\index{ESRI!shapefiles}
\index{fichiers SHP}

%The standard vector file format used in \qg is the ESRI Shapefile. It's support is provided by the OGR Simple Feature Library (\url{http://www.gdal.org/ogr/})\index{OGR}. A shapefile actually consists of a minimum of three files:\index{shapefile!format}
Le format de fichier vecteur standard utilisé par \qg est le Shapefile ESRI. 
Il est géré à travers la bibliothèque OGR Simple Feature  (\url{http://www.gdal.org/ogr/}) 
\index{OGR}. Un shapefile correspond en fait à un minimum de trois fichiers : \index{shapefile!format}

\begin{itemize}[label=--]
%\item \filename{.shp} file containing the feature geometries.
%\item \filename{.dbf} file containing the attributes in dBase format.
%\item \filename{.shx} index file.
\item \filename{.shp} fichier contenant la géométrie des entités.
\item \filename{.dbf} fichier contenant les attributs au format dBase.
\item \filename{.shx} fichier d'index.
\end{itemize}

%Ideally it comes with another file with a \filename{.prj} suffix, that contains the projection information for the shapefile. There can be more files belonging to a shapefile dataset. To have a closer look at this we recommend the technical specification for the shapefile format, that can be found at \url{http://www.esri.com/library/whitepapers/pdfs/shapefile.pdf}.\index{shapefile!specification}.
Dans l'idéal y est associé un autre fichier ayant l'extension \filename{.prj} 
qui contient les informations sur le système de coordonnées utilisé pour le 
shapefile, cependant ce n'est pas obligatoire. Il peut y avoir encore d'autres 
fichiers associés aux données shapefile. Si vous souhaitez avoir plus de détails 
nous vous recommandons de vous reporter aux spécifications techniques du format shapefile, qui se trouve notamment sur \url{http://www.esri.com/library/whitepapers/pdfs/shapefile.pdf}.\index{shapefile!spécifications}

%\minisec{Problem loading a shape .prj file}
\minisec{Problème lors du chargement d'un fichier .prj d'un shapefile}

%If you load a shapefile with \filename{.prj} file and \qg is not
%able to read the coordinate reference system from that file, you have to define the
%proper projection manually within the \tab{General} tab of the \dialog{Layer
%Properties} dialog. This is due to the fact, that \filename{.prj} files often
%do not provide the complete projection parameters, as used in \qg and listed in
%the \dialog{CRS} dialog.

Si vous chargez un shapefile avec un fichier \filename{.prj} et \qg ne peut en 
déterminer le système de coordonnées employé, vous devrez définir manuellement 
le système via l'onglet \tab{Général} des \dialog{Propriétés de la couche}. 
Cela est dû au fait que souvent les fichiers \filename{.prj} ne contiennent 
pas la totalité des paramètres tels qu'utilisés dans \qg et listé dans la 
fenêtre des \dialog{SCR}.

%For that reason, if you create a new shapefile with \qg, two different projection
%files are created. A \filename{.prj} file with limited projection parameters,
%compatible with ESRI software, and a \filename{.qpj} file, providing the complete
%parameters of the used CRS. Whenever \qg finds a \filename{.qpj} file, it will be
%used instead of the \filename{.prj}.
Pour cette raison, si vous créer un nouveau shapfile avec \qg, deux fichiers de 
projections seront crées. Un fichier \filename{.prj} avec des paramètres limités 
qui sera compatible avec un logiciel ESRI, et un fichier \filename{.qpj} contenant 
l'ensemble des paramètres du SCR utilisé. \qg utilisera par défaut ce dernier 
fichier si présent.

%\subsection{Loading a Shapefile}\label{sec:load_shapefile}
\subsection{Charger un Shapefile}\label{sec:load_shapefile}

\begin{figure}[ht]
   \begin{center}  
   \includegraphics[clip=true, width=12cm]{addvectorlayerdialog}
   \caption{Ajouter une couche vecteur \nixcaption}\label{fig:addvectorlayer}
\end{center}
\end{figure}

\begin{figure}[ht]
  \begin{center} 
  \includegraphics[clip=true, width=12cm]{shapefileopendialog}
  \caption{Fenêtre pour ouvrir une couche vecteur gérée par OGR \nixcaption}\label{fig:openshapefile}
\end{center}
\end{figure}

\begin{figure}[ht]
  \begin{center}
  \includegraphics[clip=true, width=12cm]{shapefileloaded}
    \caption{\qg avec le Shapefile de l'Alaska chargé \nixcaption}\label{fig:loadedshapefile}
\end{center}
\end{figure}

\includegraphics[width=0.7cm]{mActionAddNonDbLayer} Pour charger un shapefile, 
lancer \qg et cliquez sur \toolbtntwo{mActionAddNonDbLayer}{Ajouter une couche 
vectorielle} dans la barre d'outil\index{shapefile!chargement} ou taper simplement 
\keystroke{V}. Ce même outil peut être utilisé pour charger tous les formats 
gérés par la bibliothèque OGR.

L'outil ouvre alors une fenêtre de dialogue standard (voir figure 
\ref{fig:openshapefile}) qui vous permet de naviguer dans les 
répertoires et les fichiers et charger le shapefile ou tout autre 
format géré.
La boîte de sélection \selectstring{Fichiers de type \dots} vous 
permet de présélectionner un format de fichier géré par OGR.

Si vous le souhaitez, vous pouvez également sélectionner le type de codage du 
shapefile.

Sélectionner un shapefile dans la liste puis cliquer sur \button{Ouvrir} le 
charge dans \qg. La figure \ref{fig:loadedshapefile} montre \qg après avoir 
chargé le fichier \filename{alaska.shp}.

%\begin{Tip}\caption{\textsc{Layer Colors}}
\begin{Tip}\caption{\textsc{Couleurs de couches}}
%\qgistip{When you add a layer to the map, it is assigned a random color. When adding more than one layer at a time, different colors are assigned to each layer.}
Quand vous ajoutez une couche sur une carte, une couleur aléatoire lui est 
assignée. En ajoutant plusieurs couches en une fois, différentes couleurs 
sont assignées à chacune des couches.
\end{Tip}

%Once loaded, you can zoom around the shapefile using the map navigation tools.
%To change the symbology of a layer, open the \dialog{Layer Properties} dialog by double clicking on the layer name or by right-clicking on the name in the legend and choosing \dropmenuopt{Properties} from the popup menu. See Section \ref{sec:symbology} for more information on setting symbology of vector layers.
Une fois chargée, vous pouvez zoomer sur le shapefile en utilisant les 
outils de navigation sur la carte.
Pour changer le style d'une couche, ouvrez la fenêtre 
\dialog{Propriétés de la Couche} en double-cliquant sur le nom de la couche 
ou en faisant un clic droit sur son nom dans la légende et en choisissant 
\dropmenuopt{Propriétés} dans le menu qui apparait. Pour plus de détails 
sur les paramètres de la symbologie des couches vectorielles, référez-vous 
à la Section \ref{sec:symbology}.

\begin{Tip}\caption{\textsc{Charger une couche et un projet depuis un lecteur 
externe sous \mac}}
Sous \mac, les lecteurs portables qui sont montés à côté du disque dur primaire 
n'apparaissent pas dans Fichier -> Ouvrir un Projet comme attendu. Nous 
travaillons sur le support des fenêtres d'ouverture/enregistrement natives 
d'OS X pour résoudre ce problème. Pour y pallier, vous pouvez taper /Volumes 
dans la boîte de nom Fichier et appuyer sur Entrée. Vous pouvez ensuite 
parcourir les lecteurs externes et les montages réseau.
\end{Tip}

%\subsection{Improving Performance}
\subsection{Améliorer les performances}

%To improve the performance of drawing a shapefile, you can create a spatial index. A \index{spatial index!shapefiles} spatial index will improve the  speed of both zooming and panning. Spatial indexes used by \qg have a \filename{.qix} extension.
Pour améliorer les performances de dessin d'un shapefile, vous pouvez créer 
un index spatial. Un \index{index spatial!shapefiles} index spatial 
améliorera à la fois la vitesse d'exécution du zoom et du déplacement 
panoramique. Les index spatiaux utilisés par \qg ont une extension \filename{.qix}.

%Use these steps to create the index:
Voici les étapes de création d'un index spatial :

\begin{itemize}[label=--]
%\item Load a shapefile.
\item Chargez un shapefile
%\item Open the \dialog{Layer Properties} dialog by double-clicking on the shapefile name in the legend or by right-clicking and choosing \dropmenuopt{Properties} from the popup menu.
\item Ouvrez la fenêtre \dialog{Propriétés de la Couche} en double-cliquant sur 
le nom de la couche dans la légende ou en faisant un clic droit et en choisissant 
\dropmenuopt{Propriétés} dans le menu qui apparait.
%\item In the tab \tab{General} click the \button{Create Spatial Index} button.
\item Dans l'onglet \tab{Général}, cliquez sur le bouton \button{Créez un index spatial}.
\end{itemize}

%\subsection{Loading a \map Layer}
%\index{vector layers!MapInfo}
\section{Charger une couche MapInfo}
\index{couches vectorielles!MapInfo}

%To load a \map layer, click on the \toolbtntwo{mActionAddNonDbLayer}{Add a vector layer} toolbar bar button or type \keystroke{V}, change the file type filter to \selectstring{Files of Type}{[OGR] \map (*.mif *.tab *.MIF *.TAB)} and select the layer you want to load.
Pour charger une couche MapInfo, cliquez sur 
\toolbtntwo{mActionAddNonDbLayer}{Ajouter une couche vectorielle} dans la barre d'outils 
ou tapez \keystroke{V}, changez le type de filtre pour\\ \selectstring{Fichiers de 
type [OGR] \map (*.mif *.tab *.MIF *.TAB)} et sélectionnez la couche que vous 
souhaitez charger.

%\subsection{Loading an ArcInfo Coverage}
%\index{vector layers!ArcInfo Coverage}
\section{Charger une couverture ArcInfo binaire}
\index{couches vectorielles!couverture ArcInfo binaire}

%\includegraphics[width=0.7cm]{mActionAddNonDbLayer} To load an ArcInfo binary coverage, click on the
%\toolbtntwo{mActionAddNonDbLayer}{Add Vector Layer} toolbar button or type
%\keystroke{Ctrl-Shift-V} to open the \dialog{Add Vector Layer} dialog. Select
%\radiobuttonon{Directory}. Change to \selectstring {Type}{Arc/Info Binary Coverage}.
%Navigate to the directory that contains the coverage files and select it.

\includegraphics[width=0.7cm]{mActionAddNonDbLayer} Pour charger une couverture 
binaire ArcInfo, il faut cliquer sur le bouton\\ 
\toolbtntwo{mActionAddNonDbLayer}{Ajouter une couche vecteur} ou taper 
\keystroke{Ctrl-Shift-V} pour ouvrir le dialogue correspondant. Sélectionner le 
\radiobuttonon{Répertoire} puis définir \selectstring {Type}{Arc/Ingo Binary 
Coverage}. Naviguez jusqu'au dossier contenant vos fichiers puis choisissez-les.

De manière similaire vous pouvez directement charger les fichiers vecteurs UK 
National Transfer Format ainsi que le format TIGER brut de l'US Census Bureau.

%\section{PostGIS Layers}
\section{Couches PostGIS}
%\index{vector layers!PostGIS|see{PostGIS}}
\index{couches vectorielles!PostGIS}
%\index{PostGIS!layers}
\index{PostGIS!couches}
\label{label_postgis}

%PostGIS layers are stored in a \psq database. The advantages of \pg are the spatial indexing, filtering and query capabilities it provides. Using PostGIS, vector functions such as select and identify work more accurately than with OGR layers in \qg.
Les couches \pg sont stockées dans une base de données PostgreSQL. Les avantages 
de \pg sont les possibilités d'indexation spatiale, de filtre et de requête 
qu'il fournit. En utilisant PostGIS, les fonctions vecteur telles que la 
sélection ou l'identification fonctionnent avec plus d'exactitude qu'avec les 
couches OGR dans \qg.

%\subsection{Creating a stored Connection}\index{PostgreSQL!connection}\label{sec:postgis_stored}
\subsection{Créer une connexion enregistrée} \index{PostgreSQL!connexion} 
\label{sec:postgis_stored}

%\includegraphics[width=0.7cm]{mActionAddLayer} The first time you use a \pg data source, you must create a connection to the \psq database that contains the data. Begin by clicking on the \toolbtntwo{mActionAddLayer}{Add a \pg Layer} toolbar button, selecting the \dropmenuopttwo{mActionAddLayer}{Add a \pg Layer...} option from the \mainmenuopt{Layer} menu or typing \keystroke{D}.
\includegraphics[width=0.7cm]{mActionAddLayer} La première fois que utilisez 
une source de données PostGIS, vous devez créer une connexion à une base de 
données \psq qui contient les données. Commencez par cliquer sur le bouton 
\toolbtntwo{mActionAddLayer}{Ajouter une couche \pg} de la barre d'outils 
ou sélectionner l'option \dropmenuopttwo{mActionAddLayer}{Ajouter une 
couche PostGIS\dots} dans le menu \mainmenuopt{Couche} ou taper 
\keystroke{Ctrl-Shift-D}. Vous pouvez aussi ouvrir le dialogue 
\dialog{Ajouter une couche vecteur} et sélectionnez 
\radiobuttonon{Base de données}.

%The \dialog{Add \pg Table(s)} dialog will be displayed. To access the connection manager\index{PostgreSQL!connection manager}, click on the \button{New} button to display the \dialog{Create a New \pg Connection} dialog. The parameters required for a connection are shown in table \ref{tab:postgis_connection_parms}.
La fenêtre \dialog{Ajouter une ou plusieurs tables PostGIS} apparaît. Pour 
accéder au gestionnaire de connexion\index{PostgreSQL!gestionnaire de 
connexion}, cliquez sur le bouton \button{Nouveau} pour faire apparaitre 
la fenêtre\\ \dialog{Créer une nouvelle connexion PostGIS}. Les paramètres 
requis pour la connexion sont présentés dans le tableau \ref{tab:postgis_connection_parms}.

\begin{center}
{\setlength{\extrarowheight}{10pt}
\small
\begin{longtable}{|p{2.5cm}|p{10cm}|}
  \hline \multicolumn{2}{|c|}{\textbf{Paramètres de connexion PostGIS}} \\ 
\hline \textbf{Paramètre}&\textbf{Description} \\
\endfirsthead
\hline \textbf{Paramètre}&\textbf{Description} \\
\endhead
\hline Nom & Un nom pour cette connexion. Il peut être identique à \textsl{Base de données}.\\
\hline Service & Paramètre service à utiliser alternativement à l'hôte/port (et éventuellement
 la base de données). Cela peut être définie dans le fichier pg\_service.conf \\
\hline Hôte \index{PostgreSQL!host} & Nom pour l'hôte de la base de données. Il 
doit s'agir d'un nom existant, car il sera utilisé pour ouvrir une connexion Telnet 
ou interroger l'hôte. Si la base de données est sur le même ordinateur que \qg, 
mettez simplement localhost. \\
\hline Port \index{PostgreSQL!port}& Numéro de port que le serveur de base de 
données \psq écoute. Le port par défaut est 5432.\\
\hline Nom d'utilisateur \index{PostgreSQL!username} & Nom d'utilisateur utilisé 
pour se connecter à la base de données.\\
\hline Mot de passe \index{PostgreSQL!password} & Mot de passe utilisé avec le 
\textsl{Nom d'utilisateur} pour se connecter à la base de données.\\
\hline Base de données \index{PostgreSQL!database} & Nom de la base de données.\\
\hline Mode SSL \index{PostgreSQL!sslmode} & Comment sera négociée la connexion 
SSL avec le serveur. Voici les options :
\begin {itemize}[label=--]
\item désactiver : essayer une connexion SSL non cryptée uniquement
\item permettre : essayer une connexion non-SSL. Si cela échoue, essayer une 
connexion SSL;
\item préferer (par défaut): essayer une connexion SSL. Si cela échoue une 
connexion non-SSL;
\item requiert: essayer seulement une connexion SSL
\end {itemize}
Il faut noter qu'une accélération massive du rendu des couches \pg peut être 
obtenue en désactivant le SSL dans l'éditeur de connexion. \\
\hline
%\caption{PostGIS Connection Parameters}\label{tab:postgis_connection_parms}\medskip
\caption{Paramètres de connexion PostGIS}\label{tab:postgis_connection_parms}\index{PostgreSQL!connection parameters}
\end{longtable}}
\end{center} 

%Optional you can activate follwing checkboxes:
Vous pouvez également activer les options suivantes :

\begin{itemize}[label=--]
%\item \checkbox{Save Password}
\item \checkbox{Sauvegarder le mot de passe}
%\item \checkbox{Only look in the geometry\_columns table}
\item \checkbox{Uniquement regarder la table geometry\_columns}
%\item \checkbox{Only look in the 'public' schema}
\item \checkbox{Uniquement regarder dans le schéma 'public'}
\item \checkbox{Lister également les tables sans géométrie}
\item \checkbox{Utiliser la table des métadonnées estimées}
\end{itemize}

%Once all parameters and options are set, you can test the connection by clicking on the \button{Test Connect} button\index{PostgreSQL!connection!testing}.
Une fois que tous les paramètres et les options sont définis, vous pouvez tester 
la connexion en cliquant que le bouton  \button{Test de connexion}\index{PostgreSQL!connexion!test}.

%\begin{Tip}\caption{\textsc{\qg User Settings and Security}}\index{settings}\index{security}
\begin{Tip}\caption{\textsc{Paramètres utilisateur de \qg et Sécurité}}
\index{paramètres}\index{sécurité}
%\qgistip{Your customized settings for \qg are stored based on the operating system. \nix, the settings are stored in your home directory in \filename{.qt/qgisrc}. \win, the settings are stored in the registry. Depending on your computing environment, storing passwords in your \qg settings may be a security risk.}
Vos paramètres personnalisés pour \qg sont stockés différemment selon le 
système d'exploitation. \nix, les paramètres sont stockés dans votre répertoire 
home dans \filename{.\qg/}. \win, les paramètres sont stockés dans la base de 
registre. Selon votre environnement informatique, stocker vos mots de passe 
dans vos paramètres \qg peut présenter des risques vis-à-vis de la sécurité.
\end{Tip}

%\subsection{Loading a \pg Layer}\index{PostgreSQL!loading layers}
\subsection{Charger une couche PostGIS}\index{PostgreSQL!charger des couches}

%\includegraphics[width=0.7cm]{mActionAddLayer} Once you have one or more connections defined, you can load layers from the \psq database. Of course this requires having data in PostgreSQL. See Section \ref{sec:loading_postgis_data} for a discussion on importing data into the database.
\includegraphics[width=0.7cm]{mActionAddLayer} Une fois une ou plusieurs 
connexions définies, vous pouvez charger des couches de la base de données 
PostgreSQL. Bien sûr, cela nécessite d'avoir des données dans PostgreSQL. 
Référez-vous à la Section \ref{sec:loading_postgis_data} pour plus de 
détails concernant l'importation de données dans la base de données.

%To load a layer from PostGIS, perform the following steps:
Pour charger une couche PostGIS, suivez ces étapes :

\begin{itemize}[label=--]
%\item If the \dialog{Add \pg Table(s)} dialog is not already open, click on the \toolbtntwo{mActionAddLayer}{Add a \pg Layer} toolbar button.
\item Si la fenêtre \dialog{Ajouter une ou plusieurs tables PostGIS} n'est pas 
ouverte, cliquez sur le bouton \toolbtntwo{mActionAddLayer}{Ajouter une couche 
PostGIS} de la barre d'outils.
%\item Choose the connection from the drop-down list and click \button{Connect}.
\item Choisissez la connexion dans la liste déroulante et cliquez sur \button{Connecter}.
\item Sélectionnez ou désélectionnez \checkbox{Lister également les tables sans géométrie}
\item Éventuellement utilisez des \checkbox{Options de recherche} pour définir quelles 
entités à charger dans la couche ou utiliser l'icône \button{Construire une requête} 
pour lancer la boîte de dialogue du constructeur de requête.

\item Trouvez la couche que vous souhaitez ajouter dans la liste des couches disponibles.
%\item Select it by clicking on it. You can select multiple layers by holding down the \keystroke{shift} key while clicking. See Section \ref{sec:query_builder} for information on using the \psq Query Builder to further define the layer.
\item Sélectionnez-la en cliquant dessus. Vous pouvez sélectionner plusieurs 
couches en gradant la touche \keystroke{shift} enfoncée quand vous cliquez. 
Référez-vous à la Section \ref{sec:query_builder} pour plus d'informations 
sur l'utilisation du Constructeur de requête de \psq pour mieux définir la couche.
%\item Click on the \button{Add} button to add the layer to the map.
\item Cliquez sur le bouton \button{Ajouter} pour ajouter la couche à la carte.
\end{itemize}

%\begin{Tip}\caption{\textsc{PostGIS Layers}}
\begin{Tip}\caption{\textsc{Couches PostGIS}}
%\qgistip{Normally a \pg layer is defined by an entry in the geometry\_columns table. From version \OLD % should be 0.9.0 on, \qg can load layers that do not have an entry in the geometry\_columns table. This includes both tables and views. Defining a spatial view provides a powerful means to visualize your data. Refer to your \psq manual for information on creating views.}
Normalement, une couche \pg est définie par une entrée dans la table geometry\_columns. 
Depuis la version 0.11.0, \qg peut charger des couches qui n'ont pas d'entrée dans la 
table geometry\_columns. Ceci concerne aussi bien les tables que les vues. Définir une 
vue spatiale fournit un moyen puissant pour visualiser vos données. Référez-vous à 
votre manuel \psq pour plus d'informations sur la création des vues.
\end{Tip}

%\subsection{Some details about \psq layers}\label{sec:postgis_details}
%\index{PostgreSQL!layer details}
\subsection{Quelques éléments de détail à propos des couches PostgreSQL} \label{sec:postgis_details}
\index{PostgreSQL!détails sur les couches}

%This section contains some details on how \qg accesses \psq layers. Most of the time \qg should simply provide you with a list of database tables that can be loaded, and load them on request. However, if you have trouble loading a \psq table into \qg, the information below may help you understand any \qg messages and give you direction on changing the \psq table or view definition to allow \qg to load it.
Cette section contient quelques détails sur la manière dont \qg accède aux 
couches PostgreSQL. La plupart du temps, \qg devrait simplement fournir une 
liste de tables de base de données qui peuvent être chargées et les charges à la 
demande. Cependant, si vous avez des problèmes pour charger une table \psq dans 
\qg, les informations données ci-dessous peuvent vous aider à comprendre les 
messages de \qg et vous donnez une indication sur comment changer la table ou 
la vue \psq pour qu'elle se charge dans \qg.

%\qg requires that PostgreSQL layers contain a column that can be
%used as a unique key for the layer. For tables this usually means
%that the table needs a primary key, or a column with a unique
%constraint on it. In \qg, this column needs to be of
%type int4 (an integer of size 4 bytes). Alternatively the ctid column can be
%used as primary key. If a table lacks these items,
%the oid column will be used instead. Performance will be improved if the
%column is indexed (note that primary keys are automatically indexed in
%PostgreSQL).
\qg demande que les couches \psq aient un champ qui peut être utilisé 
comme clé unique pour la couche. Pour les tables, cela signifie qu'elles 
doivent avoir une clé primaire ou un champ ayant une contrainte d'unicité. 
De plus, \qg impose que cette colonne soit de type int4 (un entier de 4 
octets). Alternativement la colonne ctid peut être utilisée comme clé 
primaire. Si une table ne respecte pas ces conditions, le champ oid sera utilisé à la place. Les performances seront améliorées si le 
champ est indexé (notez que les clés primaires sont automatiquement 
indexées dans PostgreSQL).

%If the \psq layer is a view, the same requirements exists, but views don't have primary keys or columns with unique constraints on them. In this case \qg will try to find a column in the view that is derived from a table column that is suitable. If one cannot be found, \qg will not load the layer. If this occurs, the solution is to alter the view so that it does include a suitable column (a type of int4 and either a primary key or with a unique constraint, preferably indexed).
Si la couche \psq est une vue, les mêmes conditions s'appliquent, mais 
elles n'ont pas de clé primaire ou de champ ayant une contrainte 
d'unicité. Dans ce cas, \qg essayera de trouver un champ de la vue 
issu d'un champ une table qui convienne en parcourant la définition 
SQL de la vue. Cependant, il y a certains aspects du SQL que \qg 
ignore tel que l'utilisation d'alias ou de colonnes générées par 
des fonctions. S'il ne peut pas en trouver, \qg ne chargera pas la 
couche. Si cela arrive, la solution consiste à modifier la vue de 
telle sorte qu'elle inclut un champ qui convienne (de type int4 et 
ayant soit une clé primaire soit une contrainte d'unicité, de 
préférence indexée).

%\subsection{Importing Data into PostgreSQL}\label{sec:loading_postgis_data}\index{PostGIS!SPIT!importing data}
\subsection{Importer des données dans PostgreSQL} \label{sec:loading_postgis_data}\index{PostGIS!SPIT!importer des données}

\minisec{shp2pgsql}
%Data can be imported into \psq using a number of methods. \pg includes a utility called \filename{shp2pgsql} that can be used to import shapefiles into a \pg enabled database. For example, to import a shapefile named \filename{lakes.shp} into a \psq database named \usertext{gis\_data}, use the following command:
De multiples méthodes existent pour importer des données dans 
PostgreSQL. \pg incluent un utilitaire nommé \filename{shp2pgsql} 
qui peut être utilisé pour importer des shapefiles dans des bases 
de données disposant de PostGIS. Par exemple, pour importer le 
shapefile \filename{lakes.shp} dans une base de données \psq 
nommée \usertext{gis\_data}, utiliser la commande suivante :

\begin{verbatim}
  shp2pgsql -s 2964 lakes.shp lakes_new | psql gis_data
\end{verbatim}

%This creates a new layer named \usertext{lakes\_new} in the \usertext{gis\_data} database. The new layer will have a spatial reference identifier (SRID) of 2964. See Section \ref{label_projections} for more information on spatial reference systems and projections.
Ceci crée une nouvelle couche nommée \usertext{lakes\_new} dans la 
base de données usertext{gis\_data}. La nouvelle couche aura 
l'identifiant de référence spatiale (SRID) 2964. Référez-vous 
à la Section \ref{label_projections} pour plus d'informations sur 
les systèmes de référence spatiale et les projections.
\begin{Tip}
%\caption{\textsc{Exporting datasets from PostGIS}\index{PostGIS!Exporting}}
\caption{\textsc{Exporter des jeux de données depuis PostGIS}\index{PostGIS!Exporter}}
%\qgistip{Like the import-tool \filename{shp2pgsql} there is also a tool to export PostGIS-datasets as shapefiles: \filename{pgsql2shp}. This is shipped within your \pg distribution.}
Comme l'outil d'importation \filename{shp2pgsql}, il y a également 
un outil d'exportation de jeux de données \pg en shapefile : 
\filename{pgsql2shp}. Cet outil est inclus dans la distribution de 
PostGIS.
\end{Tip}

%\minisec{SPIT Plugin}
\minisec{Extension SPIT}
%\includegraphics[width=0.7cm]{spiticon} \qg comes with a plugin named SPIT (Shapefile to \pg Import Tool)\index{PostGIS!SPIT}. SPIT can be used to load multiple shapefiles at one time and includes support for schemas. To use SPIT, open the Plugin Manager from the \mainmenuopt{Plugins} menu, check the box next to the \checkbox{SPIT plugin} and click \button{OK}. The SPIT icon will be added to the plugin toolbar\index{PostGIS!SPIT!loading}.
\includegraphics[width=0.7cm]{spiticon.png} \qg est distribué avec 
une extension nommée SPIT (Shapefile to \pg Import Tool)\index{PostGIS!SPIT}. 
SPIT peut être utilisé pour charger plusieurs shapefiles en une fois et 
inclut la gestion des schémas. Pour utiliser SPIT, ouvrez le Gestionnaire 
d'extensions depuis le menu \mainmenuopt{Plugins}, cochez la case adjacente 
à \checkbox{SPIT plugin} et cliquez sur \button{OK}. L'icône SPIT sera 
ajoutée à la barre d'outils\index{PostGIS!SPIT!charger}.

%To import a shapefile, click on the \toolbtntwo{spiticon}{SPIT} tool in the toolbar to open the \dialog{SPIT - Shapefile to \pg Import Tool} dialog. Select the \pg database you want to connect to and click on \button{Connect}. Now you can add one or more files to the queue by clicking on the \button{Add} button. To process the files, click on the \button{OK} button. The progress of the import as well as any errors/warnings will be displayed as each shapefile is processed.
Pour importer un shapefile, cliquez sur le bouton \toolbtntwo{spiticon}{SPIT} 
dans la barre d'outils pour ouvrir la fenêtre \dialog{SPIT - Outil 
d'importation de Shapefile dans PostGIS}. Sélectionnez la base de données 
à laquelle vous voulez vous connecter et cliquez sur le bouton \button{Connecter}. 
Si vous le désirez, vous pouvez chanrger une ou plusieurs options. Vous 
pouvez alors ajouter un ou plusieurs fichiers à la liste en cliquant sur 
le bouton \button{Ajouter}. Pour traiter les fichiers, appuyez sur le bouton 
\button{OK}. La progression de l'importation aussi bien que les erreurs ou 
les alertes s'afficheront pour chaque shapefile.

%\begin{Tip}\caption{\textsc{Importing Shapefiles Containing \psq Reserved Words}}\index{PostGIS!SPIT!reserved words}
\begin{Tip}\caption{\textsc{Importer des shapefiles contenant des mots 
réservés de PostgreSQL}}\index{PostGIS!SPIT!mots réservés}
%\qgistip{If a shapefile is added to the queue containing fields that are reserved words in the \psq database a dialog will popup showing the status of each field. You can edit the field names\index{PostGIS!SPIT!editing field names} prior to import and change any that are reserved words (or change any other field names as desired). Attempting to import a shapefile with reserved words as field names will likely fail.}
Si un shapefile est ajouté à la liste et que des noms de champs correspondent 
à des mots réservés dans une base de données PostgreSQL, une fenêtre apparaitra 
et montrera le statut de chaque champ. Vous pouvez éditer les noms des champs
\index{PostGIS!SPIT!éditer des noms de champ} avant l'importation et changer 
ceux qui correspondent à un mot réservé (ou faire les changements désirés). 
Toute tentative d'importer un shapefile ayant un champ contenant un mot 
réservé devrait vraisemblablement échouer.
\end{Tip}

\minisec{ogr2ogr}
%Beside \filename{shp2pgsql} and \filename{SPIT} there is another tool for feeding geodata in PostGIS: \filename{ogr2ogr}. This is part of your GDAL installation.
En plus de \filename{shp2pgsql} et \filename{SPIT}, un autre outil est 
fourni pour importer des données géographiques dans \pg : \filename{ogr2ogr}. 
Il est inclus dans GDAL.
%To import a shapefile into PostGIS, do the following:
Pour importer un shapefile dans PostGIS, tapez la commande suivante :
\begin{verbatim}
  ogr2ogr -f "PostgreSQL" PG:"dbname=postgis host=myhost.de user=postgres \
  password=topsecret" alaska.shp
\end{verbatim}

%This will import the shapefile \filename{alaska.shp} into the PostGIS-database \usertext{postgis} using the user \usertext{postgres} with the password \usertext{topsecret} on host \server{myhost.de}.
Ceci va importer le shapefile \filename{alaska.shp} dans la base de données \pg 
\usertext{postgis} en utilisant l'utilisateur \usertext{postgres} avec le mot 
de passe \usertext{topsecret} sur l'hôte \server{myhost.de}.

%Note that OGR must be built with \psq to support PostGIS. You can see this by typing
Notez qu'OGR doit être compilé avec \psq pour gérer PostGIS. Vous pouvez 
vérifier en tapant :
\begin{verbatim}
ogrinfo --formats | grep -i post
\end{verbatim}

%If you like to use PostgreSQL's \filename{COPY}-command instead of the default \filename{INSERT INTO} method you can export the following environment-variable (at least available on \nix and \osx):
Si vous préférez utiliser la commande \psq \filename{COPY} au lieu de la 
méthode par défaut, \filename{INSERT INTO}, vous pouvez exporter la variable 
d'environnement suivante (au moins sur \nix et \osx) :
\begin{verbatim}
  export PG_USE_COPY=YES
\end{verbatim}

%\filename{ogr2ogr} does not create spatial indexes like \filename{shp2pgsl} does. You need to create them manually using the normal SQL-command \filename{CREATE INDEX} afterwards as an extra step (as described in the next section \ref{label_improve}).
\filename{ogr2ogr} ne crée pas d'index spatial comme le fait \filename{shp2pgsl}. 
Vous devez effectuer une étape supplémentaire et le créer manuellement après en 
utilisant la commande SQL classique \filename{CREATE INDEX} (comme cela est 
détaillé dans la section suivante \ref{label_improve}).

%\subsection{Improving Performance} \label{label_improve}
\subsection{Améliorer les performances} \label{label_improve}

%Retrieving features from a \psq database can be time consuming, especially over a network. You can improve the drawing performance of \psq layers by ensuring that a \index{PostGIS!spatial index} spatial index exists on each layer in the database. \pg supports creation of a \index{PostGIS!spatial index!GiST} GiST (Generalized Search Tree) index to speed up spatial searches of the data.
Récupérer des entités depuis une base de données \psq peut être long, surtout par 
un réseau. Vous pouvez améliorer les performances de dessin de couches \psq en 
vous assurant qu'un \index{PostGIS!index spatial} index spatial existe pour 
chaque couche dans la base de données. \pg gère la création d'un index 
\index{PostGIS!index spatial!GiST} GiST (Generalized Search Tree) pour 
accélérer les recherches spatiales sur les données.

%The syntax for creating a GiST\footnote{GiST index information is taken from the \pg documentation available at \url{http://postgis.refractions.net}} index is:
La syntaxe pour créer un index GiST\footnote{les informations de l'index GiST 
proviennent de la documentation de \pg disponible sur \url{http://postgis.refractions.net}} est la suivante :

\begin{verbatim}
    CREATE INDEX [indexname] ON [tablename]
      USING GIST ( [geometryfield] GIST_GEOMETRY_OPS );
\end{verbatim}

%Note that for large tables, creating the index can take a long time. Once the index is created, you should perform a \usertext{VACUUM ANALYZE}. See the \pg documentation \cite{PostGISweb} for more information.
Notez que pour de grandes tables, créer un index peut prendre du temps. Une 
fois cet index créé, vous devriez faire une \usertext{VACUUM ANALYZE}. 
Référez-vous à la documentation de \cite{PostGISweb} pour plus d'informations.

%The following is an example of creating a GiST index:
Voici un exemple de création d'un index GiST :
\begin{verbatim}
gsherman@madison:~/current$ psql gis_data
Welcome to psql 8.3.0, the \psq interactive terminal.

Type:  \copyright for distribution terms
        \h for help with SQL commands
        \? for help with psql commands
        \g or terminate with semicolon to execute query
        \q to quit

gis_data=# CREATE INDEX sidx_alaska_lakes ON alaska_lakes
gis_data-# USING GIST (the_geom GIST_GEOMETRY_OPS);
CREATE INDEX
gis_data=# VACUUM ANALYZE alaska_lakes;
VACUUM
gis_data=# \q
gsherman@madison:~/current$
\end{verbatim}

\subsection{couches vectorielles dépassants les \degrees{180} de longitude}
\index{couches vectorielles!croiser}

Beaucoup de logiciels de SIG ne traitent pas les cartes vecteurs ayant un système 
de référence géographique dépassant la ligne des \degrees{180} de longitude. 
Il en résulte que sous \qg on verra 2 emplacements distincts et éloignés qui 
devraient être proches l'un de l'autre. Sur la figure \ref{fig:vector_not_wrapping} 
le petit point tout à gauche sur le canevas cartographique (Chatham Island) 
devrait être dans la grille, à droite des îles principales de Nouvelle-Zélande.

\begin{figure}[ht]
   \begin{center}
   \includegraphics[clip=true, width=\textwidth]{vectorNotWrapping}
   \caption{Carte en lat/lon dépassant les \degrees{180} de longitude \nixcaption}
   \label{fig:vector_not_wrapping}
\end{center}
\end{figure}

Une solution est de transformer les valeurs longitudinales en utilisant \pg 
et la fonction \textbf{ST\textunderscore Shift\textunderscore Longitude}
\footnote{\url{http://postgis.refractions.net/documentation/manual-1.4/ST_Shift_Longitude.html}}. 
Cette fonction lit chaque point/sommet de chacune des entités dans une 
géométrie et si la coordonnée de longitude est inférieure à \degrees{0} 
elle lui ajoute \degrees{360}. Le résultat est une version de \degrees{0} 
-- \degrees{360} version des données sur une carte centré à \degrees{180}.

\begin{figure}[ht]
   \begin{center}
   \includegraphics[clip=true, width=9cm]{vectorWrapping}
   \caption{Carte dépassant la ligne des \degrees{180} de longitude après 
	application de la fonction ST\textunderscore Shift\textunderscore Longitude}
\label{fig:vector_wrapping}
\end{center}
\end{figure}

\minisec{Usage}

\begin{itemize}[label=--]
\item Importer des données vers \pg (\ref{sec:loading_postgis_data}) en 
utilisant par exemple l'extension de gestion de \pg (PostGIS Manager) ou 
l'extension SPIT
\item Utiliser l'interface en ligne de commande \pg pour exécuter la 
commande suivante (c'est un exemple où TABLE est bien le nom de votre 
table PostGIS) \\ 
\texttt{gis\_data=\# update TABLE set the\_geom=ST\_shift\_longitude(the\_geom);} 
\item Si tout ce passe bien, vous devriez recevoir une confirmation sur 
le nombre d'entités qui ont été mise à jour, puis vous pouvez charger 
la carte et voir la différence (Figure \ref{fig:vector_wrapping})
\end{itemize}

\section{Couches SpatiaLite} 
\index{SpatiaLite layers!properties dialog}
\index{couches vectorielles!SpatiaLite|see{SpatiaLite}}
\index{SpatiaLite!couches}
\label{label_spatialite} 

\includegraphics[width=0.7cm]{mActionAddSpatiaLiteLayer}
La première fois que vous chargerez une base SpatiaLite, commencez par 
cliquer sur le bouton \toolbtntwo{mActionAddSpatiaLiteLayer}{Ajouter une 
couche SpatiaLite} ou en sélectionnant l'option\\ \dropmenuopttwo{mActionAddSpatiaLiteLayer}{Ajouter 
une couche SpatiaLite\dots} depuis le menu \mainmenuopt{Couche} ou en 
tapant \keystroke{L}. Ceci fait apparaitre une fenêtre qui vous permet 
soit de vous connecter à une base déjà connue de \qg, que vous pouvez 
choisir dans une liste déroulante, ou définissant une nouvelle connexion. 
Pour se faire, cliquez sur \button{Nouveau} et utilisez le navigateur 
de fichier pour pointer votre base SpatiaLite qui se termine par une 
extension \filename{.SQLite }.

Si vous voulez sauver une couche vectorielle dans le format SpatiaLite, vous 
pouvez réaliser cela en cliquant avec le bouton droit de la souris sur 
la couche dans la légende. Puis cliquez sur \dropmenuopt{Sauvegarder sous}, 
définissez le nom du fichier de sortie, SQLite comme format et la projection 
puis ajouter 'SPATIALITE=YES' dans le champ d'option de création de source 
de données. Cela indique à OGR de créer une base de données SpatiaLite. 
Voyez également \url{http://www.gdal.org/ogr/drv_sqlite.html}.

\minisec{Créer une nouvelle couche SpatiaLite}

Si vous voulez créer une nouvelle couche SpatiaLite, référez-vous à la section 
\ref{sec:create spatialite}.

\begin{Tip}\caption{\textsc{Extension Gestion de base de données SpatiaLite}}\index{SpatiaLite!Gestion base de données} 
Pour la gestion de base de données SpatiaLite, vous pouvez aussi utiliser 
l'extension 'QspatiaLite' du 'dépôt des contributeurs de QGIS'. Il peut 
être téléchargé et intégré avec l'installeur d'extensions Python et 
fournir une intégration avec QGIS (import de couches QGIS, tables de vues 
spatiales et requêtes dans QGIS) un éditeur SQL avec coloration de la syntaxe 
et autocomplétion et un constructeur de requête SQL pour construire des 
requêtes complexes et d'autres fonctionnalités.
\end{Tip}

%\section{The Vector Properties Dialog}\label{sec:vectorprops}
%\index{vector layers!properties dialog}
\section{La fenêtre Propriété des couches vectorielles}\label{sec:vectorprops}
\index{couches vectorielles!fenêtre propriété}

%The \dialog{Layer Properties} dialog for a vector layer provides information about the layer, symbology settings and labeling options. If your vector layer has been loaded from a \psq / \pg datastore, you can also alter the underlying SQL for the layer - either by hand editing the SQL on the \tab{General} tab or by invoking the \dialog{Query Builder} dialog on the \tab{General} tab. To access the \dialog{Layer Properties} dialog, double-click on a layer in the legend or right-click on the layer and select \dropmenuopt{Properties} from the popup menu.
La fenêtre \dialog{Propriétés de la couche} pour une couche vectorielle 
fournit des informations sur la couche, les paramètres de représentation 
et les options d'étiquetage. Si votre couche a été chargée depuis une 
base \ppg, vous pouvez également modifier la requête SQL d'appel de la 
couche, soit manuellement en éditant le SQL dans l'onglet \tab{Général} 
soit en appelant la fenêtre \dialog{Constructeur de requête} depuis 
l'onglet \tab{Général}. Pour accéder à la fenêtre \dialog{Propriétés 
de la couche}, double-cliquez sur la couche dans la légende ou faites 
un clic droit sur la couche et sélectionnez \dropmenuopt{Propriétés} 
dans le menu qui apparait.

\begin{figure}[ht]
  \begin{center}
  %\caption{Vector Layer Properties Dialog \nixcaption}\label{fig:vector_symbology}\smallskip
  \includegraphics[clip=true, width=12cm]{vectorLayerSymbology}
  \caption{Fenêtre Propriétés d'une couche vecteur \nixcaption}\label{fig:vector_symbology}
\end{center}
\end{figure}

%\subsection{Symbology Tab}\label{sec:symbology}
%\index{vector layers!symbology}
\subsection{Onglet Style}\label{sec:symbology}
\index{couches vectorielles!symbologie}

Depuis \qg 1.4.0 une nouvelle sémiologie a été intégrée en parallèle pour 
améliorer et finalement remplacer l'ancienne sémiologie. \qg 1.7.0 utilise 
maintenant la nouvelle sémiologie par défaut, qui apporte de nombreuses 
améliorations et de nouvelles fonctionnalités.

Une description de l'ancienne sémiologie est disponible dans la section 
\ref{sec:oldsymbology}.

%\minisec{Understanding the new generation symbology}
\minisec{Comprendre la nouvelle génération de la sémiologie}

%There are three types of symbols: marker symbols (for points), line symbols and
%fill symbols (for polygons). Symbols can consist of one or more symbol layers. It
%is possible to define the color of a symbol and this color is then defined for all
%symbol layers. Some layers may have the color locked - for those the color can not
%be altered. This is useful when you define the color of a multilayer symbol.
%Similarly, it is possible to define the width for line symbols, as well as size and
%angle for marker symbols.
Il y a trois types de symboles : le symbole marqueur (pour les points), le symbole de 
lignes et les symboles de remplissage et de contour (pour les polygones). Un symbole peut lui-même 
être composé de plusieurs couches de symboles. Il est possible de définir la couleur 
d'un symbole pour l'ensemble des couches de symbole, certaines couches peuvent avoir 
leur couleur verrouillée. De la même manière, il est possible de définir la largeur 
d'une ligne ainsi que la taille et l'orientation des marqueurs.

%Available symbol layer types}
\minisec{Types de symboles disponibles}\label{symboltypes}

\begin{itemize}[label=--]
\item Couches ponctuelles
\begin{itemize}[label=--]
\item \textbf{Marqueur de police} : rendu avec une police
\item \textbf{Marqueur simple} : Affichage utilisant un marqueur "en dur" (c.-à-d., 
incorporé dans le code de \qg).
\item \textbf{Marqueur SVG}: Affichage utilisant une image SVG.
\end{itemize}
\item Couches linéaires
\begin{itemize}[label=--]
\item \textbf{Line decoration} : ajoute une ligne de décoration, par exemple une flèche 
pour indiquer la direction de la ligne.
\item \textbf{Ligne de marqueur} : Affichage d'une ligne en répétant un symbole de marqueur.
\item \textbf{Ligne simple} : Affichage d'une ligne pour laquelle il peut être spécifié l'épaisseur, 
la couleur et le type de trait.
\end{itemize}
\item Couches polygones
\begin{itemize}[label=--]
\item \textbf{Remplissage du centroide} : Remplir le centroïde du polygone avec un marqueur définie.
\item \textbf{Remplissage SVG} : Rempli un polygone d'un symbole SVG.
\item \textbf{Remplissage simple} : Affichage d'un polygone pour lequel il peut être spécifié la 
couleur et le motif de remplissage, le type de trait de contour.
\item \textbf{Contour : décoration de ligne} : Ajoute une ligne de décoration, par exemple une 
flèche pour indiquer le sens de la ligne.
\item \textbf{Contour : ligne de marqueur} : Utilise un marqueur défini comme zone de contour.
\item \textbf{Contour : ligne simple} : Définit la largeur, la couleur et le style comme zone 
de contour.
\end{itemize}
\end{itemize}

%\minisec{Color ramps}
\minisec{Palettes de couleurs}

%Color ramps are used to define a range of colors that can be used during
%the creation of renderers. The symbol's color will be set from the color ramp.
Les palettes de couleur sont utilisées pour définir des étendues de couleurs. La 
couleur d'un symbole sera tirée de la palette.

%There are three types of color ramps:
Il y a trois types de palettes :

\begin{itemize}[label=--]
%\item \textbf{Gradient}: Linear gradient from one color to some other.
\item \textbf{Dégradé}: Un dégradé linéaire d'une couleur à une autre.
%\item \textbf{Random}: Randomly generated colors from a specified area of
%color space.
\item \textbf{Aléatoire}: Couleurs générées aléatoirement à partir d'un espace de couleur.
%\item \textbf{ColorBrewer}: Create color area from a color shema and a defined
%number of color classes.
\item \textbf{Mélangeur de couleur}: Création d'un espace de couleur depuis un schéma et un nombre fixe de classes de couleurs.
\end{itemize}

%Color ramps can be defined in the Style Manager} dialog (see Section
%\ref{subsec:stylemanager}) by selecting \\
%\selectstring{Style item type:}{Color ramp} as style element type from the drop-down list, clicking on \button{Add item} button and then choosing a color ramp type.
Les palettes peuvent être créées en passant par l'onglet \tab{Palette de couleur} de la 
fenêtre \dialog{Gestionnaire de style} (voir section \ref{subsec:stylemanager}) 
en cliquant le bouton \button{Ajouter} puis en choisissant le type de palette de couleur.

%\minisec{Styles}
\minisec{Styles}

%A style groups a set of various symbols and color ramps. You can define your
%prefered or frequently used symbols, and can use it  without having to recreate
%it everytime. Style items (symbols and color ramps) have always a name by which
%they can be queried from the style. There is one default style in \qg (modifiable)
%and the user can add further styles.
Un style regroupe un ensemble de symboles et de palettes de couleurs. Vous pouvez 
choisir vos symboles préférés ou les plus utilisés afin de pouvoir les employer de 
nouveau sans devoir les recréer. Ces objets peuvent avoir un nom par lequel il est 
possible de les appeler. Il y a un style au moins par défaut dans \qg qui peut être 
modifié, l'utilisateur pouvant en créer de nouveau.

%\minisec{Renderers}
\minisec{Moteurs de Rendu}

%The renderer is responsible for drawing a feature together with the correct
%symbol. There are three types of renderers: single symbol, categorized (called
%unique color in the old symbology), and graduated. There is no continuous color
%renderer, because it is in fact only a special case of the graduated renderer.
%The categorized and graduated renderer can be created by specifying a symbol
%and a color ramp - they will set the colors for symbols appropriately.
Le moteur de rendu est responsable de l'affichage d'une entité avec le symbole 
adéquat. Là encore, il existe quatre types de rendus : symbole unique, catégorisé 
(nommé couleur unique dans l'ancienne symbologie), gradué et ensemble de règles. Il n'y a pas de rendu 
de couleur continue, car il s'agit d'un cas spécifique du rendu gradué. Les rendus 
catégorisés et gradués peuvent être crées en spécifiant un symbole et une palette 
de couleur - les couleurs des différents symboles changeront selon.

%\subsection{Working with the New Generation Symbology}
\subsubsection{Travailler avec cette nouvelle symbologie}\label{new_generation_sym}

Dans l'onglet \tab{Style} vous pouvez choisir l'un des quatre moteurs de rendu :
symbole unique, catégorisé ou gradué. Selon le moteur de rendu choisi, cette fenêtre 
proposera des options différentes. Un bouton \button{Gestionnaire de style} 
donne accès au gestionnaire de style (voir section \ref{subsec:stylemanager}), 
celui-ci permet d'éditer et de supprimer des symboles existants ou d'en ajouter
de nouveaux.

\begin{Tip}\caption{\textsc{Sélection et modification de plusieurs symboles}}\index{vector layers!symbology} 
La nouvelle génération de la sémiologie permet de sélectionner plusieurs symboles et 
de modifier la couleur, la transparence, la taille ou la largeur du contour avec 
un clic droit des entrées sélectionnées.
\end{Tip}

%\minisec{Single Symbol Renderer}
\minisec{Symboles uniques}

%The Single Symbol Renderer is used to render all features of the layer using a
%single user-defined symbol. The properties, that can be adjusted in the
%Symbology tab, depend partially on the type of the layer, but all types share
%the following structure. In the top left part of the tab, there is a preview of
%the current symbol to be rendered. In the bottom part of the tab, there is a
%list of symbols already defined for the current style, prepared to be used via
%selecting them from the list. The current symbol can be modified using the
%\button{Properties} button, which opens a \dialog{Symbol Properties} dialog, or
%the \button{Set Color} button, which opens an ordinary \dialog{Color} dialog.
%After having done any needed changes, the symbol can be added to the list of
%current style symbols (using the \button{Add to style} button) and then easily
%be used in the future.
Le moteur de rendu pour symbole unique est utilisé pour représenter toutes les 
entités de la couche avec un seul symbole défini par l'utilisateur. Les propriétés, 
qui peuvent être ajustées dans l'onglet \tab{style}, dépendent du type de 
géométrie de la couche, mais partagent une structure similaire. En haut, à gauche 
figure un aperçu du symbole tel qu'il apparaîtra. En bas est affichée la liste 
des symboles déjà existants dans le style courant, prêts à être sélectionnés 
d'un simple clic. Le symbole courant peut être modifié en utilisant le bouton 
\button{Propriétés} en dessous de la prévisualisation, qui ouvre une fenêtre 
\dialog{Propriétés du symbole}, ou en utilisant le bouton \button{Changer} 
à droite de la prévisualisation qui ouvre une boîte de dialogue \dialog{Couleur}.

Dans l'onglet \tab{Style} vous pouvez en plus de la transparence de la couche 
également définir d'utiliser les unités de la carte ou les millimètres pour la 
taille de l'échelle. Vous pouvez également utiliser une rotation et une taille 
d'échelle définie par des champs (disponible par le bouton \button{Avancé} près 
de \button{Sauvegarder le style}). Le bouton \button{Niveau de symbole} permet 
d'activer et de définir l'ordre dans lequel les couches de symboles sont rendues 
(si le symbole consiste de plusieurs couches).

Après que les modifications aient été réalisées, le symbole peut être ajouté 
à la liste des styles de symboles actuels (en utilisant le bouton \button{Sauvegarder 
comme style}) puis utilisé facilement dans le futur. 

\begin{figure}[ht]
\centering
%   \subfloat[Single symbol point properties] {\label{subfig:singleNG1}\includegraphics[clip=true, width=0.3\textwidth]{singlesymbol_ng_point}}
   \subfloat[Propriétés d'un symbole unique de point] {\label{subfig:singleNG1}\includegraphics[clip=true, width=0.3\textwidth]{singlesymbol_ng_point}}
   \hspace{1cm}
%   \subfloat[Single symbol line properties] {\label{subfig:singleNG2}\includegraphics[clip=true, width=0.3\textwidth]{singlesymbol_ng_line}}
   \subfloat[Propriétés d'un symbole unique de ligne] {\label{subfig:singleNG2}\includegraphics[clip=true, width=0.3\textwidth]{singlesymbol_ng_line}}
   \hspace{1cm}
%   \subfloat[Single symbol area properties] {\label{subfig:singleNG3}\includegraphics[clip=true, width=0.3\textwidth]{singlesymbol_ng_area}}
   \subfloat[Propriétés d'un symbole unique de surface] {\label{subfig:singleNG3}\includegraphics[clip=true, width=0.3\textwidth]{singlesymbol_ng_area}}
%\caption{New Single Symbolizing options \nixcaption}
\caption{Options pour les symboles uniques \nixcaption}
\end{figure}

%\minisec{Categorized Renderer}
\minisec{Symboles catégorisés}

%The Categorized Renderer is used to render all features from a layer, using a
%single user-defined symbol, which color reflects the value of a selected
%feature's attribute. The Symbology tab allows you to select:
Le rendu catégorisé est utilisé pour afficher toutes les entités d'une couche 
en recourant à un symbole défini par l'utilisateur dont la couleur reflètera 
la valeur d'un attribut donné (p. ex., réprésenter avec un fond rouge tous 
les polygones ayant un attribut "danger"). L'onglet \tab{Style} vous permet de 
sélectionner :

\begin{itemize}[label=--]
%\item The attribute (using the Column listbox)
\item l'attribut (en utilisant la liste de colonne)
%\item The symbol (using the Symbol dialog)
\item le symbole (en utilisant la boîte de dialogue des symboles)
%\item The colors (using the Color Ramp listbox)
\item la couleur (en utilisant la liste de rampes/palettes de couleur)
\end{itemize}

%The Advanced button in the lower right corner of the dialog allows to set
%the fields containing rotation and size scale information.
%For convenience, the list in the bottom part of the tab lists the values of
%all currently selected attributes together, including the symbols that will
%be rendered.
Le bouton \button{Avancé} dans le coin inférieur droit de la fenêtre permet de 
définir les champs contenant les informations relatives à la rotation et la 
proportion.
Pour faciliter les choix de représentation, la liste dans la partie inférieure 
de la fenêtre affiche les valeurs de tous les attributs actuellement 
sélectionnés, ce qui inclut les symboles qui seront affichés.

%The example in figure \ref{fig:catsymNG} shows the category rendering dialog
%used for the rivers layer of the \qg sample dataset.
L'exemple de la figure \ref{fig:catsymNG} montre le rendu des catégories de 
la couche des rivières de l'échantillon de données de \qg.

\begin{figure}[ht]
   \centering
%   \caption{New Categorized Symbolizing options \nixcaption}\label{fig:catsymNG}
   \caption{Options de catégorisation des symboles \nixcaption}\label{fig:catsymNG}
   \includegraphics[clip=true, width=10cm]{categorysymbol_ng_line}
\end{figure}

%You can create a custom color ramp choosing New color ramp... from the Color 
%ramp dropdown menu. A dialog will prompt for the ramp type: Gradient, Random,
%ColorBrewer, then each one has options for number of steps and/or multiple
%stops in the color ramp. See \ref{fig:ccrg} for an example of custom color
%ramp.

Vous pouvez créer une palette de couleur personnalisée en cliquant sur 
'Nouvelle palette de couleur' dans la liste déroulante des rampes/palettes. 
Une fenêtre vous demandera de choisir le type : gradué, aléatoire, mélangeur 
de couleur (ColorBrewer). Chacun des ces choix à sa propre série d'options, 
observez la figure \ref{fig:ccrg} pour un exemple de palette personnalisée.

\begin{figure}[ht]
   \centering
%   \caption{Example of custom gradient color ramp with multiple stops \nixcaption}\label{fig:ccrg}
   \caption{Exemple de palette de couleur graduée avec plusieurs arrêts \nixcaption}\label{fig:ccrg}
   \includegraphics[clip=true, width=8cm]{customColorRampGradient.png}
\end{figure}

%\minisec{Graduated rendering}
\minisec{Symboles gradués}

%The Graduated Renderer is used to render all the features from a layer, using
%a single user-defined symbol, whose color reflects the classification of a selected
%feature's attribute to a class. Like Categorized Renderer, it allows to define
%rotation and size scale from specified columns.
Le rendu gradué est utilisé pour afficher toutes les entités d'une couche, en 
utilisant un symbole de couche défini par l'utilisateur dont la couleur reflètera 
la plage d'appartenance d'une valeur d'un attribut (p. ex. une plage d'altitude 
de 0 à 100 m).

%Analogue to the categorized rendered, the symbology tab allows you to select:
De la même façon que le rendu catégorisé, l'onglet \tab{Style} vous permet de 
modifier les possibilités suivantes :

\begin{itemize}[label=--]
%\item The attribute (using the Column listbox)
\item l'attribut (en utilisant la liste de colonne)
%\item The symbol (using the Symbol Properties dialog)
\item le symbole (en utilisant le bouton Symbole)
%\item The colors (using the Color Ramp listbox)
\item la couleur (en utlisant la liste de rampe de couleur)
\end{itemize}

%Additionally, you can specify the number of classes and also the mode how to
%classify features inside the classes (using the Mode list). The available modes are:
De plus, vous pouvez choisir le nombre de classes et la méthode de classification. Les modes disponibles sont :

\begin{itemize}
% \item Equal Interval
\item Intervalles égaux
% \item Quantile
\item Quantiles
% \item Natural Breaks (Jenks)
\item Seuils naturels (Jenks)
% \item Standard Deviation
\item Déviation standard
% \item Pretty Breaks
\item Jolie rupture \footnote{Pour plus d'informations sur cette méthode : http://astrostatistics.psu.edu/datasets/R/html/base/html/pretty.html}
\end{itemize}

%The listbox in the  bottom part of the symbology tab lists the classes together with their ranges,
%labels and symbols that will be rendered.
La liste de la partie inférieure de l'onglet \tab{Style} indique les classes 
ainsi que leurs plages, étiquettes et symboles.

%The example in figure \ref{fig:gradsymNG} shows the graduated rendering dialog
%for the rivers layer of the \qg sample dataset.
L'exemple de la figure \ref{fig:gradsymNG} montre le rendu gradué de la couche des rivières de l'échantillon de données de \qg.

\begin{figure}[ht]
   \centering
   \includegraphics[clip=true, width=10cm]{graduatesymbol_ng_line}
%   \caption{New Graduated Symbolizing options \nixcaption}\label{fig:gradsymNG}
   \caption{Options des symboles gradués \nixcaption}\label{fig:gradsymNG}
\end{figure}

%\minisec{Rule-based rendering}
\minisec{Rendu basé sur des règles}

%The rule-based renderer is used to render all the features from a layer, using
%rule based symbols, whose color reflects the classification of a selected
%feature's attribute to a class.
Ce moteur de rendu est utilisé pour afficher toutes les entités d'une 
couche en utilisant un ensemble de règles prédéfinies. Les règles sont basées 
sur des requêtes SQL. Vous pouvez également utiliser le constructeur de 
requête pour les créer. La boîte de dialogue permet de grouper les règles 
par filtre ou par échelle et vous pouvez choisir d'activer les niveaux de 
symboles ou utiliser la première règle qui s'applique.

%The example in figure \ref{fig:rulesymNG} shows the rule-based rendering dialog
%for the rivers layer of the \qg sample dataset.
L'exemple de la figure \ref{fig:rulesymNG} montre le rendu basé sur des règles 
de la couche des rivières de l'échantillon de données de \qg.

\begin{figure}[ht]
   \centering
   \includegraphics[clip=true, width=10cm]{rulesymbol_ng_line}
%   \caption{New Rule-based Symbolizing options \nixcaption}\label{fig:rulesymNG}
   \caption{Options basé sur des régles \nixcaption}\label{fig:rulesymNG}
\end{figure}

%\minisec{Point displacement}
\minisec{Déplacement de point}

%The point displacement renderer offers to visualize all features of a point
%layer, even if they have the same location. To do this, the symbols of the
%points are placed on a displacement circle around a center symbol.
Le rendu de déplacement de point est seulement disponible si vous chargez
l'extension à partir de l'installeur d'extension de QGIS. Il offre une 
visualisation de tous les points d'une couche, même si ceux-ci se 
superposent. Pour se faire, les symboles des points sont répartis en 
cercle autour d'un symbole central. Vous pouvez cumuler ce type de 
rendu avec tous ceux vus précédemment pour, par exemple, afficher les 
différents services (via une catégorisation) d'une station sans avoir 
à déplacer les points à la main.

%TODO
%\begin{figure}[ht]
%   \centering
%   \includegraphics[clip=true, width=10cm]{poi_displacement}
%%   \caption{Point displacement dialog \nixcaption}\label{fig:poidissymNG}
%   \caption{Fenêtre de déplacement de points \nixcaption}\label{fig:poidissymNG}
%\end{figure}

%\minisec{Symbol Properties}
\minisec{Propriétés du symbole}

%The symbol properties dialog allows the user to specify different properties of
%the symbol to be rendered. In the top left part of the dialog, you find a preview
%of the current symbol as it will be displayed in the map canvas. Below the preview
%is the list of symbol layers. To start the symbol properties dialog, click the
%\dropmenuopttwo{mActionOptions}{Properties} button in the \tab{Symbology} tab of the
%\dialog{Layer Properties} dialog.
Le panneau des propriétés permet à l'utilisateur de spécifier les différentes 
propriétés du symbole qui sera affiché. En haut à gauche figure un aperçu du 
symbole tel qu'il sera affiché sur le canevas de la carte. Juste en dessous est 
placée une liste des différentes couches du symbole. Pour accéder à cette fenêtre, 
cliquez sur le bouton \dropmenuopttwo{mActionOptions}{Propriétés} de l'onglet 
\tab{Style} de la fenêtre des \dialog{Propriétés de la couche}.

%The control panels allow adding or removing layers, changing the position of layers,
%or locking layers for color changes. In the right part of the dialog, there are
%shown the settings applicable to the single symbol layer selected in the symbol
%layer list. The most important is the 'Symbol Layer Type' combo box, which allows
%you to choose the layer type. The available options depend on the layer type
%(Point, Line, Polygon).

Vous pouvez ajouter ou supprimer des couches, changer l'ordre de position des 
couches ou en verrouiller. Dans la partie droite sont affichés les paramètres 
relatifs à la couche de symbole sélectionnée, le plus important est sans doute 
la liste 'Type de Symbole' qui permet de choisir le type de la couche. Les options 
disponibles dépendent du type d'entités (Point, Ligne, Polygone). Les options des 
symboles des types de couches sont décrites à la section \ref{symboltypes}.

\begin{figure}[ht]
\centering
%   \subfloat[Line composed from three simple lines] {\label{subfig:symprops1}\includegraphics[clip=true, width=0.3\textwidth]{symbolproperties1}}
   \subfloat[Ligne composée de 3 lignes simples] {\label{subfig:symprops1}\includegraphics[clip=true, width=0.3\textwidth]{symbolproperties1}}
   \hspace{1cm}
%   \subfloat[Symbol properties for point layer] {\label{subfig:symprops2}\includegraphics[clip=true, width=0.3\textwidth]{symbolproperties2}}
   \subfloat[Propriétés de symbole de point] {\label{subfig:symprops2}\includegraphics[clip=true, width=0.3\textwidth]{symbolproperties2}}
   \hspace{1cm}
%   \subfloat[Filling pattern for a polygon] {\label{subfig:symprops3}\includegraphics[clip=true, width=0.3\textwidth]{symbolproperties3}}
   \subfloat[Motif de remplissage d'un polygone] {\label{subfig:symprops3}\includegraphics[clip=true, width=0.3\textwidth]{symbolproperties3}}
%\caption{Defining symbol properties \nixcaption}
\caption{Définir les propriétés d'un symbole \nixcaption}
\end{figure}

%\subsection{Style Manager to manage symbols and color ramps}\label{subsec:stylemanager}
\subsubsection{Gestionnaire de styles pour gérer les symboles et les palettes de couleur}\label{subsec:stylemanager}

%The Style Manger is a small helper application, that lists symbols and color
%ramps available in a style. It also allows you to add and/or remove items. To
%launch the Style Manager, click on \mainmenuopt{Settings} \arrow \dropmenuopt{Style
%Manager} in the main menu.
Le gestionnaire de styles est une petite application qui liste les symboles et les 
palettes de couleurs disponibles dans un style. Pour l'utiliser, cliquez sur 
\mainmenuopt{Préférences} \arrow \dropmenuopt{Gestionnaire de Style} dans le 
menu principal.

\begin{figure}[ht]
   \centering
   \includegraphics[clip=true, width=5.5cm]{stylemanager}
%   \caption{Style Manager to manage symbols and color ramps \nixcaption}
%\label{fig:stylemanager}
   \caption{Gestion des symboles et des palettes de couleurs \nixcaption}
   \label{fig:stylemanager}
\end{figure}

%\subsection{Symbology Tab}\label{sec:symbology}
%\index{vector layers!symbology}
\subsection{Onglet Convention des signes}\label{sec:symbology}
\index{couches vectorielles!symbologie}

%\qg supports a number of symbology renderers to control how vector features 
%are displayed. Currently the following renderers are available:
\qg gère différents types de représentation cartographique pour contrôler la 
manière pour les entités vectorielles seront affichées. Actuellement, voici 
les possibilités de base (d'autres plus avancées sont amenées à les remplacer, 
voir \ref{working_with_new_symbology}) :

\begin{description}
%\item[Single symbol] - a single style is applied to every object in the layer.
%\index{vector layers!renderers!single symbol}
\item \textbf{Symbole unique:}   un style unique est appliqué à tous les objets 
de la couche.\index{couches vectorielles!rendus!symbole unique}
%\item[Graduated symbol] - objects within the layer are displayed with different symbols classified by the values of a particular field.\index{vector layers!renderers!graduated symbol}
\item \textbf{Symbole gradué:}  les objets de la couche sont représentés avec 
des symboles différents selon la valeur qu'ils ont dans un champ défini.
\index{couches vectorielles!rendus!symbole gradué}
%\item[Continuous color] - objects within the layer are displayed with a spread of colours classified by the numerical values within a specified field.\index{vector layers!renderers!continuous color}
\item \textbf{Couleur continue:} les objets de la couche sont représentés avec 
une échelle de couleurs classées selon les valeurs numériques d'un champ défini.
\index{couches vectorielles!rendus!couleur continue}
%\item[Unique value] - objects are classified by the unique values within a specified field with each value having a different symbol.\index{vector layers!renderers!unique value}
\item \textbf{Valeur unique:}  les objets sont classés par valeur unique dans 
un champ défini et à chaque valeur correspond un symbole différent.\index{couches 
vectorielles!rendus!valeur unique}
\end{description}

%To change the symbology for a layer, simply double click on its legend entry and the vector \dialog{Layer Properties} dialog will be  shown.\index{symbology!changing}
Pour changer la symbologie d'une couche, double-cliquez simplement dessus dans 
la légende et la fenêtre de \dialog{Propriétés de la couche} apparaîtra.
\index{symbologie!changer}

\begin{figure}[p]
\centering
  %\subfigure[Single symbol] {\label{subfig:single_symbol}\includegraphics[clip=true, width=0.4\textwidth]{vectorClassifySingle}}\goodgap
  \subfloat[Symbole unique] {\label{subfig:single_symbol}\includegraphics[clip=true, width=0.48\textwidth]{vectorClassifySingle}}
  %\subfigure[Graduated symbol] {\label{subfig:graduated_symbol}\includegraphics[clip=true, width=0.4\textwidth]{vectorClassifyGraduated}}\\
\hspace{0.1cm}
  \subfloat[Symbole gradué] {\label{subfig:graduated_symbol}\includegraphics[clip=true, width=0.48\textwidth]{vectorClassifyGraduated}}\\
  %\subfigure[Continous color] {\label{subfig:cont_color}\includegraphics[clip=true, width=0.4\textwidth]{vectorClassifyContinous}}\goodgap
  \subfloat[Couleur continue] {\label{subfig:cont_color}\includegraphics[clip=true, width=0.48\textwidth]{vectorClassifyContinous}}
  %\subfigure[Unique value] {\label{subfig:unique_val}\includegraphics[clip=true, width=0.4\textwidth]{vectorClassifyUnique}}
  \hspace{0.1cm}
  \subfloat[Valeur unique] {\label{subfig:unique_val}\includegraphics[clip=true, width=0.48\textwidth]{vectorClassifyUnique}}
  \caption{Options de symbolisation \nixcaption}
\end{figure}

%\minisec{Style Options} \label{sec:style_options} \index{vector layers!styles}
\minisec{Options de style} \label{sec:style_options}\index{couches vectorielles!styles}
%Within this dialog you can style your vector layer. Depending on the selected rendering option you have the possibility to also classify your mapfeatures.
Dans cette fenêtre vous pouvez donner un style à votre couche vecteur. Selon 
l'option de rendu sélectionnée, vous avez la possibilité de classer vos entités.

%At least the following styling options apply for nearly all renderers:
Les options de style suivantes s'appliquent quasiment à tous les types de rendus :
\begin{description}
%\item[Outline style] - pen-style for your outline of your feature. you can also set this to 'no pen'.
\item \textbf{Style de la bordure:} style de la ligne qui fait le contour de 
vos entités. Vous pouvez également le définir à \selectstring{Style de la 
bordure:}{Pas de crayon} ce qui ne fera apparaître aucun trait pour la bordure.
%\item[Outline color] - color of the ouline of your feature
\item \textbf{Couleur de la bordure:} couleur du contour de vos entités.
%\item[Outline width] - width of your features
\item \textbf{Largeur de la bordure:} épaisseur du contour de vos entités.
%\item[Fill color] - fill-color of your features.
\item \textbf{Couleur de remplissage:} couleur de remplissage de vos entités.
%\item[Fill style] - Style for filling. Beside the given brushes you can select \selectstring{Fill style}{? texture} and click the \browsebutton button for selecting your own fill-style. Currently the fileformats \filename{*.jpeg, *.xpm, and *.png} are supported.
\item \textbf{Style de remplissage:} en plus des pinceaux proposés, vous pouvez 
sélectionner \selectstring{Style de remplissage:}{texture} et cliquer sur le 
bouton \browsebutton pour sélectionner votre propre style de remplissage. 
Actuellement, les formats de fichier \filename{*.svg, *.jpeg, *.xpm et *.png} 
sont supportés.
\end{description}

%Once you have styled your layer you also could save your layer-style to a separate file (with \filename{*.qml}-ending). To do this, use the button \button{Save Style \ldots}. No need to say that \button{Load Style \ldots} loads your saved layer-style-file.
Une fois que vous avez défini le style de votre couche, vous pouvez le 
sauvegarder dans un fichier séparé (avec l'extension \filename{*.qml}). Pour 
faire cela, utilisez le bouton \button{Sauvegarder le style \ldots} Inutile de 
dire que \button{Charger le style \ldots} charge vos fichiers sauvegardés.

%If you wish to always use a particular style whenever the layer is loaded, use the \button{Save As Default} button to make your style the default. Also, if you make changes to the style that you are not happy with, use the \button{Restore Default Styel} button to revert to your default style.
Si vous voulez utiliser en permanence un style particulier chaque fois que la 
couche est chargée, utilisez le bouton \button{Sauvegarder comme défaut} pour 
en faire le style par défaut. Aussi, si le style ne vous plait pas et que vous 
le modifiez, utilisez le bouton \button{Restaurer le style par défaut} pour en 
faire votre style par défaut.

%\minisec{Vector transparency} \label{sec:vect_transparency}
%\index{vector layers!transparency}
\minisec{Transparence d'une couche vectorielle} \label{sec:vect_transparency} 
\index{couches vectorielles!transparence}

%\qg allows to set a transparency for every vector layer. This can be done with
%the slider \\
%\slider{Transparency} inside the \tab{symbology} tab (see
%fig. \ref{subfig:single_symbol}). This is very useful for overlaying several
%vector layers.
\qg permet de définir une transparence pour chaque couche vecteur. Ceci peut-être 
fait avec le curseur \slider{Transparence} de l'onglet \tab{Convention des signes} 
(voir fig. \ref{fig:vector_symbology}). Ceci est très utile pour superposer 
plusieurs couches vectorielles sur un même canevas.

%\subsection{Labels Tab}
\subsection{Onglet Étiquettes}

%As for the symbology \qg 1.7.0 currently provides an old and a new labeling 
%engine in parallel. The \tab{Labels} tab still contains the old labeling. The 
%new labeling is implemented as a core application and will replace the features 
%of the old labels tab in one of the next versions.
De même que pour le style, \qg 1.7.0 fournit actuellement un ancien et un nouveau 
moteur d'étiquette en parallèle. L'onglet \tab{Étiquette} contient toujours 
l'ancien système d'étiquetage. Le nouveau moteur d'étiquetage a été implémenté 
au coeur de l'application et remplacera les fonctionnalités de l'ancien onglet 
d'étiquette dans une future version.

%We recommend to switch to the new labeling, described in section \ref{newlabel}.
Nous recommandons de passer au nouveau système d'étiquetage, décrit dans la 
section \ref{newlabel}.

%The old labeling in the \tab{Labels} tab allows you to enable labeling features 
%and control a number of options related to fonts, placement, style, alignment 
%and buffering. We will illustrate this by labelling the lakes shapefile of the
L'ancien système d'étiquetage dans l'onglet \tab{Étiquettes} vous permet d'activer 
les fonctionnalités d'étiquetage et de gérer plusieurs options liées à la police 
de caractère, au placement, au style, à l'alignement et de buffer.

%We will illustrate this by labelling the lakes shapefile of the \filename{qgis\_example\_dataset}:
Nous allons illustrer tout cela en étiquetant le shapefile des lacs du jeu de 
données\\ \filename{qgis\_example\_dataset} :

\begin{enumerate}
%\item Load the Shapefile \filename{alaska.shp} and GML file \filename{lakes.gml} in \qg.
\item Charger le shapefile \filename{alaska.shp} et le fichier GML \filename{lakes.gml} dans \qg
%\item Zoom in a bit to your favorite area with some lake.
\item Zoomez légèrement sur votre coin préféré avec quelques lacs
%\item Make the \filename{lakes} layer active.
\item Rendez active la couche \filename{lakes}
%\item Open the \dialog{Layer Properties} dialog.
\item Ouvrez la fenêtre \dialog{Propriétés de la couche}
%\item Click on the \tab{Labels} tab.
\item Cliquez sur l'onglet \tab{Étiquettes}
%\item Check the \checkbox{Display labels} checkbox to enable labeling.
\item Cochez la case \checkbox{Afficher les étiquettes} pour activer l'étiquetage
%\item Choose the field to label with. We'll use \selectstring{Field containing 
%label}{NAMES}.
\item Choisissez le champ à utiliser pour les étiquettes. Ici, nous utiliserons 
le\\ \selectstring{Champ contenant une étiquette}{NAMES}
%\item Enter a default for lakes that have no name. The default label will be 
%used each time \qg encounters a lake with no value in the \guilabel{NAMES} field.
\item Choisissez un libellé par défaut pour les lacs n'ayant pas de nom. Ce 
libellé sera utilisé chaque fois que \qg rencontre un lac n'ayant pas de valeur 
dans le champ \guilabel{NAMES}
\item Si des étiquettes s'étendent sur plusieurs lignes, cochez 
\checkbox{Etiquettes multilignes ?} \qg cherchera un retour à la ligne dans le 
champ de l'étiquette pour insérer une rupture en accord. Un véritable retour à 
la ligne est un caractère \textbf{unique} \textbackslash n, (et non pas 2 
caractères séparés comme un antislash \textbackslash ~suivi par un n).  Pour 
insérer un retour à la ligne dans un champ attributaire configurez le panel 
d'édition afin qu'il soit un éditeur de texte (et pas de ligne).
%\item Click \button{Apply}.
\item Cliquez sur \button{Appliquer}
\end{enumerate}

%Now we have labels. How do they look? They are probably too big and poorly 
%placed in relation to the marker symbol for the lakes.
Maintenant, nous avons des étiquettes. De quoi ont-elles l'air ? Elles sont 
probablement trop grandes et mal placées par rapport au symbole marqueur des lacs.

%Select the \tab{Font} entry and use the \button{Font} and \button{Color} buttons 
%to set the font and color. You can also change the angle and the placement of 
%the text-label.
Sélectionnez l'entrée \tab{Police} et utilisez les boutons \button{Police} et 
\button{Couleur} pour définir la police et la couleur. Vous pouvez également 
changer l'angle et le placement de l'étiquette.

%To change the position of the text relative to the feature:
Pour changer la position du texte par rapport à l'entité :

\begin{enumerate}
%\item Click on the \tab{Font} entry.
\item Cliquez sur l'entrée \tab{Police}
%\item Change the placement by selecting one of the radio buttons in the 
\classname{Placement} group. To fix our labels, choose the \radiobuttonon{Right} radio button.
\item Changer le placement en sélectionnant l'un des boutons radio dans le groupe 
\classname{Placement}. Pour corriger nos étiquettes, choisissez le bouton radio 
\radiobuttonon{Droite}
%\item the \classname{Font size units} allows you to select between 
%\radiobuttonon{Points} or \radiobuttonon{Map units}.
\item La \classname{Taille de la police des unités} vous permet de choisir entre 
des \radiobuttonon{Points} ou des \radiobuttonon{Unités de carte}
%\item Click \button{Apply} to see your changes without closing the dialog.
\item Cliquez sur \button{Appliquer} pour visualiser les changements sans fermer 
la fenêtre
\end{enumerate}

%Things are looking better, but the labels are still too close to the marker. To 
%fix this we can use the options on the \tab{Position} entry. Here we can add 
%offsets for the X and Y directions. Adding an X offset of 5 will move our 
%labels off the marker and make them more readable. Of course if your marker 
%symbol or font is larger, more of an offset will be required.
Ça à l'air plus joli, mais les étiquettes sont encore trop proches des marqueurs. 
Pour corriger cela, nous pouvons utiliser les options de l'entrée \tab{Position}. 
Ici, nous pouvons ajouter un décalage dans les directions X et Y. Ajouter un 
décalage de 5 en X déplacera vos étiquettes et les rendra plus lisibles. Bien 
sûr si vos symboles marqueurs ou votre police sont plus grands un décalage plus 
important sera nécessaire.

%The last adjustment we'll make is to \tab{buffer} the labels. This just means 
%putting a backdrop around them to make them stand out better. To buffer the 
%lakes labels:
Un dernier ajustement reste à faire sur les étiquettes : un \tab{tampon}. Il 
s'agit de créer un arrière-plan autour des étiquettes pour les faire mieux ressortir. 
Pour faire un tampon sur les étiquettes des lacs :

\begin{enumerate}
%\item Click the \checkbox{Buffer Labels?} checkbox to enable buffering.
\item Cliquez sur la case à cocher \checkbox{Tampon d'étiquette ?} pour activer 
le tampon
%\item Choose a size for the buffer using the spin box.
\item Choisissez une taille de tampon en utilisant les flèches
%\item Choose a color by clicking on \button{Color} and choosing your favorite 
%from the color selector. You can also set some transparency for the  buffer if 
%you prefer.
\item Choisissez une couleur en cliquant sur \button{Couleur} puis choisissez 
votre couleur favorite grâce au sélecteur. Si vous le souhaitez, vous pouvez 
également ajouter un peu de transparence au tampon
%\item Click \button{Apply} to see if you like the changes.
\item Cliquez sur \button{Appliquer} pour voir si les changements vous plaisent
\end{enumerate}

%If you aren't happy with the results, tweak the settings and then test again by 
%clicking \button{Apply}.
Si le résultat ne vous plaît pas, ajustez les paramètres et re-testez en c
liquant sur \button{Appliquer}

%A buffer of 1 points seems to give a good result. Notice you can also specify 
%the buffer size in map units if that works out better for you.
Le tampon d'une taille d'un point semble donner un bon résultat. Notez que vous 
pouvez également spécifier une taille de tampon en unités de la carte si cela 
vous convient mieux.

%The remaining entries inside the \tab{Label} tab allow you control the 
%appearance of the labels using attributes stored in the layer. The entries 
%beginning with \tab{Data defined} allow you to set all the parameters for the 
%labels using fields in the layer.
Les autres entrées de l'onglet \tab{Étiquettes} vous permettent de contrôler 
l'apparence des étiquettes en utilisant les attributs stockés dans la couche. 
Les entrées commençantes par \tab{Data defined} vous permettent de définir tous 
les paramètres des étiquettes en utilisant des champs de la couche.

%Not that the \tab{Label} tab provides a \classname{preview-box} where your 
%selected label is shown.
Notez que l'onglet \tab{Étiquettes} propose une \classname{Prévisualisation} 
montrant une de vos étiquettes.

%/////////////////////////////////////////////
%/////////////////////////////////////////////
%/////////////////////////////////////////////
%/////////////////////////////////////////////

%\subsubsection{New Labeling}\index{New labeling}\label{newlabel}
\subsubsection{Nouvel Étiquetage}\index{Nouvel étiquetage}\label{newlabel}

%The new \toolbtntwo{labeling}{Labeling} core application provides smart labeling
%for vector point,  line and polygon layers and only requires a few parameters.
%This new application will replace the current QGIS labeling, described in section
%\ref{labeltab} and also supports on-the-fly transformated layers.
Le nouvel \toolbtntwo{labeling}{Étiquetage} fournit un placement intelligent des 
étiquettes de vecteurs avec quelques paramètres. Cette version remplacera à terme 
l'actuel système (présenté dans la section \ref{labeltab}). 

%\minisec{Using new labeling}
\minisec{Utiliser le nouvel étiquetage}

\begin{enumerate}
%  \item Start QGIS and load a vector point, line or polygon layer.
  \item  Lancez \qg et charger une couche vectorielle
%  \item Activate the layer in the legend and click on the
%  \toolbtntwo{labeling}{Labeling} icon in the QGIS toolbar menu.
  \item Activerla couche dans la légende et cliquez sur le bouton 
  \toolbtntwo{labeling}{Étiquetage} dans la barre d'outils.
\end{enumerate}

%\minisec{Labeling point layers}
\minisec{Étiqueter une couche de points}

%First step is to activate the \checkbox{Label this layer} checkbox and select an attribute
%column to use for labeling. After that you can define the label placement and text style,
%labeling priority, scale-based visibility, if every part of multipart feature is to be
%labeled and if features act as obstacles for labels or not (see
%Figure \ref{fig:pointlabel}).
Le premiers pas est de cocher la case \checkbox{Étiqueter cette couche} et sélectionner un attribut à afficher dans la liste déroulante. Après cela vous pouvez définir le positionnement de l'étiquette et le style du texte, l'échelle de visualisation, l'affichage d'une étiquette pour chaque partie d'une entité multipartite et si \ref{fig:pointlabel}).

\begin{figure}[ht]
\centering
   \includegraphics[clip=true, width=10cm]{label_points}
%   \caption{Smart labeling of vector point layers \nixcaption}\label{fig:pointlabel}
   \caption{Étiquetage intelligent d'une couche de points \nixcaption}\label{fig:pointlabel}
\end{figure}

%\minisec{Labeling line layers}
\minisec{Étiqueter une couche de ligne}

%First step is to activate the \checkbox{Label this layer} checkbox and select an attribute
%column to use for labeling. After that you can define the label placement, orientation,
%distance to feature, text style, labeling priority, scale-based visibility, if every part
%of a multipart line is to be labeled, if lines shall be merged to avoid duplicate labels
%and if features act as obstacles for labels or not (see Figure \ref{fig:linelabel}).
Cochez la case \checkbox{Étiqueter cette couche} et sélectionnez un attribut à afficher dans la liste déroulante. Après cela vous pourrez définir le positionnement de l'étiquette et le style du texte, l'orientation, le décalage par rapport à l'entité, l'échelle de visualisation, l'affichage d'une étiquette pour chaque partie d'une ligne multipartite, la fusion des lignes pour éviter d'avoir des doublons et si les entités doivent agir comme des obstacles pour les étiquettes (voir figure \ref{fig:linelabel}).

\begin{figure}[ht]
\centering
   \includegraphics[clip=true, width=10cm]{label_line}
%   \caption{Smart labeling of vector line layers \nixcaption}\label{fig:linelabel}
   \caption{Étiquetage intelligent d'une couche de lignes \nixcaption}\label{fig:linelabel}
\end{figure}

%\minisec{Labeling polygon layers}
\minisec{Étiqueter une couche de polygone}

%First step is to activate the \checkbox{Label this layer} checkbox and select an attribute
%column to use for labeling. After that you can define the label placement, distance and text
%style, labeling priority, scale-based visibility, if every part of multipart feature is to be
%labeled and if features act as obstacles for labels or not (see Figure \ref{fig:arealabel}).
Cochez la case \checkbox{Étiqueter cette couche} et sélectionnez un attribut à 
afficher dans la liste déroulante. Après cela vous pourrez définir le 
positionnement de l'étiquette et le style du texte, l'échelle de visualisation, 
l'affichage d'une étiquette pour chaque partie d'une ligne multipartite, la 
fusion des lignes pour éviter d'avoir des doublons et si les entités doivent 
agir comme des obstacles pour les étiquettes (voir figure \ref{fig:arealabel}).

\begin{figure}[ht]
\centering
   \includegraphics[clip=true, width=10cm]{label_area}
%   \caption{Smart labeling of vector polygon layers \nixcaption}\label{fig:arealabel}
   \caption{Étiquetage intelligent d'une couche de polygone \nixcaption}\label{fig:arealabel}
\end{figure}

%\minisec{Change engine settings}
\minisec{Modifier le paramètrage du moteur}

%Additionally you can click the \button{Engine settings} button and select the search method,
%used to find the best label placement. Available is Chain, Popmusic Tabu, Popmusic Chain,
%Popmusic Tabu Chain and FALP.
Vous pouvez en plus cliquez sur le bouton \button{Paramètrage du moteur} et 
sélectionner la méthode recherche utilisée pour trouver le meilleur placement de 
l'étiquette. Chaine, Popmusic Tabu, Popmusic Chain, Popmusic Tabu Chain et FALP.

\begin{figure}[ht]
\centering
   \includegraphics[clip=true, width=5cm]{label_engine}
%   \caption{Dialog to change label engine settings \nixcaption}\label{fig:labelengine}
\end{figure}

%Furthermore the number of candidates can be defined for point, line and polygon features,
%and you can define whether to show all labels (including colliding labels) and label
%candidates for debugging.
Le nombre de candidats peut aussi être défini pour les points, lignes et 
polygones, ainsi que choisir d'afficher toutes les étiquettes (dont les étiquettes 
en collision) et les candidats d'étiquettes pour débogage.

%\minisec{Keywords to use in attribute columns for labeling}
\minisec{Mots-clés à utiliser dans les colonnes attributaires pour les étiquettes}

%There is a list of supported key words, that can be used for the placement of 
%labels in defined attribute colums?
Il y a une liste de mots-clés gérés qui peut être utilisée pour le placement 
d'étiquettes dans des colonnes définies.

\begin{itemize}[label=--]
%\item \textbf{For horizontal alignment}: left, center, right
%\item \textbf{For vertical alignment}: bottom, base, half, top
%\item \textbf{Colors can be specified in svg notation}, e.g. \#ff0000
%\item \textbf{for bold, underlined, strikeout and italic}: 0 = false 1 = true
\item \textbf{Pour l'alginement horizontal} : left, center, right
\item \textbf{Pour l'alignement vertical} : bottom, base, half, top
\item \textbf{Couleur qui peut être définie en notation svg}, par exemple \#ff0000
\item \textbf{Pour le gras, le soulignement, le barré et l'italique} : 0 = false 1 = true
\end{itemize}

%A combination of key words in one column also works, e.g.: base right or bottom left.
Une combinaison de mots-clés dans une colonne fonctionne aussi, par exemple base 
right ou bottom left.

%\subsection{Attributes Tab}\index{attributes}\label{label_attributes}
\subsection{Onglet attributs}\index{attributs}\label{label_attributes}

%Within the \tab{Attributes} tab the attributes of the selected dataset can be 
%manipulated. The buttons \toolbtntwo{mActionNewAttribute}{New Column} and 
%\toolbtntwo{mActionDeleteAttribute}{Delete Column} can be used, when the dataset 
%is \toolbtntwo{mActionToggleEditing}{editing mode}. At the moment only columns 
%from \pg layers can be removed and added. The OGR library supports to add new 
%columns, but not to remove them, if you have a GDAL version >= 1.6 installed. 
%In the GDAL/OGR trac there is a ticket with a patch that
%awaits to be committed (\url{http://trac.osgeo.org/gdal/ticket/2671}). Until then QGIS
%(and any other software that uses GDAL/OGR) can only use a workaround to delete
%Shapefile columns. In QGIS this ``workaround'' is a third-party plugin called
%Table Manager.
Dans l'onglet \tab{Attributs}, il est possible de manipuler les attributs du jeu 
de données sélectionné. Les boutons \toolbtntwo{mActionNewAttribute}{Ajouter une 
colonne} et \toolbtntwo{mActionDeleteAttribute}{Supprimer une colonne} peuvent 
être utilisés lorsque le jeu de données est en mode édition. Actuellement, 
seules les colonnes des couches \pg peuvent être effacées ou ajoutées. La 
bibliothèque OGR, dans les versions $\geq$ à la 1.6, supporte l'ajout de 
nouvelles colonnes, mais pas la suppression. Dans le trac de GDAL/OGR il y a un 
ticket avec un patch en attente d'intégration (\url{http://trac.osgeo.org/gdal/ticket/2671}). 
Jusqu'alors, QGIS (et tous les logiciels qui utilisent GDAL/OGR) peut seulement utilisé une solution de contournement pour supprimer les colonnes Shapefiles. 
Dans \qg cette solution est une extension externe appelée Gestionnaire de table.

%The \button{Toggle editing mode} button toggles this mode.
Le bouton \button{Basculer en mode édition} permet de passer dans ce mode.

%\minisec{edit widget}
\minisec{Outils d'édition}

\begin{figure}[H]
   \begin{center}
   \includegraphics[clip=true, width=12cm]{editwidgetsdialog}
   \caption{Dialogue pour la sélection d'un widget d'édition pour une colonne 
    attributaire \nixcaption}\label{fig:editwidget}
\end{center}
\end{figure}

%Within the \tab{Attributes} tab you also find an \texttt{edit widget} and a 
%\texttt{value} column. These two columns can be used to define values or a 
%range of values that are allowed to be added to the specific attribute table 
%columns. They are used to produce different edit widgets in the attribute 
%dialog. These widgets are:
Dans l'onglet \tab{Attributs} vous trouverez un colonne \texttt{Outils d'édition} 
et une colonne \texttt{valeur}. Ces deux colonnes peuvent être utilisées pour 
définir les valeurs ou les plages de valeurs permises lors de l'ajout d'attributs 
dans une colonne. Elles sont utilisées pour générer différents outils d'édition 
dans la fenêtre des attributs. Ces outils sont :
\begin{itemize}[label=--]
\item \textbf{édition de ligne} : un champ d'édition qui permet d'entrer du 
texte simple (ou de restreindre à des nombres pour des attributs de type 
numériques)
%\item Classification: Displays a combo box with the values used for 
%classification, if you have chosen 'unique value' as legend type in the 
%\tab{Style} tab of the properties dialog.
\item \textbf{Classification} : Affiche une boîte combo avec les valeurs 
utilisées pour la classification, si vous avez choisi valeur unique comme type 
de légende dans l'onglet \tab{Style} de la boîte de dialogue des propriétés.
%\item Range: Allows to set numeric values from a specific range. The edit widget 
%can be either a slider or a spin box.
\item \textbf{Portée} : Permet d'indiquer des valeurs numériques depuis une 
portée spécifiée. L'outil d'édition peut être une barre coulissante ou une 
spinbox
%\item Unique value: The user can select one of the values already used in the 
%attribute table. If editable is activated, a line edit is shown with 
%autocompletion support, otherwise a combo box is used.
\item \textbf{valeurs uniques} : l'utilisateur peut sélectionner une des valeurs 
déjà utilisées dans la table attributaire. Si l'édition est activée, une ligne 
est affichée avec le support de l'autocomplétition, autrement une boîte est 
utilisée
%\item File name: Simplifies the selection by adding a file chooser dialog.
\item \textbf{nom de fichier} : Simplifie la sélection par l'ajout d'un dialogue 
de sélection de fichier.
%\item Value map: a combo box with predefined items. The value is stored in the 
%attribute, the description is shown in the comboo box. You can define values 
%manually or load them from a layer or a csv file. 
\item \textbf{Carte de valeur} : une boîte combo avec des objets prédéfinis. La 
valeur peut être stocké dans l'attribut, la description est montrée dans la 
boîte combo. Vous pouvez définir les valeurs manuellement ou les charger depuis 
une couche ou un fichier csv
%\item Enumeration: Opens a combo box with values that can be used within the 
%columns type. This is currently only supported by the postgres provider.
\item \textbf{Enumération} : Ouvre une boîte combo avec des valeurs qui peut 
être utilisé dans le tpe des colonnes. Seul le prestataire postgres le supporte 
pour l'instant
%\item Immutable: The immutable attribute column is read-only. The user is not 
%able to modify the content. 
\item \textbf{Immuable} : L'attribut immuable est en lecture seule, l'utilisateur 
ne peut pas modifié le contenu
\end{itemize}

%\subsection{General Tab}\label{vectorgeneraltab}
\subsection{Onglet Général}\label{vectorgeneraltab}
%The \tab{General} tab is essentially like that of the raster dialog. It allows 
%you to change the display name, set scale dependent rendering options, create a 
%spatial index of the vector file (only for OGR supported formats and PostGIS) 
%and view or change the projection of the specific vector layer. 
%Additionally it is possible to define a certain Edit User Interface for the 
%vector layer written with the   IDE and tools at 
%\url{http://qt.nokia.com/products/developer-tools}. 
L'onglet \tab{Général} des couches vectorielles est très proche de celui des 
couches rasters. Il vous permet de changer le nom affiché, définir des rendus 
différents selon l'échelle, créer un index spatial du fichier vecteur (uniquement 
pour les formats gérés par OGR et \pg) et visualiser ou changer la projection de 
la couche. Il est aussi possible de définir pour la couche vecteur une Interface 
d'édition utilisateur écrite avec les outils et l'IDE Qt Creator sur 
url{http://qt.nokia.com/products/developer-tools}. 

%The \button{Query Builder} button allows you to create a subset of the features 
%in the layer - but this button currently only is available when you open the 
%attribute table and select the \button{Advanced ...} button.
Le bouton \button{Constructeur de requête} vous permet de créer un sous-ensemble 
d'entité au sein de la couche - mais ce bouton ne fonctionne actuellement que 
lorsque vous ouvrez la table attributaire et cliquez sur le bouton \button{\dots} 
à côté de la recherche avancée.

%\subsection{Metadata Tab}
\subsection{Onglet Métadadonnées}

%The \tab{Metadata} tab contains information about the layer, including specifics 
%about the type and location, number of features, feature type, and the editing 
%capabilities. The \guiheading{Layer Spatial Reference System} section, providing 
%projection information, and the \guiheading{Attribute field info} section, listing 
%fields and their data types, are displayed on this tab. This is a quick way to get 
%information about the layer.
L'onglet \tab{Métadadonnées} contient les informations sur la couche dont le type 
et la localisation, le nombre d'entités, le type des entités et les possibilités 
d'éditions. Les sections \guiheading{Système spatial de référence de la couche} 
qui fournit les informations sur la projection et \guiheading{Information de 
champ d'attribut} qui liste les champs et leur type sont affichées dans cet 
onglet. Cet onglet constitue un moyen rapide d'obtenir des informations sur 
une couche.

%\subsection{Actions Tab}\index{actions}\label{label_actions}
\subsection{Onglet Actions}\index{actions}\label{label_actions}

%\qg provides the ability to perform an action based on the attributes of a 
%feature. This can be used to perform any number of actions, for example, 
%running a program with arguments built from the attributes of a feature or 
%passing parameters to a web reporting tool.
\qg est capable d'effectuer des actions basées sur les attributs d'une entité. 
Il peut s'agir de nombreuses actions, par exemple exécuter un programme avec 
des arguments construits à partir des attributs d'une entité, ou encore, passer 
des paramètres à un outil de publication de rapports sur internet.

%Actions are useful when you frequently want to run an external application or 
%view a web page based on one or more values in your vector layer. An example 
%is performing a search based on an attribute value. This concept is used in 
%the following discussion.
Les actions sont utiles si vous voulez exécuter fréquemment une application 
externe ou charger une page web basée sur une ou plusieurs valeurs de votre 
couche vecteur. Un exemple d'application serait d'effectuer une recherche basée 
sur une valeur d'attribut. C'est l'idée utilisée dans les paragraphes qui 
suivent.

%\minisec{Defining Actions}\index{actions!defining}
\minisec{Définir des actions}\index{actions!définir}

%Attribute actions are defined from the vector \dialog{Layer Properties} dialog. 
%To define an action, open the vector \dialog{Layer Properties} dialog and click 
%on the \tab{Actions} tab. Provide a descriptive name for the action. The action 
%itself must contain the name of the application that will be executed when the 
%action is invoked. You can add one or more attribute field values as arguments 
%to the application. When the action is invoked any set of characters that start 
%with a \% followed by the name of a field will be replaced by the value of that 
%field. The special characters \%\% \index{\%\%}will be replaced by the value of 
%the field that was selected from the identify results or attribute table (see 
%Using Actions below).  Double quote marks can be used to group text into a single 
%argument to the program, script or command. Double quotes will be ignored if 
%preceded by a backslash.
Les actions sur les attributs sont définies dans la fenêtre \dialog{Propriétés 
de la couche} des couches vectorielles. Pour définir une action, ouvrez la 
fenêtre de \dialog{Propriétés de la couche} et cliquez sur l'onglet \tab{Actions}. 
Donnez un nom descriptif à l'action. L'action elle-même doit contenir le nom de 
l'application qui sera exécutée quand l'action sera invoquée. Vous pouvez 
ajouter un ou plusieurs champs d'attributs comme argument pour l'application. 
Quand l'action est invoquée n'importe quelle chaîne de caractère précédée de \% 
et correspondant au nom d'un champ sera remplacé par la valeur de ce champ. Le 
caractère spécial \%\% sera remplacé par la valeur d'un champ qui a été 
sélectionné par le résultat d'un Identifier ou dans la table d'attributs (voir 
Utiliser les actions, ci-dessous). Des guillemets peuvent être utilisés pour 
grouper du texte en un seul argument pour le programme, le script ou la 
commande. Les guillemets seront ignorés s'ils sont précédés d'un antislash.

%If you have field names that are substrings of other field names (e.g., \usertext{col1} and \usertext{col10}) you should indicate so, by surrounding the field name (and the \% character) with square brackets (e.g., \usertext{[\%col10]}). This will prevent the \usertext{\%col10} field name being mistaken for the \usertext{\%col1} field name with a \usertext{0} on the end. The brackets will be removed by \qg when it substitutes in the value of the field. If you want the substituted field to be surrounded by square brackets, use a second set like this: \usertext{[[\%col10]]}.
Si vous avez des noms de champs qui sont contenus dans d'autres noms de champs 
(par exemple, \usertext{col1} et \usertext{col10}), vous devez l'indiquer en 
entourant le nom de champ (le caractère \%) par des crochets (par exemple 
\usertext{[\%col10]}). Ceci évitera de prendre le nom de champ \usertext{\%col10} 
pour \usertext{\%col1} avec un \usertext{0} à la fin. Les crochets seront retirés 
quand \qg substituera le nom par la valeur du champ. Si vous voulez que le champ 
à substituer soit entouré de crochets, utilisez un deuxième jeu de crochets 
comme ici : \usertext{[[\%col10]]}.

%The \dialog{Identify Results} dialog box includes a {\em (Derived)} item that contains information relevant to the layer type. The values in this item can be accessed in a similar way to the other fields by using preceeding the derived field name by \usertext{(Derived).}. For example, a point layer has an \usertext{X} and \usertext{Y} field and the value of these can be used in the action with \usertext{\%(Derived).X} and \usertext{\%(Derived).Y}. The derived attributes are only available from the \dialog{Identify Results} dialog box, not the \dialog{Attribute Table} dialog box.
La fenêtre \dialog{Résultats identifiés} inclut une entrée {\em (Dérivé)} qui 
contient des informations pertinentes selon le type de couche. Les valeurs de 
cette entrée sont accessibles de la même manière que les autres champs en 
ajoutant \usertext{(Derived).} avant le nom du champ. Par exemple, une couche 
de points à un champ \usertext{X} et \usertext{Y} et leur valeur peut être 
utilisée dans l'action avec \usertext{\%(Derived).X} et \usertext{\%(Derived).Y}. 
Les attributs dérivés sont disponibles uniquement depuis la fenêtre 
\dialog{Résultats identifiés} et pas la \dialog{Table d'attributs}.

%Two example actions are shown below:\index{actions!examples}
Deux exemples d'action sont proposés ci-dessous : \index{actions!exemples}

\begin{itemize}[label=--]
  \item \usertext{konqueror http://www.google.com/search?q=\%nam}
  \item \usertext{konqueror http://www.google.com/search?q=\%\%}
\end{itemize}

%In the first example, the web browser konqueror is invoked and passed a URL to open. The URL performs a Google search on the value of the \usertext{nam} field from our vector layer. Note that the application or script called by the action must be in the path or you must provided the full path. To be sure, we could rewrite the first example as: \usertext{/opt/kde3/bin/konqueror http://www.google.com/search?q=\%nam}. This will ensure that the konqueror application will be executed when the action is invoked.
Dans le premier exemple, le navigateur internet Konqueror est lancé avec une URL. 
L'URL effectue une recherche Google sur la valeur du champ \usertext{nam} de la 
couche vecteur. Notez que l'application ou le script appelé par l'action doit 
être dans le path sinon vous devez fournir le chemin complet vers l'application. 
Pour être certain, nous pouvons réécrire le premier exemple de cette manière : 
\usertext{/opt/kde3/bin/konqueror http://www.google.com/search?q=\%nam}. Ceci 
assurera que l'application konqueror sera exécutée quand l'action sera invoquée.

%The second example uses the \%\% notation which does not rely on a particular field for its value. When the action is invoked, the \%\% will be replaced by the value of the selected field in the identify results or attribute table.
Le deuxième exemple utilise la notation \%\% dont la valeur ne dépend pas d'un 
champ en particulier. Quand l'action est invoquée, \%\% sera remplacé par la 
valeur du champ sélectionné dans les résultats de l'identification ou dans la 
table d'attributs.

%\minisec{Using Actions}\index{actions!using}\label{label_usingactions}
\minisec{Utiliser les actions}\index{actions!utiliser}\label{label_usingactions}
%Actions can be invoked from either the \dialog{Identify Results} dialog or an \dialog{Attribute Table} dialog. (Recall that these dialogs can be opened by clicking \toolbtntwo{mActionOpenTable}{Identify Features} or \toolbtntwo{mActionOpenTable}{Open Table}.)
Les actions peuvent être invoquées soit depuis la fenêtre \dialog{Résultats 
identifiés} soit depuis la \dialog{Table d'attributs}. (Rappelez-vous que ces 
fenêtres s'ouvrent en cliquant sur\\ \toolbtntwo{mActionOpenTable}{Identifier les 
données} ou \toolbtntwo{mActionOpenTable}{Ouvrir la table d'attributs}.)
%To invoke an action, right click on the record and choose the action from the popup menu. Actions are listed in the popup menu by the name you assigned when defining the actions. Click on the action you wish to invoke.
Pour invoquer une action, faites un clic droit sur un enregistrement et choisissez 
l'action depuis le menu qui apparaît. Les actions sont listées dans le menu par 
le nom que vous leur avez donné en les définissant. Cliquez ensuite sur l'action 
que vous souhaitez invoquer.

%If you are invoking an action that uses the \%\% notation, right-click on the field value in the \dialog{Identify Results} dialog or the \dialog{Attribute Table} dialog that you wish to pass to the application or script.
Si vous invoquez une action qui utilise la notation \%\%, faites un clic droit 
sur la valeur du champ que vous souhaitez passer en argument à l'application ou
 au script dans la fenêtre \dialog{Résultats identifiés} ou la \dialog{Table 
 d'attributs}.

%Here is another example that pulls data out of a vector layer and inserts them into a file using bash and the \usertext{echo} command (so it will only work \nix or perhaps \osx). The layer in question has fields for a species name \usertext{taxon\_name}, latitude \usertext{lat} and longitude \usertext{long}. I would like to be able to make a spatial selection of a localities and export these field values to a text file for the selected record (shown in yellow in the \qg map area). Here is the action to achieve this:
Voici un autre exemple qui récupère des données d'une couche vecteur et qui les 
insère dans un fichier utilisant bash et la commande \usertext{echo} (cela ne 
marchera que sur \nix et peut-être \osx). La couche en question à des champs 
pour le nom d'espèce \usertext{taxon\_name}, la latitude \usertext{lat} et la 
longitude \usertext{long}. Je souhaiterais faire une sélection spatiale des 
localités et exporter ces valeurs des enregistrements sélectionnés dans un 
fichier texte (ils apparaissent en jaune sur la carte dans \qg). Voici l'action 
qui permettra de le faire :

\begin{verbatim}
  bash -c "echo \"%taxon_name %lat %long\" >> /tmp/species_localities.txt"
\end{verbatim}

%After selecting a few localities and running the action on each one, opening the output file will show something like this:
Après avoir sélectionné quelques localités et lancé l'action sur chacune, le fichier de destination ressemblera à ça :

\begin{verbatim}
  Acacia mearnsii -34.0800000000 150.0800000000
  Acacia mearnsii -34.9000000000 150.1200000000
  Acacia mearnsii -35.2200000000 149.9300000000
  Acacia mearnsii -32.2700000000 150.4100000000
\end{verbatim}

%As an exercise we create an action that does a Google search on the \filename{lakes} layer. First we need to determine the URL needed to perform a search on a keyword. This is easily done by just going to Google and doing a simple search, then grabbing the URL from the address bar in your browser. From this little effort we see that the format is: \url{http://google.com/search?q=qgis}, where \usertext{qgis} is the search term. Armed with this information, we can proceed:
Comme exercice, nous allons créer une action qui réalise une recherche Google 
sur la couche \filename{lakes}. Tout d'abord, nous avons besoin de déterminer 
l'URL nécessaire pour effectuer une recherche sur un mot clé. Il suffit 
simplement d'aller sur Google et faire une recherche simple puis récupérer 
l'URL dans la barre d'adresse de votre navigateur. De cela, nous en déduisons 
la formulation : \url{http://google.com/search?q=qgis}, où \usertext{qgis} est 
le terme recherché. À partir de tout cela, nous pouvons poursuivre :

\begin{enumerate}
%\item Make sure the \filename{lakes} layer is loaded.
\item Assurez-vous que la couche \filename{lakes} est chargée
%\item Open the \dialog{Layer Properties} dialog by double-clicking on the layer in the legend or right-click and choose \dropmenuopt{Properties} from the popup menu.
\item Ouvrez la fenêtre \dialog{Propriétés de la couche} en double cliquant sur la couche dans la légende ou en faisant un clic droit et en choisissant \dropmenuopt{Propriétés} dans le menu qui apparaît
%\item Click on the \tab{Actions} tab.
\item Cliquez sur l'onglet \tab{Actions}
%\item Enter a name for the action, for example \usertext{Google Search}.
\item Entrez un nom pour l'action, par exemple \usertext{Recherche Google}
%\item For the action, we need to provide the name of the external program to run. In this case, we can use Firefox. If the program is not in your path, you need to provide the full path.
\item Pour l'action, nous devons fournir le nom du programme externe à lancer. Dans ce cas, nous allons utiliser Firefox. Si le programme n'est pas dans votre path, vous devez fournir le chemin complet
%\item Following the name of the external application, add the URL used for doing a Google search, up to but not included the search term:  \url{http://google.com/search?q=}
\item A la suite du nom de l'application externe, ajoutez l'URL utilisée pour faire la recherche Google, jusqu'au terme de recherche, mais sans l'ajouter :\\ \url{http://google.com/search?q=}
%\item The text in the \guilabel{Action} field should now look like this:\\
%\usertext{firefox \url{http://google.com/search?q=}}
\item Le texte dans le champ \guilabel{Action} devrait ressembler à ça :\\
\usertext{firefox \url{http://google.com/search?q=}}
%\item Click on the drop-down box containing the field names for the \usertext{lakes} layer. It's located just to the left of the \button{Insert Field} button.
\item Cliquez sur le menu déroulant contenant les noms des champs pour la couche \usertext{lakes}. Il est situé juste à gauche du bouton \button{Insérer un champ}
%\item From the drop-down box, select \selectstring{}{NAMES} and click \button{Insert Field}.
\item Dans le menu déroulant, sélectionnez \selectstring{NAMES} et cliquez sur \button{Insérer un champ}
%\item Your action text now looks like this:\\
%\usertext{firefox \url{http://google.com/search?q=\%NAMES}}
\item Le texte de votre action devrait maintenant ressembler à ça :\\
\usertext{firefox \url{http://google.com/search?q=\%NAMES}}
%\item Fo finalize the action click the \button{Insert action} button..
\item Pour finaliser l'action, cliquez sur le bouton \button{Insérer une action}
\end{enumerate}

%This completes the action and it is ready to use. The final text of the action should look like this:
L'action est donc entièrement définie et prête à être utilisée. Le texte final 
de l'action devrait correspondre à ça :

\usertext{firefox \url{http://google.com/search?q=\%NAMES}}

%We can now use the action. Close the \dialog{Layer Properties} dialog and zoom in to an area of interest. Make sure the \filename{lakes} layer is active and identify a lake. In the result box you'll now see that our action is visible:
Nous pouvons maintenant utiliser l'action. Fermez la fenêtre \dialog{Propriétés de la couche} et zoomez sur une zone d'intérêt. Assurez-vous que la couche \filename{lakes} est active puis identifiez un lac. Dans la fenêtre de résultats, vous constatez que notre action est maintenant visible :

\begin{figure}[H]
  \begin{center}
  %\caption{Select feature and choose action \nixcaption}\label{fig:identify_action}\smallskip
  \includegraphics[clip=true, width=8cm]{action_identifyaction}
  \caption{Sélectionnez une entité et choisissez une action \nixcaption}\label{fig:identify_action}
\end{center}
\end{figure}

%When we click on the action, it brings up Firefox and navigates to the URL \url{http://www.google.com/search?q=Tustumena}. It is also possible to add further attribute fields to the action. Therefore you can add a ``+'' to the end of the action text, select another field and click on \button{Insert Field}. In this example there is just no other field available that would make sense to search for.
Quand vous cliquez sur l'action, cela ouvre Firefox et charge l'URL \url{http://www.google.com/search?q=Tustumena}. Il est également possible d'ajouter d'autres champs attributs à l'action. Pour faire cela, vous pouvez ajouter un + à la fin du texte de l'action, sélectionnez un autre champ et cliquez sur \button{Insérer un champ}. Dans cet exemple, la recherche sur un autre champ n'aurait pas de sens.

%You can define multiple actions for a layer and each will show up in the \dialog{Identify Results} dialog. You can also invoke actions from the attribute table by selecting a row and right-clicking, then choosing the action from the popup menu.
Vous pouvez définir de multiples actions pour une couche et chacune apparaitra dans la fenêtre \dialog{Résultats identifiés}. Vous pouvez également invoquer des actions depuis la table d'attributs en sélectionnant une colonne et en faisant un clic droit puis en choisissant l'action dans le menu qui apparaît.

%You can think of all kinds of uses for actions. For example, if you have a point layer containing locations of images or photos along with a file name, you could create an action to launch a viewer to display the image. You could also use actions to launch web-based reports for an attribute field or combination of fields, specifying them in the same way we did in our Google search example.
Vous pouvez imaginer toute sorte d'utilisations pour ces actions. Par exemple, si vous avez une couche de points contenant la localisation d'images ou de photos ainsi qu'un nom de fichier, vous pouvez créer une action qui lancera un visionneur pour afficher les images. Vous pouvez également utiliser les actions pour lancer des rapports sur internet pour un champ attributaire ou une combinaison de champs, en les spécifiant de la même manière que pour une recherche

%\subsection{Joins Tab}\label{sec:joins}
\subsection{Onglet jointure}\label{sec:joins}
\index{vector layers!joins}

%The \tab{Joins} tab allows you to join a loaded attribute table to a loaded
%vector layer. As key columns you have to define a join layer, a join field and
%a target field. QGIS currently supports to join non spatial table formats
%supported by OGR, delimited text and the PostgreSQL provider (see figure~\ref{fig:join_attributes}).
L'onglet \tab{Joins} vous permet de joindre une table attributaire chargée avec 
une table vecteur chargée. Parmi les colonnes clés, vous devez définir une table 
de jointure, un champ de jointure et un champ cible. QGIS gère les formats gérés 
par OGR, texte délimité et le lecteur PosgreSQL pour les formats de table non 
spatiale (voir figure~\ref{fig:join_attributes}).

\begin{figure}[ht]
   \centering
   \includegraphics[clip=true, width=8cm]{join_attributes}
%   \caption{Join an attribute table to an existing vector layer \nixcaption}
    \caption{Joindre une table attributaire à une couche vecteur existante \nixcaption}
   \label{fig:join_attributes}
\end{figure}

%Additionally the add vector join dialog allows to:
La boîte de dialogue de jointure vecteur vous permet de faire également :

\begin{itemize}[label=--]
%\item \checkbox{Cache join layer in virtual memory}
\item \checkbox{Mettre en cache la couche de jointure dans une mémoire virtuel}
%\item \checkbox{Create attribute index on the join field}
\item \checkbox{Créer un index d'attribut sur le champ de jointure}
\end{itemize}


\subsection{Diagramme}\label{sec:diagram}
\index{couches vectorielles!diagramme}
%The \tab{Diagram} tab allows you to add a grahic overlay to a vector layer (see
% figure~\ref{fig:diagramtab}).
L'onglet \tab{Diagramme} permet d'ajouter une couche de graphiques sur une couche 
vecteur (voir figure~\ref{fig:diagramtab}).

\begin{figure}[ht]
   \begin{center}
   \includegraphics[clip=true, width=12cm]{diagram_tab}
   \caption{Dialogue des propriétés de diagrammes \nixcaption}
   \label{fig:diagramtab}
\end{center}
\end{figure}

%The current core implementation of diagrams provides support for piecharts and 
%text diagrams, and for linear scaling of the diagram size according to a 
%classification attribute. The placement of the diagrams interacts with the 
%new labeling. We will demonstrate an example and overlay the alaska 
%boundary layer a piechart diagram showing some temperature data from a climate 
%vector layer. Both vector layers are part of the \qg sample dataset (see
% Section~\ref{label_sampledata}.
L'implémentation interne actuelle des diagrammes permet de visualiser des graphiques sous 
forme de camemberts, de barres ou de lignes selon la valeur d'un attribut de 
classification. Le placement des diagrammes interagit avec le nouveau système 
d'étiquetage. Nous allons vous montrer un exemple en incrustant dans les 
frontières de l'Alaska des données concernant la température issues d'une couche 
vecteur portant sur le climat. Toutes ces couches sont disponibles dans 
l'échantillon de données \qg (voir section~\ref{label_sampledata}).

\begin{enumerate}
\item Cliquez sur l'icône \toolbtntwo{mActionAddOgrLayer}{Ajouter une couche 
vecteur}, parcourez le répertoire de l'échantillon \qg et chargez 
\filename{alaska.shp} and \filename{climate.shp}
\item Double-cliquez sur la couche \filename{climate} dans la légende pour ouvrir 
la fenêtre de\\ \dialog{Propriétés de la Couche}
\item Cliquez sur l'onglet de \tab{Diagramme Incrusté} et sélectionnez 
\button{Diagramme en camembert} comme type de diagramme
\item Nous cherchons à représenter les valeurs de trois colonnes 
\filename{T\_F\_JAN, T\_F\_JAN} et \filename{T\_F\_MEAN}. Sélectionnez d'abord 
\filename{T\_F\_JAN} dans la liste des attributs puis cliquez sur le bouton vert 
\button{+} ensuite \filename{T\_F\_JUL} et enfin \filename{T\_F\_MEAN}
\item Pour une mise à l'échelle linéaire de la taille du diagramme, nous définissons \filename{T\_F\_JUL} comme étant l'attribut de classification
\item Maintenant, cliquez sur \button{Trouver la valeur maximale}, choisissez 10 
comme valeur de taille puis cliquez sur \button{Appliquer} pour afficher le 
diagramme dans la fenêtre principale
\item Vous pouvez adapter la taille du diagramme ou changer la couleur des 
attributs en double-cliquant sur les valeurs colorimétriques dans la liste 
attributaire. La figure~\ref{fig:climatediagram} vous donne une impression du 
résultat
\item Et pour finir cliquez \button{Ok}.
\end{enumerate}

\begin{figure}[ht]
   \begin{center}
   \includegraphics[clip=true, width=12cm]{climate_diagram}
   \caption{Diagramme des températures superposé sur une carte \nixcaption}
   \label{fig:climatediagram}
\end{center}
\end{figure}

%\section{Editing}\index{editing}
\section{Éditer}\index{éditer}

%\qg supports various capabilities for editing OGR, PostGIS and Spatialite
%vector layers. \textbf{Note} - the procedure for editing GRASS layers is
%different - see Section \ref{grass_digitising} for details.
\qg compte un support étendu de l'édition de données provenant de couches 
vectorielles OGR, PostGIS et SpatiaLite. \textbf{Note} - la procédure pour éditer 
des couches GRASS est différente - voir Section \ref{grass_digitising} pour plus 
de détails.

%\begin{Tip}[ht]\caption{\textsc{Concurrent Edits}}
\begin{Tip}[ht]\caption{\textsc{Éditions concurrentes}}
%\qgistip{This version of \qg does not track if somebody else is editing a
%feature at the same time as you. The last person to save their edits wins.
Cette version de \qg ne vérifie pas si quelqu'un d'autre est en train d'éditer 
une entité en même temps que vous, la dernière personne qui enregistre sa 
modification gagne !
\end{Tip}

%\subsubsection{Setting the Snapping Tolerance and Search Radius}
%\label{snapping_tolerance}
\subsubsection{Définir le rayon de tolérance d'accrochage et de recherche}
\label{snapping_tolerance}

%Before we can edit vertices, it is very important to set the snapping tolerance 
%and search radius to a value that allows us an optimal editing of the vector 
%layer geometries.
Avant de pouvoir éditer des sommets, il est très important de fixer la tolérance 
d'accrochage et le rayon de recherche à des valeurs qui nous permettent d'éditer 
les géométries vectorielles de manière optimale.

%\minisec{Snapping tolerance}
\minisec{Tolérance d'accrochage}

%Snapping tolerance is the distance \qg uses to \usertext{search} for the closest vertex and/or segment you are trying to connect when you set a new vertex or move an existing vertex. If you aren't within the snap tolerance, \qg will leave the vertex where you release the mouse button, instead of snapping it to an existing vertex and/or segment.
La tolérance d'accrochage est la distance que \qg utilise pour \usertext{chercher} le sommet ou le segment le plus près que vous cherchez à connecter lorsque vous créez un nouveau sommet ou en déplacez un existant. Si vous n'êtes pas dans la tolérance d'accrochage, \qg va laisser le vertex à l'endroit où vous lâchez le bouton de la souris, au lieu de l'accrocher à un sommet ou un segment existant.

\begin{enumerate}
%\item A general, project wide snapping tolerance can be defined choosing \mainmenuopt{Settings} > \dropmenuopttwo{mActionOptions}{Options}. In the \tab{Digitizing} tab you can select between to vertex, to segment or to vertex and segment as default snap mode. You can also define a default snapping tolerance and a search radius for vertex edits. Remember the tolerance is in layer units. In our digitizing project (working with the Alaska dataset), the units are in feet. Your results may vary, but something on the order of 300ft should be fine at a scale of 1:10 000 should be a reasonable setting.
\item Une tolérance générale, commune à tout le projet, peut-être définie dans 
\mainmenuopt{Préférences} > \dropmenuopttwo{mActionOptions}{Options} (On Mac : 
allez dans \mainmenuopt{\qg} \arrow Préférences, sous Linux : \mainmenuopt{Édit} \
arrow \dropmenuopttwo{mActionOptions}{Options}.). Dans 
l'onglet \tab{Numérisation}, vous pouvez choisir le mode d'accrochage par défaut 
: sur un sommet, sur un segment ou sur un sommet ou un segment. Vous pouvez 
également définir une tolérance d'accrochage par défaut et un rayon de recherche 
pour les éditions de sommets. La tolérance peut être définie dans l'unité de la 
couche ou en pixel, l'avantage du pixel est qu'elle n'a pas à être changée pour 
tenir compte des zooms. 
Dans notre projet de numérisation (le travail sur le jeu de données Alaska), les 
unités sont en pieds. Le résultat peut varier, mais une tolérance de l'ordre de 
300 pieds devrait être convenable pour une échelle de 1:10000e.
%\item A layer based snapping tolerance can be defined by choosing 
%\mainmenuopt{Settings} (or \mainmenuopt{File}) \arrow 
%\button{Snapping options\dots} to enable and adjust snapping mode and tolerance 
%on a layer basis (see Figure~\ref{fig:snappingoptions}).
\item Une tolérance d'accrochage liée à une couche peut être définie dans 
\mainmenuopt{Préférences} (ou \mainmenuopt{Fichier}) \arrow \button{Options 
d'accrochage\dots} pour activer et ajuster le mode d'accrochage et la tolérance 
pour chaque couche (voir figure~\ref{fig:snappingoptions}).
\end{enumerate}

%Note that this layer based snapping overrides the global snapping option set in the Digitizing tab. So if you need to edit one layer, and snap its vertices to another layer, then enable snapping only on the \usertext{snap to} layer, then decrease the global snapping tolerance to a smaller value.  Furthermore, snapping will never occur to a layer which is not checked in the snapping options dialog, regardless of the global snapping tolerance. So be sure to mark the checkbox for those layers that you need to snap to.

Veuillez noter que l'accrochage défini pour cette couche est prioritaire par 
rapport à celui défini dans les options générales. Si vous avez besoin d'éditer 
une couche en vous accrochant à une autre, il vous faut donc activer l'accrochage 
uniquement sur la couche à accrocher et réduire la tolérance générale d'accrochage 
à une valeur moindre. De plus, l'accrochage ne se produira jamais sur une couche 
dont l'accrochage n'a pas été activé, qu'importe l'option générale. Assurez-vous 
de cocher la case idoine sur les couches que vous voulez pouvoir utiliser.

\begin{figure}[H]
  \begin{center}
  %\caption{Edit snapping options on a layer basis \nixcaption}\label{fig:snappingoptions}\smallskip
  \includegraphics[clip=true, width=12cm]{editProjectSnapping}
  \caption{Édition des options d'accrochage pour chaque couche \nixcaption}
  \label{fig:snappingoptions}
\end{center}
\end{figure}

%\minisec{Search radius}
\minisec{Rayon de recherche}

%Search radius is the distance \qg uses to \usertext{search} for the closest vertex you are trying to move when you click on the map. If you aren't within the search radius, \qg won't find and select any vertex for editing and it will pop up an annoying warning to that effect. Snap tolerance and search radius are set in map units so you may find you need to experiment to get them set right. If you specify too big of a tolerance, \qg may snap to the wrong vertex, especially if you are dealing with a large number of vertices in close proximity. Set search radius too small and it won't find anything to move.
Le rayon de recherche est la distance que \qg utilise pour \usertext{chercher} 
le sommet le plus proche que vous souhaitez déplacer quand vous cliquez sur la 
carte. Si vous n'êtes pas dans le rayon de recherche, \qg ne trouvera ni ne 
sélectionnera de sommets à éditer et une fenêtre d'alerte désagréable apparaitra. 
La tolérance d'accrochage et le rayon de recherche sont définis dans les unités 
de la carte, vous allez peut-être avoir besoin d'expérimenter différentes valeurs 
avant de trouver la bonne. Si vous spécifiez une tolérance trop grande, \qg 
risque d'accrocher le mauvais sommet, surtout si vous avez un grand nombre de 
sommets à proximité. Définissez un rayon de recherche trop petit et \qg ne 
trouvera rien à déplacer.

%The search radius for vertex edits in layer units can be defined in the \tab{Digitizing} tab under \mainmenuopt{Settings} > \dropmenuopttwo{mActionOptions}{Options}. The same place where you define the general, project wide snapping tolerance.
Le rayon de recherche pour l'édition des sommets dans l'unité de la couche peut 
être défini dans l'onglet \tab{Numérisation} de \mainmenuopt{Préférences} > 
\dropmenuopttwo{mActionOptions}{Options}. Au même endroit que vous définissez 
la tolérance d'accrochage pour tout le projet.

% \subsection{Zooming and Panning}
\subsection{Zoomer et se déplacer}

% Before editing a layer, you should zoom in to your area of interest. This
% avoids waiting while all the vertex markers are rendered across the entire
% layer.

Avant d'éditer une couche, vous devriez zoomer sur la zone qui vous intéresse. Cela évite de devoir attendre que tous les sommets soient calculés sur l'ensemble de la couche.

% Apart from using the \toolbtntwo{mActionPan}{pan} and
% \toolbtntwo{mActionZoomIn}{zoom-in}/\toolbtntwo{mActionZoomOut}{zoom-out}
% icons on the toolbar with the mouse, navigating can also be done with the
% mouse wheel, spacebar and the arrow keys.

Au lieu d'utiliser les icones \toolbtntwo{mActionPan}{Se déplacer dans la carte} et \toolbtntwo{mActionZoomIn}{zoom +}/\toolbtntwo{mActionZoomOut}{zoom -} de la barre d'outils avec la souris, la navigation peut également se faire avec la roulette de la souris, la barre espace et les flèches du clavier.

% \minisec{Zooming and panning with the mouse wheel}
\minisec{Zoomer et bouger avec la souris}

% While digitizing you can press the mouse wheel to pan inside of the main
% window and you can roll the mouse wheel to zoom in and out on the map. For
% zooming place the mouse cursor inside the map area and roll it forward (away
% from you) to zoom in and backwards (towards you) to zoom out. The mouse cursor position
% will be the center of the zoomed area of interest. You can customize the behavior
% of the mouse wheel zoom using the \tab{Map tools} tab under the
% \mainmenuopt{Settings} >\dropmenuopt{Options} menu.

Lorsque vous vectorisez, vous pouvez appuyer sur la roulette de la souris pour vous déplacer dans la fenêtre principale et la faire rouler pour zoomer la carte. Pour vous rapprocher, faites rouler la molette vers l'avant tandis que si vous voulez vous éloigner vous devrez la faire rouler vers vous. La position du curseur sera le centre la zone affichée. Vous pouvez personnaliser le zoom en utilisant l'onglet \tab{Outils cartographiques} dans le menu \mainmenuopt{Préférences} >\dropmenuopt{Options}.

% \minisec{Panning with the arrow keys}
\minisec{Se déplacer avec les touches du clavier}

% Panning the Map during digitizing is possible with the arrow keys. Place
% the mouse cursor inside the map area and click on the right arrow key to
% pan east, left arrow key to pan west, up arrow key to pan north and down
% arrow key to pan south.

Il est possible de se déplacer sur la carte en utilisant les flèches du clavier. Placez votre curseur sur la carte et appuyez sur la flèche de droite pour vous déplacer vers l'Est, la flèche de gauche pour aller à l'Ouest, la flèche du haut pour le Nord et celle du bas pour le Sud.

% You can also use the spacebar to temporarily cause mouse movements to pan
% then map. The PgUp and PgDown keys on your keyboard will cause the map
% display to zoom in or out without interrupting your digitising session.

Vous pouvez utiliser la barre d'espace pour que les mouvements de la souris se traduisent par un déplacement sur la carte. Les touches PgUp et PgDown vous permettront de zoomer sans devoir interrompre votre numérisation.

%\subsection{Topological editing}
\subsection{Édition topologique}

%Besides layer based snapping options you can also define some topological 
%functionalities in the \dialog{Snapping options \dots} dialog in the 
%\mainmenuopt{Settings} (or \mainmenuopt{File}) menu. Here you can define 
%\checkbox{Enable topological editing} and/or for polygon layers you can 
%activate the column \checkbox{Avoid Int.} which avoids intersection of new 
%polygons.
Au-delà des options d'accrochage pour chaque couche, vous pouvez définir des 
fonctionnalités topologiques dans la boîte de dialogue \dialog{Options d'accrochage \dots} 
dans le menu \mainmenuopt{Paramètres} (ou \mainmenuopt{Fichier}). Vous pouvez ici 
définir \checkbox{Activer l'édition topologique} et/ou pour les couches polygones 
vous pouvez activer la colonne  \checkbox{Éviter la superposition} qui évite 
l'intersection de nouveaux polygones.

%\minisec{Enable topological editing}
\minisec{Activer l'édition topologique}

%The option \checkbox{Enable topological editing} is for editing and maintaining common boundaries in polygon mosaics. \qg "detects" a shared boundary in a polygon mosaic and you only have to move the vertex once and \qg will take care about updating the other boundary.
L'option \checkbox{Activer l'édition topologique} permet d'éditer en gardant 
des limites communes entre les polygones. QGIS "détecte" une limite commune entre 
les polygones et vous avez simplement à déplacer le sommet une fois et \qg 
s'occupera de mettre à jour l'autre limite.

%\minisec{Avoid intersections of new polygons}
\minisec{Éviter les intersections de nouveaux polygones}

%The second topological option in the \checkbox{Avoid Int.} column, called 
%'Avoid intersections of new polygons' avoids overlaps in polygon mosaics. It is for quicker digitizing of 
%adjacent polygons. If you already have one polygon, it is possible with this 
%option to digitise the second one such that both intersect and qgis then cuts 
%the second polygon to the common boundary. The advantage is that users don't 
%have to digitize all vertices of the common boundary.
La deuxième option topologique dans la colonne \checkbox{Éviterl la superposition}, 
'Éviter les intersections de nouveaux polygones', permet d'éviter des 
recouvrements entre les polygones. Cela permet de numériser des polygones 
adjacents plus rapidement. Si vous avez déjà un polygone, avec cette option, 
vous pouvez numériser le second de manière à ce qu'ils intersectent et \qg 
coupera le second polygone aux limites communes. L'avantage est que les 
utilisateurs n'ont pas à numériser tous les sommets des limites communes.

%\subsection{Editing an Existing Layer}
\subsection{Numériser une couche existante}
\index{couches vectorielles!numériser}
\index{numériser!une couche existante}
\label{sec:edit_existing_layer}

%By default, \qg loads layers read-only: This is a safeguard to avoid accidentally editing a layer if there is a slip of the mouse. However, you can choose to edit any layer as long as the data provider supports it, and the underlying data source is writable (i.e. its files are not read-only).
Par défaut, \qg charge les couches en lecture seule : c'est une sécurité pour 
éviter d'éditer accidentellement une couche si la souris a glissé. Cependant, 
vous pouvez choisir d'éditer une couche du moment que le fournisseur de données 
le gère et que la source de données est éditable (c.-à-d. fichiers qui ne sont 
pas en lecture seule). 
%Layer editing is most versatile when used on PostgreSQL/PostGIS data sources.
L'édition d'une couche est plus flexible lorsqu'il s'agit de sources de données 
PostgreSQL/PostGIS.

% In general, editing vector layers is divided into a digitizing and an advanced
% digitizing toolbar, described in Section \ref{sec:advanced_edit}. You can
% select and unselect both under \mainmenuopt{Settings} > \dropmenuopt{Toolbars}.
% Using the basic digitizing tools you can perform the following functions:

En général, l'édition des couches vectorielles est répartie dans une barre de 
numérisation et une barre de numérisation avancée telle que décrite dans la 
section \ref{sec:advanced_edit}. Vous pouvez sélectionner les deux dans 
\mainmenuopt{Préférences} > \dropmenuopt{Barre d'Outils}. En utilisant les 
outils basiques de numérisation, vous pouvez accomplir les actions suivantes :

%\begin{Tip}[ht]\caption{\textsc{Data Integrity}}
\begin{Tip}[ht]\caption{\textsc{Intégrité des données}}
%\qgistip{It is always a good idea to back up your data source before you start editing. While the authors of \qg have made every effort to preserve the integrity of your data, we offer no warranty in this regard.}
Sauvegarder vos données avant de se lancer dans une édition est toujours une 
bonne idée. Bien que les auteurs de \qg ont fait beaucoup d'efforts pour 
préserver l'intégrité de vos données, nous n'offrons aucune garantie.
\end{Tip}
%traduction
\begin{table}[ht]\index{couches vectorielles!outils d'édition basique}
\centering

\begin{tabular}{|l|p{5.5cm}|l|p{5.5cm}|}
\hline \textbf{Icon} & \textbf{Purpose} & \textbf{Icon} & \textbf{Purpose} \\
\hline \includegraphics[width=0.7cm]{mActionToggleEditing}
   & Basculer en mode édition
   & \includegraphics[width=0.7cm]{mActionCapturePoint}
   & Ajouter une entité: Créer un point \\
\hline \includegraphics[width=0.7cm]{mActionCaptureLine}
   & Ajouter une entité: Créer une Ligne
   & \includegraphics[width=0.7cm]{mActionCapturePolygon}
   & Ajouter une entité: Créer un polygone \\
\hline \includegraphics[width=0.7cm]{mActionMoveFeature}
   & Déplacer une entité
   & \includegraphics[width=0.7cm]{mActionMoveVertex}
   & Déplacer un sommet\\
\hline \includegraphics[width=0.7cm]{mActionAddVertex}
   & Outil de noeud \\
\hline \includegraphics[width=0.7cm]{mActionDeleteSelected}
   & Effacer la sélection
   & \includegraphics[width=0.7cm]{mActionEditCut}
   & Couper une entité \\
\hline \includegraphics[width=0.7cm]{mActionEditCopy}
   & Copier une entité 
   & \includegraphics[width=0.7cm]{mActionEditPaste} 
   & Coller une entité \\
\hline \includegraphics[width=0.7cm]{mActionFileSave}
   & Enregistrer les modifications et continuer
   &  &  \\
\hline
\end{tabular}
%\caption{Vector layer basic editing toolbar}\label{tab:vector_editing}\medskip
\caption{Barre d'touils de numérisation}\label{tab:vector_editing}
\end{table}

% All editing sessions start by choosing the
% \dropmenuopttwo{mActionToggleEditing}{Toggle editing} option.
% This can be found in the context menu after right clicking on the legend
% entry for that layer.\index{Allow Editing}

Toutes les sessions d'édition débutent par la sélection de l'option\\ 
\dropmenuopttwo{mActionToggleEditing}{Basculer en mode édition}. Elle se trouve 
dans le menu contextuel après un clic droit sur la couche voulue.

% Alternately, you can use the \index{Toggle Editing}
% \toolbtntwo{mActionToggleEditing}{Toggle editing} button from the digitizing
% toolbar to start or stop the editing mode.\index{editing!icons} Once the
% layer is in edit mode, markers will appear at the vertices, and additional
% tool buttons on the editing toolbar will become available.

Alternativement, vous pouvez utiliser le bouton \index{Basculer en mode 
édition}\toolbtntwo{mActionToggleEditing}{Basculer en mode édition} dans la 
barre de numérisation pour débuter ou terminer une session d'édition.
\index{numériser!icônes} Une fois que la couche est éditable, les marqueurs vont 
apparaître sur les sommets et de nouveaux outils seront disponibles dans la 
barre d'outils.

%\begin{Tip}[ht]\caption{\textsc{Save Regularly}}
\begin{Tip}[ht]\caption{\textsc{Fréquence de sauvegarde}}
%\qgistip{Remember to \toolbtntwo{mActionFileSave}{Save Edits} regularly. This will
%+also check that your data source can accept all the changes.}
N'oubliez pas de cliquer sur \toolbtntwo{mActionToggleEditing}{Sauver l'édition} 
régulièrement. Cela vous permet de confirmer que votre source de données accepte 
toutes vos modifications.
\end{Tip}

%All editing sessions start by choosing the \dropmenuopttwo{mActionToggleEditing}{Toggle editing} option. This can be found in the context menu after right clicking on the legend entry for that layer.\index{Allow Editing} Alternately, you can use the \index{Toggle Editing} \toolbtntwo{mActionToggleEditing}{Toggle editing} button from the toolbar to start or stop the editing mode.\index{editing!icons} Once the layer is in edit mode, markers will appear at the vertices, and additional tool buttons on the editing toolbar will become available.
Toute session d'édition commence par un clic sur 
\dropmenuopttwo{mActionToggleEditing}{Basculer en mode édition}. Ceci se trouve 
dans le menu contextuel qui apparaît après un clic droit sur la couche dans la 
légende.\index{Autoriser l'édition} Sinon, vous pouvez utiliser le bouton 
\index{Basculer en mode édition} \toolbtntwo{mActionToggleEditing}{Basculer en 
mode édition} de la barre d'outils pour lancer ou stopper l'édition. 
\index{éditer!icônes} Une fois la couche en mode édition, les marqueurs 
apparaissent sur les sommets et de nouveaux outils de la barre d'outils édition 
sont disponibles.

%\minisec{Adding Features}
%\index{vector layers!adding!feature}
\minisec{Ajouter des entités} \index{couches vectorielles!déplacer!entités}

%Before you start adding features, use the \toolbtntwo{mActionPan}{pan} and \toolbtntwo{mActionZoomIn}{zoom-in}/\toolbtntwo{mActionZoomOut}{zoom-out} tools to first navigate to the area of interest.
Avant de commencer à ajouter des entités, utiliser les outils \toolbtntwo{mActionPan}{Se déplacer dans la carte} et \toolbtntwo{mActionZoomIn}{zoom +}/\toolbtntwo{mActionZoomOut}{zoom -} pour naviguer vers la zone d'intérêt.

%Then you can use the \toolbtntwo{mActionCapturePoint}{Capture point}, \toolbtntwo{mActionCaptureLine}{Capture line} or \toolbtntwo{mActionCapturePolygon}{Capture polygon} icons on the toolbar to put the \qg cursor into digitizing mode.
Vous pouvez utiliser \toolbtntwo{mActionCapturePoint}{Capturer le Point}, \toolbtntwo{mActionCaptureLine}{Capturer la Ligne} ou\\ \toolbtntwo{mActionCapturePolygon}{Capturer le Polygone} dans la barre d'outils pour mettre le curseur de \qg en mode numérisation.

%For each feature, you first digitize the geometry, then enter its attributes.
Pour chaque entité, vous numérisez d'abord la géométrie puis entrez les attributs.

%To digitize the geometry, left-click on the map area to create the first point of your new feature.
Pour numériser la géométrie, faites un clic gauche sur la zone de la carte pour créer le premier point de votre nouvelle entité.

%For lines and polygons, keep on left-clicking for each additional point you wish to capture.  When you have finished adding points, right-click anywhere on the map area to confirm you have finished entering the geometry of that feature.
Pour les lignes ou les polygones, continuer à faire des clics gauches pour chaque nouveau point que vous souhaitez capturer. Lorsque vous avez fini d'ajouter des points, faites un clic droit n'importe où sur la carte pour confirmer que vous avez fini d'entrer la géométrie de cette entité.

%The attribute window will appear, allowing you to enter the information for the new feature. Figure \ref{fig:vector_digitising} shows setting attributes for a fictitious new river in Alaska. In the \tab{Digitising} tab under the
%\mainmenuopt{Settings} \arrow \dropmenuopt{Options} menu, you can also activate 
%\\
%\checkbox{Suppress attributes pop-up windows after each created feature}. 
% \checkbox{Reuse last entered attribute values}.
La fenêtre des attributs apparaît, ce qui vous permet d'entrer les informations 
sur la nouvelle entité. La figure \ref{fig:vector_digitising} montre les 
attributs d'édition pour une nouvelle rivière fictive en Alaska. Dans le panneau 
\tab{Numérisation} du menu \mainmenuopt{Préférences} \arrow \dropmenuopt{Options}, 
vous pouvez activez \checkbox{Supprimer la fenêtre d'avertissement lors de la 
création de chaque entité} et \checkbox{Réutiliser les dernières valeurs d'attribut 
entrée}.

\begin{figure}[ht]
  \begin{center}
  %\caption{Enter Attribute Values Dialog after digitizing a new vector feature \nixcaption}\label{fig:vector_digitising}\smallskip
   \includegraphics[clip=true, width=6cm]{editDigitizing}
  \caption{Fenêtre de saisie d'attributs suivant la création d'une nouvelle entité \nixcaption}\label{fig:vector_digitising}
\end{center}
\end{figure}

Avec l'icône \toolbtntwo{mActionMoveFeature}{Déplacer Entités}, vous pouvez déplacer des entités existantes.

%\begin{Tip}[ht]\caption{\textsc{Attribute Value Types}}
\begin{Tip}[ht]\caption{\textsc{Types des valeurs d'attribut}}
%\qgistip{At least for shapefile editing the attribue types are validated during the entry. Because of this, it is not possible to enter a number into the text-column in the dialog \dialog{Enter Attribute Values} or vica versa. If you need to do so, you should edit the attributes in a second step within the \dialog{Attribute table} dialog.}
Pour l'édition des shapefiles au moins, les types des attributs sont validés au moment de la saisie. À cause de cela, il n'est pas possible d'entrer un nombre dans un champ de type texte dans la fenêtre \dialog{Entrez les valeurs d'attributs} et vice-versa. Si vous avez besoin de le faire, vous devez éditer les attributs par la suite dans la fenêtre \dialog{Table d'attributs}.
\end{Tip}

%\minisec{Node Tool}
%\index{vector layers!node!tool}
\minisec{Outil de nœud}
\index{couches vectorielles!noeud!outil}

%For both PostgreSQL/PostGIS and shapefile-based layers, the
%\toolbtntwo{mActionNodeTool}{Node Tool} provides manipulation capabilites
%of feature vertices similar to CAD programs. It is possible to simply select
%multiple vertices at once and to move, add or delete them alltogether. The node
%tool also works with 'on the fly' projection turned on and supports
%the topological editing feature. This tool is, unlike other tools in Quantum GIS,
%persistent, so when some operation is done, selection stays active for this
%feature and tool. If the node tool couldn't find any features, a warning will be
%displayed.

Pour les couches PostgreSQL/PostGIS et shapefile, l'\toolbtntwo{mActionNodeTool}
{Outil de nœud} offre des capacités de manipulation des sommets des entités 
semblables à celles des logiciels de CAO. Il est possible de sélectionner 
plusieurs sommets ensemble et de les bouger, ajouter ou supprimer en une fois. 
Cet outil fonctionne sur les couches reprojetées "à la volée" et supporte des 
fonctionnalités d'éditions topologiques. Contrairement aux autres outils de 
Quantum GIS, la sélection persiste même lorsqu’une autre opération est effectuée. 
Si l'outil de nœud ne trouve pas d'entités, un avertissement sera affiché.

%Important is to set the property \mainmenuopt{Settings} \arrow
%\dropmenuopttwo{mActionOptions}{Options} \arrow
%\tab{Digitizing} \arrow \selectnumber{Search Radius}{10} to a number greater than
%zero. Otherwise \qg will not be able to tell which vertex is being edited.
Vous devez définir le paramètre \mainmenuopt{Préférences}>
\dropmenuopttwo{mActionOptions}{Options} \\ \arrow \tab{Numérisation} \arrow 
\selectnumber{Rayon de recherche}{10} à un nombre supérieur à zéro. Sinon \qg 
ne sera pas en mesure de dire quelle entité est éditée.

%\begin{Tip}\caption{\textsc{Vertex Markers}}
\begin{Tip}[ht]\caption{\textsc{Marqueurs de sommets}}
%The current version of \qg supports three kinds of vertex-markers -
%Semi transparent circle, Cross and None. To change the marker style, choose
%\dropmenuopttwo{mActionOptions}{Options} from the \mainmenuopt{Settings} menu
%and click on the \tab{Digitizing} tab and select the appropriate entry.
La version actuelle de \qg présente 3 types de marqueurs - un cercle 
semi-transparent, une croix ou rien. Pour changer de style de marqueurs, allez 
dans le menu \dropmenuopttwo{mActionOptions}{Options} et cliquez sur l'onget 
\tab{Numérisation} et sélectionnez le symbole voulu dans la liste déroulante.
\end{Tip}

%\minisec{Basic operations}\index{vector layers!Node Tool}
\minisec{Opérations basiques}\index{couches vectorielles!outil!noeud}

%Start by activating the \toolbtntwo{mActionNodeTool}{Node Tool} and selecting
%some features by clicking on it. Red boxes appear at each vertex of this feature.
%%Perhaps the error message mentioned below is in fact a bug, in which case the
%%bug should be fixed rather than including this note
%Note that to select a polygon you must click one of its vertices or edges;
%clicking inside it will produce an error message. Once a feature
%is selected the following functionalities are available:
Commencez par cliquer sur le bouton \toolbtntwo{mActionNodeTool}{Outil de nœud} 
puis sélectionnez une entité. Notez que pour sélectionner un polygone vous devez 
cliquer sur un de ses sommets ou bords ; cliquer à l'intérieur affichera un 
message d'erreur. Une fois qu'une entité est sélectionnée, les fonctionnalités 
suivantes sont disponibles :

\begin{itemize}[label=--]
%\item \textbf{Selecting vertices}: You can select vertices by clicking on them
%one at a time, by clicking on an edge to select the vertices at both ends, or
%by clicking and dragging a rectangle around some vertices.  When a vertex is
%selected its color changes to blue. To add more vertices to the current selection,
%hold down the \keystroke{Ctrl} key while clicking. Hold down
%\keystroke{Ctrl}\keystroke{Shift} when clicking to toggle the selection state of
%vertices (vertices that are currently unselected will be selected as usual, but
%also vertices that are already selected will become unselected).
\item \textbf{Sélectionner un sommet} : Vous pouvez sélectionner des sommets en 
cliquant dessus l'un après l'autre, en cliquant sur un bord pour sélectionner les 
sommets des deux côtés ou en dessinant un rectangle autour des sommets. Quand un 
sommet est sélectionné, sa couleur est modifiée en bleu. Pour ajouter plusieurs 
sommets à la sélection, gardez la touche \keystroke{Ctrl} appuyée pour inverser 
l'état de sélection du sommet (les sommets qui ne sont pas sélectionnés seront sélectionnés comme d'habitude, mais les sommets sélectionnés seront désélectionnés).
%\item \textbf{Adding vertices}: To add a vertex simply double click near
%an edge and a new vertex will appear on the edge near to the cursor. Note that
%the vertex will appear on the edge, not at the cursor position, therefore it has
%to be moved if necessary.
\item \textbf{Ajouter un sommet} : Pour ajouter un sommet faites simplement un 
double clic près d'un segment et un nouveau sommet apparaîtra sur le segment, pas 
sur la position du curseur, il devra donc être déplacé.
%\item \textbf{Deleting vertices}: After selecting vertices for deletion, click the
%\keystroke{Delete} key. Note that you cannot use the
%\toolbtntwo{mActionNodeTool}{Node Tool} to delete a complete feature; \qg will
%ensure it retains the minimum number of vertices for the feature type you are
%working on. To delete a complete feature use the
%\toolbtntwo{mActionDeleteSelected}{Delete Selected} tool.
\item \textbf{Supprimer un sommet} : Après avoir sélectionné des sommets, 
appuyez sur la touche \keystroke{Suppr}. Notez que vous ne pouvez pas utiliser 
\toolbtntwo{mActionNodeTool}{Outil de nœud} pour supprimer une entité complète ; 
\qg veillera cependant à ce qu'il reste suffisamment de sommets pour conserver 
une entité topologiquement correcte (p. ex. jamais moins de 3 sommets pour un 
polygone), pour supprimer l'entité entière utilisez l'outil 
\toolbtntwo{mActionDeleteSelected}{Supprimer la sélection}
%\item \textbf{Moving vertices}: Select all the vertices you want to move. Click
%on a selected vertex or edge and drag in the direction you wish to move. All the
%selected vertices will move together. If snapping is enabled,
%the whole selection can jump to the nearest vertex or line.
\item \textbf{Déplacer un sommet} : Sélectionnez tous les sommets que vous voulez 
déplacer. Cliquez sur un sommet sélectionné ou un segment dans la direction que 
vous souhaitez. Tous les sommets bougeront dans la même direction que le curseur. 
Si l'option d'accrochage est activée, la sélection complète peut sauter sur la 
ligne ou le sommet le plus proche du curseur.
\end{itemize}

%Each change made with the node tool is stored as a separate entry in the undo dialog.
%Remember that all operations support topological editing when this is turned on.
%On the fly projection is also supported, and the node tool provides tooltips to
%identify a vertex by hovering the pointer over it.
Chaque changement réalisé avec l'outil de nœud est stocké comme entrée séparée 
dans la boîte dialogue de retour en arrière. Souvenez-vous que toutes les 
opérations gèrent l'édition topologique lorsqu'elle est activée. La projection à la 
volée est également gérée, et l'outil de nœud fournit une astuce pour identifier 
les sommets en plaçant la souris au dessus.

%\minisec{Cutting, Copying and Pasting Features}
\minisec{Couper, Copier et Coller des entités}

%\index{vector layers!cut!feature}
%\index{vector layers!copy!feature}
%\index{vector layers!paste!feature}
%\index{editing!cutting features}
%\index{editing!copying features}
%\index{editing!pasting features}
\index{couches vectorielles!couper!entité}
\index{couches vectorielles!copier!entité}
\index{couches vectorielles!coller!entité}
\index{éditer!couper des entités}
\index{éditer!copier des entités}
\index{éditer!coller des entités}

%Selected features can be cut, copied and pasted between layers in the same \qg project, as long as destination layers are set to  \toolbtntwo{mActionToggleEditing}{Toggle editing} beforehand.
Une entité sélectionnée peut être coupée, copiée et collée entre des couches d'un 
même projet \qg, du moment que les couches de destination sont 
\toolbtntwo{mActionToggleEditing}{Basculées en mode édition} au préalable.

%Features can also be pasted to external applications as text:  That is, the features are represented in CSV format with the geometry data appearing in the OGC Well-Known Text (WKT) format.
Les entités peuvent également être collées dans des applications externes au 
format texte. Les entités sont alors représentées au format CSV et leur géométrie 
apparaît dans le format OGC Well-Known Text (WKT).

%However in this version of \qg, text features from outside \qg cannot  be pasted to a layer within \qg. When would the copy and paste function come in handy? Well, it turns out that you can edit more than one layer at a time and copy/paste features between layers. Why would we want to do this?  Say we need to do some work on a new layer but only need one or two lakes, not the 5,000 on our \filename{big\_lakes} layer. We can create a new layer and use copy/paste to plop the needed lakes into it.
Cependant, dans cette version de \qg, les entités au format texte venant 
d'applications externes ne peuvent pas être collées à une couche dans \qg. En 
quoi les fonctions copier et coller sont-elles utiles ? Et bien il se trouve 
que vous pouvez éditer plus d'une couche à la fois et que vous pouvez alors 
utiliser les fonctions copier/coller entre les couches. Pourquoi voudrions-nous 
faire cela ? Imaginons que nous devions travailler sur une nouvelle couche, 
mais que nous avions besoin que d'un ou deux lacs, pas les 5 000 de notre 
couche \filename{big\_lakes}. Nous pouvons créer une nouvelle couche puis 
utiliser copier/coller pour y insérer les quelques lacs.

%As an example we are copying some lakes to a new layer:
Voici un exemple de copie de quelques lacs dans une nouvelle couche :

\begin{enumerate}
%\item Load the layer you want to copy from (source layer)
\item Chargez la couche dont vous voulez copier des entités (couche source)
%\item Load or create the layer you want to copy to (target layer)
\item Chargez ou créez la couche sur laquelle vous voulez coller des entités (couche cible)
%\item Start editing for target layer
\item Lancez l'édition pour la couche cible
%\item Make the source layer active by clicking on it in the legend
\item Assurez-vous que la couche source est active en cliquant dessus dans la légende
%\item Use the \toolbtntwo{mActionSelect}{Select} tool to select the feature(s) on the source layer
\item Utilisez l'outil \toolbtntwo{mActionSelect}{Sélection} pour sélectionner les entités dans la couche source
%\item Click on the \toolbtntwo{mActionEditCopy}{Copy Features} tool
\item Cliquez sur l'outil \toolbtntwo{mActionEditCopy}{Copier Entités}
%\item Make the destination layer active by clicking on it in the legend
\item Assurez-vous que la couche cible est active en cliquant dessus dans la légende
%\item Click on the \toolbtntwo{mActionEditPaste}{Paste Features} tool
\item Cliquez sur l'outil \toolbtntwo{mActionEditPaste}{Coller Entités}
%\item Stop editing and save the changes
\item Stoppez l'édition et sauvegardez les changements
\end{enumerate}

%What happens if the source and target layers have different schemas (field names and types are not the same)? \qg populates what matches and ignores the rest. If you don't care about the attributes being copied to the target layer, it doesn't matter how you design the fields and data types. If you want to make sure everything - feature and its attributes - gets copied, make sure the schemas match.
Qu'arrive-t-il si les couches sources et cibles ont différents schémas de données 
(noms et type des champs différents) ? \qg remplit ceux qui correspondent et 
ignore les autres. Si la copie des attributs ne vous intéresse pas, la façon 
dont vous designer les champs et les types de données n'a pas d'importance. Si 
vous voulez être sûr que tout - entité et ses attributs - est copié, assurez-vous 
que les schémas de données correspondent.

%\begin{Tip}[ht]\caption{\textsc{Congruency of Pasted Features}}
\begin{Tip}[ht]\caption{\textsc{Congruence des entités copiées}}
%\qgistip{If your source and destination layers use the same projection, then the pasted features will have geometry identical to the source layer. However if the destination layer is a different projection then \qg cannot guarantee the geometry is identical. This is simply because there are small rounding-off errors involved when converting between projections.}
Si vos couches sources et cibles utilisent la même projection, les entités collées 
auront la même géométrie que dans la couche source. Cependant, si la couche cible 
n'a pas la même projection, \qg ne peut garantir que les géométries seront 
identiques. Cela est simplement dû aux erreurs d'arrondissement faites lors de 
la conversion de projection.
\end{Tip}

%\minisec{Deleting Selected Features}
%\index{vector layers!deleting!feature}
\minisec{Supprimer des entités sélectionnées}
\index{couches vectorielles!effacer!entité}

% If we want to delete an entire polygon, we can do that by first selecting 
% the polygon using the regular \toolbtntwo{mActionSelect}{Select Features} tool. You can select 
% multiple features for deletion. Once you have the selection set, use the 
% \toolbtntwo{mActionDeleteSelected}{Delete Selected} tool to delete the features. 
Si nous voulons supprimer un polygone en entier, nous pouvons le faire en 
sélectionnant d'abord le polygone en utilisant l'outil \toolbtntwo{mActionSelect}
{Sélectionner les données}. Vous pouvez sélectionner plusieurs objets pour la 
suppression. Une fois le ou les objets sélectionnés, utilisez l'outil 
\toolbtntwo{mActionDeleteSelected}{Effacer la sélection} pour supprimer les 
entités.

% The \toolbtntwo{mActionEditCut}{Cut Features} tool on the digitizing toolbar can
% also be used to delete features. This effectively deletes the feature but
% also places it on a ``spatial clipboard". So we cut the feature to delete. 
% We could then use the \toolbtntwo{mActionEditPaste}{paste tool} to put it back, giving us a one-level undo 
% capability. Cut, copy, and paste work on the currently selected features, 
% meaning we can operate on more than one at a time.
L'outil \toolbtntwo{mActionEditCut}{Couper Entités} de la barre d'outils 
numérisation peut également être utilisé pour supprimer des entités. Ceci 
supprime effectivement les entités et les place également dans un presse-papier 
spatial. Donc nous coupons les entités pour les supprimer. Nous pouvons ensuite 
utiliser l'outil \toolbtntwo{mActionEditPaste}{Coller Entités} pour les 
récupérer, nous donnant alors la capacité d'annuler une fois les changements. 
Couper, copier et coller marchent sur les entités sélectionnées ce qui signifie 
que nous pouvons travailler sur plus d'un objet à la fois.

%\begin{Tip}[ht]\caption{\textsc{Feature Deletion Support}}.
\begin{Tip}[ht]\caption{\textsc{Gestion de la suppression d'entités}}
%\qgistip{When editing ESRI shapefiles, the deletion of features only works if \qg is linked to a GDAL version 1.3.2 or greater. The OS X and Windows versions of \qg available from the download site are built using GDAL 1.3.2 or higher.}
Lors de l'édition de shapefile, la suppression d'entités ne fonctionne que si 
\qg est lié à une version 1.3.2 ou supérieure de GDAL. Les versions OS X et 
Windows de \qg disponibles depuis le site de téléchargement incluent GDAL 1.3.2 
ou supérieur.
\end{Tip}

%\minisec{Saving Edited Layers}
%\index{editing!saving changes}
\minisec{Sauvegarder les couches éditées}
\index{éditer!sauvegarder des changements}

%When a layer is in editing mode, any changes remain in the memory of \qg.
%Therefore they are not committed/saved immediately to the data source or disk.
%If you want to save edits to the current layer but want to continue editing
%without leaving the editing mode, you can click the
%\toolbtntwo{mActionFileSave}{Save Edits} button. When you turn editing mode
%off with the \toolbtntwo{mActionToggleEditing}{Toggle editing} (or quit
%\qg for that matter), you are also asked if you want to save your changes
%or discard them.
Quand une couche est en mode édition, tous les changements sont stockés en 
mémoire par \qg. Ils ne sont pas sauvegardés immédiatement dans la source de 
données ou sur le disque. Lorsque vous déactivez le mode édition en cliquant 
sur \toolbtntwo{mActionToggleEditing}{Basculer en mode édition}(ou quittez \qg), 
il vous est demandé si vous souhaitez sauvegarder les changements ou les annuler. 
Si vous voulez enregistrer les modifications sans quitter le mode d'édition 
alors il vous faut cliquer sur le bouton \toolbtntwo{mActionFileSave}{Sauvegarder 
les modifications}

%If the changes cannot be saved (e.g. disk full, or the attributes have values that are out of range), the \qg in-memory state is preserved.  This allows you to adjust your edits and try again.
Si les changements ne peuvent pas être sauvés (par exemple à cause d'un disque 
plein ou des valeurs d'attributs dépassant la plage prévue), l'état de la 
mémoire de \qg est préservé. Cela vous permet d'ajuster vos éditions et réessayer.

\begin{Tip}[ht]\caption{\textsc{Intégrité des données}}
C'est toujours une bonne idée de sauvegarder vos sources de données avant de les 
éditer. Bien que les auteurs de \qg s'efforcent de préserver l'intégrité de vos 
données, il n'y pas de garantie à cet égard.
\end{Tip}

% \subsection{Advanced digitizing}
% \index{vector layers!advanced digitizing}
% \index{advanced digitizing!an existing layer}
% \label{sec:advanced_edit}

\subsection{Numérisation avancée}
\index{couches vectorielles!numérisation avancée}
\index{couches vectorielles!numérisation avancée!couche existante}
\label{sec:advanced_edit}

\begin{table}[h]\index{couches vectorielles!outils de numérisation avancée}
\centering
%\small
\begin{tabular}{|l|p{5.5cm}|l|p{5.5cm}|}
\hline \textbf{Icône} & \textbf{But} & \textbf{Icône} & \textbf{But} \\
\hline \includegraphics[width=0.7cm]{mActionUndo}
   & Défaire 
   & \includegraphics[width=0.7cm]{mActionRedo}
   & Refaire \\
\hline \includegraphics[width=0.7cm]{mActionSimplify}
   & Simplifier Entité
   & \includegraphics[width=0.7cm]{mActionAddRing}
   & Ajouter un anneau \\
\hline \includegraphics[width=0.7cm]{mActionAddIsland}
   & Ajouter une partie
   & \includegraphics[width=0.7cm]{mActionDeleteRing}
   & Effacer un anneau \\
\hline \includegraphics[width=0.7cm]{mActionDeletePart}
   & Effacer une partie
   & \includegraphics[width=0.7cm]{mActionReshape}
   & Remodeler une entité \\
\hline \includegraphics[width=0.7cm]{mActionSplitFeatures}
   & Couper une entité
   & \includegraphics[width=0.7cm]{mActionMergeFeatures}
   & Fusionner les entités sélectionnées \\
\hline \includegraphics[width=0.7cm]{mActionMergeFeatures}
   & Fusionner les attributs des entités sélectionnées
   &\includegraphics[width=0.7cm]{mActionRotatePointSymbols}
   & Rotation des symboles de point\\
\hline
\end{tabular}
\caption{Outils de numérisation avancée}\label{tab:advanced_editing}
\end{table}

\minisec{Annuler et refaire}
\index{couches vectorielles!annuler}
\index{couches vectorielles!refaire}

%The \toolbtntwo{mActionUndo}{Undo} and \toolbtntwo{mActionRedo}{Redo} tools
%allow the user to undo or redo vector editing operations. There is also a dockable
%widget, which shows all operations in the undo/redo history (see
%Figure \ref{fig:vector_redoundo}). This widget is not displayed by
%default; it can be displayed by right clicking on the toolbar and activating the
%Undo/Redo check box. Undo/Redo is however active, even if the widget is not
%displayed.
Les outils \toolbtntwo{mActionUndo}{Annuler} et \toolbtntwo{mActionRedo}{Refaire} 
permettent à l'utilisateur d'annuler ou revenir des opérations d'édition. Il y a 
également un panel ancrable, qui affiche les opérations dans l'historique annuler/refaire
(voir figure~\ref{fig:vector_redoundo}). Ce panel n'est pas affiché par défaut, 
mais peut l'être  par un clic droit sur une barre d'outils puis en cochant 
Annuler/Refaire. L'outil est actif même quand le panel n'est pas actif.
\par
%When Undo is hit, the state of all features and attributes are reverted to the
%state before the reverted operation happened. Changes other than normal vector
%editing operations (for example changes done by a plugin), may or may not be
%reverted, depending on how the changes were performed.
Quand on clique sur Annuler, l'état de toutes les entités retourne à l'état 
connu avant que les changements dus à une opération quelconque aient été appliqués. 
Les modifications autres que les opérations normales d'édition (par exemple des 
modifications réalisées par une extension) peuvent ou non être annulées, en fonction 
de la manière dont sont réalisées les modifications.
\par
%To use the undo/redo history widget simply click to select an operation in the
%history list; all features will be reverted to the state they were in after
%the selected operation.
Pour utiliser le panel de l'historique annuler/Refaire  cliquez simplement pour 
sélectionner une opération dans la liste de l'historique ; toutes les entités 
seront remises à l'état où elles étaient avant l'opération sélectionnée.

\begin{figure}[ht]
   \begin{center}   
   \includegraphics[clip=true, width=12cm]{redo_undo}
   \caption{Annuler et Refaire \nixcaption}\label{fig:vector_redoundo}
\end{center}
\end{figure}

% \minisec{Simplify Feature}
% \index{vector layers!simplify}
\minisec{Simplifier une entité}
\index{couches vectorielles!simplifier}

%The \toolbtntwo{mActionSimplify}{Simplify Feature} tool allows to reduce the
%number of vertices of a feature, as long as the geometry doesn't change. You
%need to select a feature, it will be highlighted by a red rubber band and a
%slider appears. Moving the slider, the red rubber band is changing its shape
%to show how the feature is being simplified. Clicking \button{OK} the new,
%simplified geometry will be stored. If a feature cannot be simplified (e.g.
%MultiPolygons), a message shows up.
L'outil \toolbtntwo{mActionSimplify}{Simplifier une entité} permet de réduire le nombre de sommets qui composent une entité aussi longtemps que cela ne change pas le type de géométrie. Vous devez sélectionner une ou plusieurs entités qui seront alors surlignées par un contour rouge, une barre coulissante est affichée pour choisir le degré de simplification que vous désirez appliquer (le contour rouge reflète la forme que vous obtiendrez). Cliquez sur \button{OK} et la nouvelle forme sera retenue. Si une entité ne peut être simplifiée (p. ex. un polygone multiple), un message vous le signalera.

% \minisec{Add Ring}
% \index{vector layers!add!ring}
\minisec{Ajouter un anneau}
\index{couches vectorielles!ajouter!anneau}

% You can create ring polygons using the \toolbtntwo{mActionAddRing}{Add Ring}
% icon in the toolbar. This means inside an existing area it is
% possible to digitize further polygons, that will occur as a 'hole', so only
% the area in between the boundaries of the outer and inner polygons remain as
% a ring polygon.

Vous pouvez créer des anneaux de polygones en utilisant l'icône \toolbtntwo{mActionAddRing}{Ajouter un anneau} Cela signifie qu'il est possible de dessiner des polygones à l'intérieur d'une zone existante et d'en tirer un trou,  seule la zone entre les limites externes des polygones sera conservée.

% \minisec{Add Part}
% \index{vector layers!add!island}
\minisec{Ajouter une partie}
\index{couches vectorielles!ajouter!partie}

%You can \toolbtntwo{mActionAddIsland}{add part} polygons to a selected multipolygon.
%The new part polygon has to be digitized outside the selected multipolygon.
Vous pouvez \toolbtntwo{mActionAddIsland}{ajouter une partie} à un multipolygone sélectionné. La nouvelle île doit être dessinée en dehors de celui-ci.

% \minisec{Delete Ring}
% \index{vector layers!delete!ring}
\minisec{Effacer un anneau}
\index{couches vectorielles!effacer!anneau}

% The \toolbtntwo{mActionDeleteRing}{Delete Ring} tool allows to delete ring
% polygons inside an existing area. This tool only works with polygon layers. 
% It doesn't change anything when it is used on the outer ring of the polygon. 
% This tool can be used on polygon and mutli-polygon features.Before
% you select the vertices of a ring, adjust the vertex edit tolerance.
L'outil \toolbtntwo{mActionDeleteRing}{Effacer un anneau} permet de supprimer un anneau existant dans un polygone. Il ne change rien lorsqu’il est utilisé sur la bordure extérieure du polygone. Cet outil peut être utilisé sur un polygone ou un polygone multiple. Avant de sélectionner un sommet d'un anneau, ajustez la tolérance d'édition du sommet.

% \minisec{Delete Part}
% \index{vector layers!delete!part}
\minisec{Effacer une partie}
\index{couches vectorielles!effacer!partie}

%The \toolbtntwo{mActionDeletePart}{Delete Part} tool allows to delete parts
%from multifeatures (e.g. to delete polygons from a multipolygon feature). It
%won't delete the last part of the feature, this last part will stay untouched.
%This tool works with all multi-part geometries point, line and polygon. Before
%you select the vertices of a part, adjust the vertex edit tolerance.
L'outil \toolbtntwo{mActionDeletePart}{Effacer une partie} permet de supprimer des parties d'une entité multipartite (p.ex. un polygone composé de multiples polygones distincts). Cela n'effacera pas la dernière partie restante. Cet outil marche avec toutes les géométries multiparties.

% \minisec{Reshape Features}
% \index{vector layers!reshape!feature}
\minisec{Remodeler une entité}
\index{couches vectorielles!remodeler!entité}

%You can reshape line and polygon features using the
%\toolbtntwo{mActionReshape}{Reshape Features} icon on the toolbar. It
%replaces the line or polygon part from the first to the last intersection
%with the original line. With polygons this can sometime lead to unintended
%results. It is mainly useful to replace smaller parts of a polygon, not major
%overhauls and the reshapeline is not allowed to cross several polygon rings
%as this would generate an invalide polygon.
Il est possible de retoucher des lignes ou des polygones grâce à l'outil\\ 
\toolbtntwo{mActionReshape}{Remodeler une entité}. Vous pouvez changer la forme 
d'une ligne ou d'un polygone en traçant une nouvelle forme entre 2 sommets, la 
modification viendra s'ajouter à l'existant ou le remplacer selon la taille de 
l'intervalle entre le premier sommet et celui clôturant le remodelage. Cette 
méthode convient pour remplacer de petites portions d'une entité, la ligne de 
remodelage n'est pas autorisée à croiser plusieurs anneaux de polygones, car 
cela générerait un polygone invalide.

%For example, you can edit the boundary of a polygon with this tool. First,
%click in the inner area of the polygon next to the point where you want to 
%add a new vertex. Then, cross the boundary and add the vertices outside the
%polygon. To finish, right-click in the inner area of the polygon. The tool
%will automatically add a node where the new line crosses the border. It is 
%also possible to remove part of the area from the polygon, starting the new
%line outside the polygon, adding vertices inside, and ending the line outside
%the polygon with a right click.
Par exemple, vous pouvez éditer la frontière d'un polygone avec cet outil. D'abord, 
cliquez sur la bonne interne du polygone près du point où vous souhaitez placer 
un nouveau sommet. Puis, croisez la frontière et ajoutez un sommet en dehors du 
polygone. Pour terminer, faites un clic droit dans la zone interner du polygone. 
L'outil va automatiquement ajouter un noeud où la nouvelle ligne a traversé la 
frontière. Il est également possible de supprimer une partie du polygone, en 
débutant la nouvelle ligne à l'extérieure du polygone, ajouter un sommet à 
l'intérieure et en terminant la ligne en dehors du polygone avec un clic droit.
 
% \textbf{Note}: The reshape tool may alter the starting position of a polygon
% ring or a closed line. So the point that is represented 'twice' will not be
% the same any more. This may not be a problem for most applications, but it is
% something to consider.
\textbf{Note}: L'outil de remodelage peut altérer la position de départ d'un 
anneau polygonal ou d'une ligne close, le point "double" ne sera plus le même. 
Ce n'est pas un problème pour la plupart des applications, mais c'est quelque 
chose à considérer.

% \minisec{Split Features}
% \index{vector layers!split!feature}
\minisec{Couper une entité}
\index{couches vectorielles!couper!entité}

%You can split features using the \toolbtntwo{mActionSplitFeatures}{Split
%Features} icon on the toolbar. Just draw a line across the feature you
%want to split.
Vous pouvez diviser une entité en utilisant le bouton \toolbtntwo{mActionSplitFeatures}{Couper Entités} situé dans la barre de numérisation. Pour couper, dessinez une ligne en travers de l'entité avec cet outil et terminez avec un clic droit.

% \minisec{Merge selected features}
% \index{vector layers!merge!features}
\minisec{Fusionner les entités sélectionnées}
\index{couches vectorielles!fusionner!entité}

% The \toolbtntwo{mActionMergeFeatures}{Merge Selected Features} tool allows to
% merge features that have common boundaries and the same attributes.  
L'outil \toolbtntwo{mActionMergeFeatures}{Fusionner les entités sélectionnées} permet de combiner des entités ayant une bordure commune et des attributs similaires.

%\minisec{Merge attributes of selected features}
%\index{vector layers!merge!attributes of features}
\minisec{Fusion des attributs d'entités sélectionnées}
\index{couches vectorielles!fusion!attributs des entitées}

%The \toolbtntwo{mActionMergeFeatures}{Merge Attributes of Selected Features} 
%tool allows to merge attributes of features with common boundaries and 
%attributes without merging their boundaries.
L'outil \toolbtntwo{mActionMergeFeatures}{Fusion des attributs d'entités sélectionnées} 
permet de fusionner des attributs d'entités avec des frontières communes et des 
attributs sans fusionner leurs frontières.  

%\minisec{Rotate Point Symbols}
\minisec{Rotation d'un symbole de point}
%\index{vector layers!rotate!symbol}
\index{couches vectorielles!rotation!symbole}

%%% FIXME change, if support in new symbology is available, too
%The \toolbtntwo{mActionRotatePointSymbols}{Rotate Point Symbols} tool is 
%currently only supported by the old symbology engine. It allows to change the 
%rotation of point symbols in the map canvas, if you have defined a rotation 
%column from the attribute table of the point layer in the \tab{Style} tab of 
%the \dialog{Layer Properties}. Otherwise the tool is inactive.
L'outil \toolbtntwo{mActionRotatePointSymbols}{Rotation d'un symbole de point} est 
pour le moment seulement géré par l'ancien système de style. Il permet de 
modifier l'orientation d'un symbole de point sur le canevas de la carte, si vous 
avez défini une colonne attributaire contenant l'orientation dans le panneau 
\tab{Style} de la fenêtre des \dialog{Propriétés de la couche}. Dans le cas 
contraire, l'outil restera inactif.

\begin{figure}[ht]
   \centering
   \includegraphics[clip=true, width=6cm]{rotatepointsymbol}
%   \caption{Rotate Point Symbols \nixcaption}\label{fig:rotatepoint}
   \caption{Rotation d'un symbole de point \nixcaption}\label{fig:rotatepoint}
\end{figure}

%To change the rotation, select a point feature in the map canvas and rotate
%it holding the left mouse button pressed. A red arrow with the rotation value
%will be visualized (see Figure~\ref{fig:rotatepoint}). When you release the
%left mouse button again, the value will be updated in the attribute table.
Pour changer l'orientation, sélectionnez une entité ponctuelle sur le canevas et faites la tourner en gardant le bouton gauche de votre souris appuyé. Une flèche 
rouge avec la valeur de rotation est visible (voir figure \ref{fig:rotatepoint}). 
Lorsque vous relâchez le bouton, la valeur sera mise à jour dans la table 
attributaire.

%\textbf{Note}: If you hold the \keystroke{Ctrl} key pressed, the rotation will be done
%in 15 degree steps.
\textbf{Note}: Si vous gardez la touche \keystroke{Ctrl} enfoncée, la rotation 
se fera par paliers de 15 degrés.

%\subsection{Creating a new Shapefile and Spatialite layer}\label{sec:create shape}
%\index{editing!creating a new shape layer}
\subsection{Créer de nouvelles couches Shapefile et Spatialite}\label{sec:create 
shape}\index{éditer!créer une nouvelle couche .shp}

%\qg allows to create new Shapefile layers and new Spatialite layers.
%Creation of a new GRASS layer is supported within the GRASS-plugin. Please refer
%to section \ref{sec:creating_new_grass_vectors} for more information on
%creating GRASS vector layers.
La création de couches GRASS est gérée par l'intermédiaire de l'extension GRASS. 
Référez-vous à la section \ref{sec:creating_new_grass_vectors} pour plus 
d'informations sur ce sujet.

%\subsection{Creating a New Layer}\label{sec:create shape}\index{editing!creating a new layer}
\subsection{Créer une nouvelle couche Shapefile}\label{sec:create shape}
\index{éditer!créer une nouvelle couche}

%To create a new layer for editing, choose \toolbtntwo{mActionNewVectorLayer}{New Vector Layer} from the \mainmenuopt{Layer} menu. The \dialog{New Vector Layer} dialog will be displayed as shown in Figure \ref{fig:newvectorlayer}. Choose the type of layer (point, line or polygon) and the CRS (Coordinate Reference System)
Pour créer une nouvelle couche Shapefile à éditer, allez dans 
\toolbtntwo{mActionNewVectorLayer}{Nouvelle couche Shapefile} du menu 
\mainmenuopt{Couche}. La fenêtre \dialog{Nouvelle couche vecteur} apparaitra 
telle que montrée dans la figure \ref{fig:newvectorlayer}. Choisissez le type de 
géométrie de la couche (point, ligne ou polygone) et la projection (CRS Coordinate 
Reference System).

\begin{figure}[ht]
  \begin{center}
  %\caption{Creating a New Vector Dialog \nixcaption}\label{fig:newvectorlayer}\smallskip
  \includegraphics[clip=true, width=7cm]{editNewVector}
  \caption{Fenêtre Nouvelle couche Shapefile \nixcaption}\label{fig:newvectorlayer}
\end{center}
\end{figure}

%Note that \qg does not yet support creation of 2.5D features (i.e. features with 
%X,Y,Z coordinates) or measure features. At this time, only shapefiles can be 
%created. In a future version of \qg, creation of any OGR or \psq layer type 
%will be supported.
Notez que \qg ne gère pas encore la création d'entité 2.5D (c.-à-d. des entités 
avec des coordonnées X, Y, Z). Pour le moment, seuls des shapefiles peuvent être 
créés. Dans une version future de \qg, la création de n'importe format de couches 
géré par OGR ou \psq sera possible.

%To complete the creation of the new Shapefile layer, add the desired attributes by
%clicking on the \button{Add} button and specifying a name and type for the
%attribute. A first 'id' column is added as default but can be removed, if not 
%wanted. Only \selectstring{Type}{real}, \selectstring{Type}{integer}, and
%\selectstring{Type}{string} attributes are supported. Additionally and
%according to the attribute type you can also define the width and precision
%of the new attribute column. Once you are happy with the attributes, click
%\button{OK} and provide a name for the shapefile. \qg will automatically add
%a \filename{.shp} extension to the name you specify. Once
%the layer has been created, it will be added to the map and you can edit it in
%the same way as described in Section \ref{sec:edit_existing_layer} above.
Pour terminer la création de la nouvelle couche, ajouter les attributs désirés 
en cliquant sur le bouton \button{Ajouter un attribut} et en spécifiant le nom 
et le type de l'attribut. Une première colonne 'id' est ajoutée par défaut, mais 
peut être supprimée si nécessaire. Seuls les attributs de type \selectstring{Type}{réel}, 
\selectstring{Type}{entier}, et \selectstring{Type}{string} sont gérés. De plus, 
selon le type d'attribut vous pouvez définir la largeur et la précision de la 
nouvelle colonne. Une fois satisfait de vos attributs, cliquez sur \button{OK} 
et donnez un nom pour le shapefile. \qg va automatiquement ajouter l'extension 
\filename{.shp} au nom que vous lui avez spécifié. Une fois la couche créée, 
elle sera ajoutée à la carte et vous pouvez l'éditer de la manière décrite dans 
la Section \ref{sec:edit_existing_layer} ci-dessus.

%\minisec{Creating a new SpatiaLite layer}\label{sec:create spatialite}\index{editing!creating a new spatialite layer}
\minisec{Créer une nouvelle couche SpatiaLite}\label{sec:create spatialite}\index{éditer!créer une nouvelle couche spatialite}

%To create a new SpatiaLite layer for editing, choose \button{new} \arrow
%\toolbtntwo{mActionNewVectorLayer}{New SpatiaLite Layer} from the
%\mainmenuopt{Layer} menu. The \dialog{New SpatiaLite Layer} dialog will be
%displayed as shown in Figure \ref{fig:newspatialitelayer}.
Pour créer une nouvelle couche SpatiaLite à éditer, allez dans 
\toolbtntwo{mActionNewVectorLayer}{Nouvelle couche SpatiaLite} du menu 
\mainmenuopt{Couche}. La fenêtre \dialog{Nouvelle couche SpatiaLite} apparaitra 
telle que montrée dans la figure \ref{fig:newspatiaLitelayer}.

\begin{figure}[ht]
   \centering
   \includegraphics[clip=true, width=7cm]{editNewSpatialite}
%   \caption{Creating a New Spatialite layer Dialog \nixcaption}\label{fig:newspatialitelayer}
   \caption{Fenêtre de création d'une nouvelle couche Spatialite \nixcaption}\label{fig:newspatialitelayer}
\end{figure}

%First step is to select an existing Spatialite database or to create a new
%Spatialite database. This can be done with the browse \button{...} button
%to the right of the database field. Then add a name for the new layer and
%define the layer type and the EPSG SRID. If desired you can select to
%\checkbox{create an autoincrementing primary key}.

Choisissez ensuite une base de données SpatiaLite existante ou bien procédez à 
la création d'une nouvelle base en utilisant le bouton \button{...} à droite de 
la liste des bases de données. Ajoutez ensuite un nom pour cette nouvelle couche 
et définissez son type de géométrie ainsi que le SCR (EPSG). Si vous le désirez,
 vous pouvez cocher la case \checkbox{créer une clé primaire autoincrémentée}.

%To define an attribute table for the new Spatialite layer, add the names
%of the attribute columns you want to create with the according column type
%and click on the \button{Add to attribute list} button. Once you are happy
%with the attributes, click \button{OK}. \qg will automatically add the new
%layer to the legend and you can edit it in the same way as described in
%Section \ref{sec:edit_existing_layer} above.

Pour défnir une table attributaire, ajoutez les noms des colonnes avec leur type 
de données et cliquez sur le bouton \button{OK}. \qg ajoutera automatiquement 
cette nouvelle couche à la légende où vous pourrez l'éditer comme indiqué dans 
la section \ref{sec:edit_existing_layer}.

%The spatialite creation dialog allows to create multiple layers without
%closing the dialog when you click \button{Apply}.
La fenêtre de création permet de créer plusieurs couches et colonnes attributaires 
en une fois, les modifications ne sont faites que lorsque vous cliquez sur le 
bouton \button{Appliquer}.

%\subsection{Working with the Attribute Table}\label{sec:attribute table}\index{editing!working with the attribute table}
\subsection{Travailler avec la table attributaire}\label{sec:attribute table}
\index{éditer!travailler avec la table attributaire}

%The attribute table displays features of a selected layer. Each row in the table 
%represents one map feature and each column contains a particular piece of
%information about the feature. Features in the table can be searched, selected,
%moved or even edited.
La table attributaire affiche les entités de la couche sélectionnée. Chaque 
ligne représente une entité et chaque colonne contient un morceaux d'information 
de l'entité. Les entités de la table peuvent être recherchées, sélectionnées, 
déplacées et éditées.

%To open the attribute table for a vector layer, make the layer active by clicking 
%on it in the map legend area. Then use \mainmenuopt{Layer} from the main menu 
%and and choose \dropmenuopttwo{mActionOpenTable}{Open Attribute Table} 
%from the menu. It is also possible to rightlick on the layer and 
%choose \dropmenuopttwo{mActionOpenTable}{Open Attribute Table} from the 
%dropdown menu. 
Pour ouvrir la table attributaire d'une couche vecteur, activez la couche en 
cliquant dessus depuis la zone de légende de la carte. Puis dans le menu 
\mainmenuopt{Couche}, faites \dropmenuopttwo{mActionOpenTable}{Ouvrir la table 
d'attribut}. Vous pouvez aussi y accéder avec un clic droit sur la couche.

%This will open a new window which displays the attributes for 
%every feature in the layer (figure \ref{fig:attributetable}). The number of features
%are shown in the attribute table title.
Cela 
ouvrira une nouvelle fenêtre qui comportera les attributs de toutes les entités 
de la couche (voir figure \ref{fig:attributetable}). Le nombre des entités est 
affiché dans la barre de titre de la table attributaire.

\begin{figure}[ht]
   \begin{center}
   \includegraphics[clip=true, width=12cm]{vectorAttributeTable}
    \caption{Table d'attributs pour la couche Alaska \nixcaption}\label{fig:attributetable}
\end{center} 
\end{figure}

\minisec{Sélectionner des entités depuis la table}

%\textbf{Each selected row} in the attribute table displays the attributes of a
%selected feature in the layer. If the set of features selected in the main window
%is changed, the selection is also updated in the attribute table.
%Likewise, if the set of rows selected in the attribute table is changed, the
%set of features selected in the main window will be updated.
\textbf{Chaque ligne sélectionnée} dans la table attributaire affiche les 
attributs d'une entité sélectionnée dans la couche. Si un ensemble d'entités 
sélectionnées dans la fenêtre principale est modifié, la sélection est également 
mise à jour dans la table attributaire. De même, si un ensemble de lignes 
sélectionnées dans la table attributaire est modifié, l'ensemble d'entités 
sélectionnées dans la fenêtre principale sera mis à jour.

%Rows can be selected by clicking on the row number on the left side of the 
%row. \textbf{Multiple rows} can be marked by holding the \keystroke{Ctrl} key. A
%\textbf{continuous selection} can be made by holding the \keystroke{Shift} key and
%clicking on several row headers on the left side of the rows. All rows between the
%current cursor position and the clicked row are selected. Moving the cursor
%position in the attribute table, by clicking a cell in the table, does not change
%the row selection. Changing the selection in the main canvas does not move the
%cursor position in the attribute table.
Les lignes peuvent être sélectionnées en cliquant sur le numéro de ligne placé 
tout à gauche. \textbf{Plusieurs lignes} peuvent être retenues en maintenant la 
touche \textbf{Ctrl}. \textbf{Une sélection continue} s'effectue en gardant 
appuyée la touche \textbf{Shift} et en cliquant sur une nouvelle ligne, toutes 
les lignes entre la première sélection et la dernière seront prises. Déplacer le 
curseur de la position dans la table attributaire, en cliquant sur une cellule 
dans la table ne change pas la sélection. Changer la sélection dans la carte ne 
modifie pas la position du curseur dans la table attributaire.

%The table can be sorted by any column, by clicking on the column header. A small
%arrow indicates the sort order (downward pointing means descending values from the
%top row down, upward pointing means ascending values from the top row down).
Une table peut être ordonnée sur toutes colonnes en cliquant sur l'en-tête. Une petite 
flèche indique l'ordre de tri (une flèche pointant vers le bas indiquera un tri 
descendant, vers le haut signifie des valeurs ascendantes du haut vers le bas).

%For a \textbf{simple search by attributes} on only one column the \button{Look for}
% field can be used. Select the field (column) from which the search should be
%performed from the dropdown menu and hit the \button{Search} button. The matching
%rows will be selected and the total number of
%matching rows will appear in the title bar of the attribute table, and in the
%status bar of the main window. For more complex searches use
%the Advanced search \button{...}, which will launch the Search Query Builder
% described in Section \ref{sec:select_by_query}.
Pour une \textbf{recherche par attribut simple} sur une seule colonne le champ 
\button{Chercher} peut être utilisé. Sélectionnez le champ (colonne) dans lequel 
la recherche doit avoir lieu dans la liste déroulante et cliquez sur le bouton 
\button{Chercher}. Les lignes qui correspondent seront sélectionnées et le nombre 
total de ligne apparaitra dans la barre de titre de la table attributaire et dans 
la barre de statut de la fenêtre principale. Pour des recherches plus complexes 
utiliser le bouton \button{Recherche avancée} qui lancera le constructeur de 
requête de recherche décrit à la section~ \ref{sec:select_by_query}.

%To show selected records only, use the checkbox \checkbox{Show selected records only}. To search selected records only, use the checkbox \checkbox{Search selected records only}. The other buttons at the bottom left of the attribute table window provide following functionality: 
Pour afficher uniquement les enregistrements que vous avez sélectionnés, utiliser 
la boite à cocher \checkbox{Montrer seulement les enregistrements sélectionnés}. 
Pour limiter la recherche à la sélection, activer la boîte à cocher 
\checkbox{Rechercher seulement les enregistrements sélectionnés} Les autres 
boutons disposés à gauche de la fenêtre fournissent les fonctionnalités 
suivantes :

\begin{itemize}[label=--]
%\item \toolbtntwo{mActionOpenTable}{Unselect all} also with \keystroke{Ctrl-U}
\item \toolbtntwo{mActionOpenTable}{Desélectionner tout} ainsi que \keystroke{Ctrl-U}.
%\item \toolbtntwo{mActionSelectedToTop}{Move selected to top} also with 
%\keystroke{Ctrl-T}
\item \toolbtntwo{mActionSelectedToTop}{Déplacer la sélection au sommet} ainsi 
que \keystroke{Ctrl-T}.
%\item \toolbtntwo{mActionInvertSelection}{Invert selection} also with 
%\keystroke{Ctrl-S}
\item \toolbtntwo{mActionInvertSelection}{Inverser la sélection} ainsi que 
\keystroke{Ctrl-S}.
%\item \toolbtntwo{mActionCopySelected}{Copy selected rows to clipboard} 
%also with \keystroke{Ctrl-C}
\item \toolbtntwo{mActionCopySelected}{Copier les lignes sélectionnées dans le 
presse-papier} ainsi que \keystroke{Ctrl-C}.
%\item \toolbtntwo{mActionZoomToSelected}{Zoom map to selected rows} 
%also with \keystroke{Ctrl-J}
\item \toolbtntwo{mActionZoomToSelected}{Zoomer la carte sur les lignes 
sélectionnées} ainsi que \keystroke{Ctrl-J}.
%\item \toolbtntwo{mActionToggleEditing}{toggle editing mode} to edit single 
%values of attribute table also with \keystroke{Ctrl-E}
\item \toolbtntwo{mActionToggleEditing}{Activer le mode d'édition} pour modifier 
les valeurs des attributs ainsi que \keystroke{Ctrl-E}.
%\item \toolbtntwo{mActionDeleteSelected}{Delete Selected Features} also with 
%\keystroke{Ctrl-D}
\item \toolbtntwo{mActionDeleteSelected}{Effacer les entités sélectionnées} ainsi 
que \keystroke{Ctrl-D}. 
%\item \toolbtntwo{mActionNewAttribute}{New Column} for PostGIS layers and for 
%OGR layers with GDAL version >= 1.6 also with \keystroke{Ctrl-W}
\item \toolbtntwo{mActionNewAttribute}{Nouvelle colonne} pour les couches OGR 
(>=1.6) et PostGIS ainsi que \keystroke{Ctrl-W}
%\item \toolbtntwo{mActionDeleteAttribute}{Delete Column} only for PostGIS layers 
%yet also with \keystroke{Ctrl-L}
\item \toolbtntwo{mActionDeleteAttribute}{Effacer une colonne}, uniquement pour 
les couches PostGIS ainsi que \keystroke{Ctrl-L}
%\item \toolbtntwo{mActionCalculateField}{Open field calcultor} also with 
%\keystroke{Ctrl-I}
\item \toolbtntwo{mActionCalculateField}{Ouvrir la calculatrice de champ} ainsi 
que \keystroke{Ctrl-I}
\end{itemize}

%\minisec{Save selected features as new layer}
%\index{editing!save selection as new layer}
\minisec{Enregistrer les entités sélectionnées dans une nouvelle couches}
\index{éditer!enregistrer la sélection dans une nouvelle couche}

%The selected features can be saved as any OGR supported vector format and also
%transformed into another Coordinate Reference System (CRS). Just open the right mouse
%menu of the layer and click on \dropmenuopt{Save selection as} to define the
%name of the output file, its format and CRS (see Section \ref{label_legend}). It is 
%also possible to specify OGR creation options within the dialog.
Les entités sélectionnées peuvent être enregistrées dans un nouveau fichier (dans tout format supporté par OGR) et transformées dans n'importe quel système de coordonnées (SCR). Il suffit de faire un clic droit sur la couche où est la sélection dans la liste des couches, de cliquer sur \dropmenuopt{Sauvegarder la sélection sous} pour définir le nom du fichier en sortie, le format et le SCR (voir section \ref{label_legend}). Il est possible de définir des options de création OGR à cette étape.

%\begin{Tip}[ht]\caption{\textsc{Manipulating Attribute data}}
%\qgistip{Currently only \pg layers are supported for adding or dropping
%attribute columns within this dialog. In future versions of \qg, other
%datasources will be supported, because this feature was recently implemented
%in GDAL/OGR > 1.6.0
\begin{Tip}[ht]\caption{\textsc{Manipuler les données attributaires}}
Actuellement seules les couches \pg sont supportées pour l'ajout ou la 
suppression de colonnes. Les prochaines versions de \qg étendront ce support à 
d'autres sources de données grâce aux apports des versions de GDAL/OGR 
postérieures à la 1.6.0.
\end{Tip}

%\minisec{Working with non spatial attribute tables}
%\index{editing!working with non spatial tables}
\minisec{Travailler avec tables non-spatialisées}
\index{éditer!travailler avec tables non-spatiales}

%QGIS allows also to load non spatial tables. This includes currently tables supported 
%by OGR, delimited text and the PostgreSQL provider. The tables can be used for field 
%lookups or just generally browsed and edited using the table view. When you load the 
%table you will see it in the legend field. It can be opened e.g. with the 
%\dropmenuopttwo{mActionOpenTable}{Open Attribute Table} tool and is then editable 
%like any other layer attribute table. 
QGIS permet de charger des tables n'ayant pas d'informations spatiales, cela 
comprend les tables supportées par OGR, les fiches de texte délimité et le 
prestataire PostgreSQL. les tables peuvent être utilisées pour regarder les champs, 
pour des requêtes ou pour de l'édition. Lorsque vous chargez une table de ce 
type, elle apparaîtra dans la liste des couches, elle peut être ouverte avec 
l'outil \dropmenuopttwo{mActionOpenTable} comme une table attributaire.

%As an example you can use columns of the non spatial table to define attribute values or 
%a range of values that are allowed to be added to a specific vector layer during digitizing. 
%Have a closer look at the edit widget in section~\ref{label_attributes} to find out more.
Vous pouvez ainsi utiliser ces colonnes pour définir des valeurs d'attributs ou 
un intervalle de valeurs qui sont autorisées à être ajoutées à une couche 
vectorielle spécifique durant une numérisation. Jetez un oeil du côté de l'outil 
d'édition pour en savoir plus (voir section \ref{label_attributes}).

%\section{Query Builder}\label{sec:query_builder}
\section{Constructeur de requêtes}\label{sec:query_builder}
%\index{Query Builder}
\index{Constructeur de requêtes}

%The \button{Advanced search\dots} button opens the Query Builder and allows you 
%to define a subset of a table using a SQL-like WHERE clause, display the result 
%in the main window and save it as a Shapefile. For example, if you have a 
%\filename{towns} layer %with a \usertext{population} field you could select 
%only larger towns by entering \usertext{population > 100000} in the SQL box of 
%the query builder. Figure
%\ref{fig:query_builder} shows an example of the query builder populated with
%data from a \pg layer with attributes stored in PostgreSQL. 
%The Fields, Values and Operators sections help the user to construct the SQL-like

Le bouton de recherche avancée ouvre le constructeur de requêtes qui vous permet 
de définir un sous-ensemble de la table en utilisant une clause SQL de type 
WHERE et de l'afficher comme une couche dans \qg. Par exemple, si vous avez une 
couche \filename{towns} avec un champ \usertext{population}, vous pouvez 
sélectionner uniquement les plus grandes villes en entrant \usertext{population 
>\,\numprint{100000}} dans le cadre SQL du constructeur de requête. La figure 
\ref{fig:query_builder} montre un exemple de requête avec les données d'une
couche \pg dont les attributs sont stockés dans PostgreSQL. Les sections Champs, 
 Valeurs et Opérations aide l'utilisateur à construire la requête SQL. 

\begin{figure}[ht]
  \begin{center}
    %\caption{Query Builder \nixcaption}\label{fig:query_builder}\smallskip
    \includegraphics[clip=true, width=11.5cm]{queryBuilder}
    \caption{Constructeur de requêtes \nixcaption} \label{fig:query_builder}
  \end{center}
\end{figure}

%The \textbf{Fields list} contains all attributes of the attribute table to be 
%searched. To add an attribute to the SQL where clause field, double click its 
%name in the Fields list. Generally you can use the various fields, values and 
%operators to construct the query or you can just type it into the SQL box. 
La liste des champs contient tous les attributs de la table attributaire pouvant 
être parcourus par la recherche. Pour ajouter un attribut à la clause WHERE, 
double-cliquez son nom dans cette liste. Vous pouvez cliquer sur les différents 
champs, valeurs et opérateurs qui composent votre requête ou bien l'écrire 
directement dans le cadre SQL.

%The \textbf{Values list} lists the values of an attribute. To list all possible 
%values of an attribute, select the attribute in the Fields list and click the 
%\button{All} button\index{Query Builder!getting all values}. To list all values 
%of an attribute that are present in the sample table, select the attribute in 
%the Fields list and click the \button{Sample} button\index{Query Builder!generating sample list}. To add a value to the SQL 
%where clause field, double click its name in the Values list.   
La liste des valeurs recense toutes les valeurs d'un attribut. Pour en lister la 
totalité, sélectionnez l'attribut dans la liste de champs puis cliquez sur le 
bouton \button{Tout}\index{constructeur de requête!obtenir toutes les valeurs}. 
Pour lister toutes les valeurs présentes dans la liste d'échantillons, 
sélectionnez l'attribut puis le bouton \button{Échantillon }\index{constructeur 
de requête!générer une liste d'échantillon}. Pour ajouter une valeur à la clause 
WHERE et la prendre en compte dans la requête, il vous suffit de faire un 
double-clic dessus.

%The \textbf{Operators section} contains all usable operators. To add an operator 
%to the SQL where clause field, click the appropriate button. Relational operators 
%( = , > , \dots), string comparison operator ( LIKE ), logical operators ( AND , OR 
%, \dots) are available. 
La section des opérateurs contient toutes les opérations menables sur une 
recherche. Pour ajouter un opérateur à la requête SQL, cliquez sur le bouton 
approprié. Les opérateurs relationnels (=, >, \dots), les opérateurs de 
comparaison (LIKE), les opérateurs logiques ( AND, OR,\dots) sont disponibles.

%The 
%\button{Test} button shows a message box with the number of features satisfying 
%the current query, which is usable in the process of query construction. The 
%\button{Clear} button clears the text in the SQL where clause text field. The 
%\button{Save} and \button{Load} button allow to save and load SQL queries. The 
%\button{OK} button closes the window and selects the features satisfying the 
%query. The \button{Cancel} button closes the window without changing the current 
%selection. 
Le bouton \button{Effacer} nettoye le texte présent dans le cadre SQL. Le 
\button{Test} affiche une fenêtre comptabilisant le nombre d'entités satisfaisant 
à votre requête, vous permettant de savoir si votre requête fonctionne au fil 
de sa construction. Les outons \button{Save} et \button{Charger} permettent de 
sauver et de charger les requêtes SQL. Le bouton \button{OK} ferme la fenêtre et 
effectue la recherche définie. Le bouton \button{Annuler} clôt la fenêtre, mais 
sans la sélection en cours.

%\begin{Tip}\caption{\textsc{Changing the Layer Definition}}\index{Query Builder!changing layer definitions}
\begin{Tip}\caption{\textsc{Changer la définition d'une couche}}\index{Constructeur 
de requête!changer des définitions de couche.}

%\qgistip{You can change the layer definition after it is loaded by altering the SQL query used to define the layer. To do this, open the  vector \dialog{Layer Properties} dialog by double-clicking on the layer in the legend and click on the \button{Query Builder} button on the \tab{General} tab. See Section \ref{sec:vectorprops} for more information.}
Vous pouvez changer la définition d'une couche après son chargement en modifiant 
la requête SQL utilisée pour définir la couche. Pour faire cela, ouvrez la fenêtre 
\dialog{Propriétés de la couche} en double-cliquant sur la couche dans la légende 
puis cliquez sur le bouton \button{Constructeur de requête} dans l'onglet 
\tab{Général}. Voir Section \ref{sec:vectorprops} pour plus d'informations.
\end{Tip}

%\section{Select by query}\label{sec:select_by_query}
%\index{PostgreSQL!query builder}
%\index{PostGIS!query builder}
%\index{query builder!PostgreSQL}
%\index{query builder!PostGIS}
\section{Sélection par requête}\label{sec:select_by_query}
\index{PostgreSQL!constructeur de requête}
\index{PostGIS!constructeur de requête}
\index{constructeur de requête!PostgreSQL}
\index{constructeur de requête!PostGIS}

%With \qg it is possible also to select features using a similar query builder interface to that used in \ref{sec:query_builder}. In the above section the purpose of the query builder is to only show features meeting the filter criteria as a 'virtual layer' / subset. The purpose of the select by query function is to highlight all features that meet a particular criteria. Select by query can be used with all vector data providers.
Dans \qg, il est possible de sélectionner des entités en utilisant une interface 
similaire à celle du constructeur de requêtes utilisé dans \ref{sec:query_builder}. 
Dans la section ci-dessus, le but du constructeur de requêtes était seulement de 
montrer les entités répondant aux critères de filtre comme une couche virtuelle/ 
sous-ensemble. Le but de la fonction de sélection par requête est de surligner 
toutes les entités qui répondent à un critère particulier. La sélection par 
requête peut être utilisée sur tous les prestataires de données vectorielles.

%To do a `select by query' on a loaded layer, click on the button \toolbtntwo{mActionOpenTable}{Open Table} to open the attribute table of the layer. Then click the \button{Advanced...} button at the bottom. This starts the Query Builder that allows to define a subset of a table and display it as described in Section \ref{sec:query_builder}.
Pour faire une sélection par requête sur une couche chargée, cliquez sur le 
bouton\\ \toolbtntwo{mActionOpenTable}{Ouvrir la table des attributs} pour ouvrir 
la table de la couche. Ensuite cliquez sur le bouton \button{Avancée\dots} en bas 
de la fenêtre. Cela lance le Constructeur de requête qui permet de définir un 
sous-ensemble de la table et l'affiche comme décrit dans la section~
\ref{sec:query_builder}.

%\section{Field Calculator}\label{sec:field_calculator}
\section{Calculatrice de champ}\label{sec:field_calculator}
\index{PostgreSQL!calculatrice de champ}
\index{PostGIS!calculatrice de champ}
\index{OGR!calculatrice de champ}
\index{calculatrice de champ!PostgreSQL}
\index{calculatrice de champ!PostGIS}
\index{calculatrice de champ!OGR}

%The \toolbtntwo{mActionCalculateField}{Field Calculator} button in the
%attribute table allows to perform calculations on basis of existing
%attribute values or defined functions, e.g to calculate length or area
%of geometry features. The results can be written to a new attribute column
%or it can be used to update values in an already existing column. The creation
%of new attribute fields is currently only possible in PostGIS and with OGR
%formats, if GDAL version is >= 1.6.0.
Le bouton \toolbtntwo{mActionCalculateField}{Calculatrice de champs} de la table 
attributaire permet d'opérer des calculs sur la base des valeurs attributaires 
ou d'utiliser des fonctions, p. ex. pour calculer la longueur ou la surface 
d'entités. Les résultats peuvent être écrits dans une nouvelle colonne attributaire 
ou mettre à jour une colonne existante. La création d'une nouvelle colonne est 
uniquement possible sous PostGIS et avec les formats d'OGR (>= 1.6.0).

%You have to bring the vector layer in editing mode, before you can click on
%the field calculator icon to open the dialog (see Figure
%\ref{fig:field_calculator}). In the dialog you first have to select, whether
%you want to update an existing field, only update selected features or
%create a new attribute field, where the results of the calculation will be added.
Vous devez basculer la couche vectorielle en mode d'édition avant de pouvoir 
cliquer sur le bouton de la calculatrice (voir figure \ref{fig:field_calculator}). 
Il vous faut d'abord choisir si une nouvelle colonne doit être créée ou une 
autre mise à jour.

\begin{figure}[ht]
  \centering
    \includegraphics[clip=true, width=11.5cm]{fieldcalculator}
%    \caption{Field Calculator \nixcaption}\label{fig:field_calculator}    
    \caption{Calculatrice de champs \nixcaption}\label{fig:field_calculator}
\end{figure}

%If you choose to add a new field, you need to enter a field name, a field type
%(integer, real or string), the total field width, and the field precision.
%For example, if you choose a field width of 10 and a field precision of 3 it
%means you have 6 signs before the dot, then the dot and another 3 signs for the
%precision.
Si vous choisissez d'ajouter un nouveau champ, vous devez lui donner un nom, un 
type (entier, flottant ou chaîne de caractère), une longueur et sa précision. Par 
exemple, vous pouvez créer un champ d'une longueur de 10 doté d'une précision 
de 3, ce qui signifie que vous aurez 6 chiffres avant la virgule, la virgule puis 
3 autres chiffres, soit 10 caractères au total.

%The \textbf{Fields list} contains all attributes of the attribute table to be
%searched. To add an attribute to the Field calculator expression field, double
%click its name in the Fields list. Generally you can use the various fields,
%values and operators to construct the calculation expression or you can just
%type it into the box.
La liste des \textbf{champs} comporte tous les attributs de la table pouvant 
être recherchés. Pour ajouter un attribut dans une expression, double-cliquez 
sur la liste des champs. Vous pouvez utiliser les champs, valeurs et opérateurs 
pour construire la formule de calcul, ou bien tout saisir manuellement.

%The \textbf{Values list} lists the values of an attribute field. To list all
%possible values, select the attribute field in the Fields list and click the
%\button{All} button\index{Field Calculator!getting all values}. To list all
%values of an attribute field that are present in the sample table, select the
%attribute in the Fields list and click the \button{Sample} button\index{Field
%Calcultor!generating sample list}. The procedure is the same as for the Query
%Builder. To add a value to the Field calculator expression box, double click its
%name in the Values list.
La zone des \textbf{valeurs} liste les valeurs du champ. Pour lister toutes les 
valeurs possibles, sélectionnez l'attribut puis cliquez sur le bouton \button{Tout}. 
Vous pouvez limiter la liste des valeurs à un court échantillon en cliquez cette 
fois sur le bouton \button{Echantillon}. Pour ajouter une valeur à l'expression 
en cours de rédaction, un double-clic sur le nom de la valeur suffira.

%The \textbf{Operators section} contains all usable operators. To add an operator
%to the Field calculator expression box, click the appropriate button. Mathematical
%calculations ( + , - , * \dots), trigonometric functions ( sin, cos, tan, \dots),
%extract geometric information ( length and area ) together with
%concatenator (||) and row counter. Stay tuned for more operators to come!
La zone des \textbf{opérateurs} contient tous les opérateurs utilasble sur la 
couche. Pour ajouter un opérateur à une expression, faites un clic sur le bouton 
de l'opérateur désiré. Les opérateurs disponibles actuellement sont les calculs 
mathématiques ( + , - , *,\dots), trigonométriques (sin, cos,tan,\dots), 
géométriques (longueur ou surface), la concaténation\footnote{Ce terme signifie 
la mise bout à bout de chaînes de caractères, p. ex. "Brian is in" || "the kitchen" 
donnera "Brian is in the kitchen".} (||) et le décompte des lignes. Cette liste 
sera encore étendue dans les prochaines versions de \qg.

%A short example illustrates how the field calculator works. We want to calculate
%the length of the 'railroads' layer from the \filename{\qg\_example\_dataset}:
Un court exemple pour illustrer la manière dont la calculatrice fonctionne. Nous 
voulons calculer la longueur des lignes de la couche 'railroads' de l'échantillon 
de données de \qg.

\begin{enumerate}
%\item Load the Shapefile \filename{railroads.shp} in \qg and open
%the \dialog{Attribute Table} dialog.
\item Chargez le fichier Shapefile \filename{railroads.shp} dans \qg et ouvrez 
sa \dialog{Table d'Attributs}
%\item Click on \toolbtntwo{mActionToggleEditing}{Toggle editing mode} and
%open the \toolbtntwo{mActionCalculateField}{Field Calculator} dialog.
\item Cliquez sur\toolbtntwo{mActionToggleEditing}{Basculer en mode édition} et 
ouvrez la \toolbtntwo{mActionCalculateField}{Calculatrice de champs}
%\item Unselect the \checkbox{Update existing field} checkbox to enable the
%new field box.
\item Désélectionnez la case \checkbox{Champ de mise à jour existant} puis cochez 
la case de création d'un nouveau champ.
%\item Add 'length' as output field name, 'real' as output field type and define
%output field width 10 and a precision of 3.
\item Ajoutez 'longueur' dans le nom de ce champ, 'réel' en tant que type et 
définissez une longueur de 10 et une précision de 3.
%\item Now click on Operator 'length' to add it as \$length into the
%field calculator expression box and click \button{Ok}.
\item Maintenant cliquez sur l'opérateur 'longueur' pour l'ajouter dans 
l'expression (sous la forme \$length) et cliquez \button{Ok}.
\end{enumerate}

%Due to limited space screeen, not all the operators are available through the 
%buttons. They are all listed in the following table.

\begin{center}
{\setlength{\extrarowheight}{10pt}
\small
\begin{longtable}{|p{4cm}|p{10cm}|}
%\hline \multicolumn{2}{|c|}{\textbf{List of operators supported by the field calculator}}\\
\hline \multicolumn{2}{|c|}{\textbf{Liste d'opérateurs gérés par le calculateur 
de champs}}\\
%\hline \textbf{String}&\textbf{Literal string value}\\
\hline \textbf{String}&\textbf{Valeur de la chaîne littérale}\\
\endfirsthead
%\hline \textbf{String}&\textbf{Literal string value}\\
\hline \textbf{String}&\textbf{Valeur de la chaîne littérale}\\
\endhead
%\hline \multicolumn{2}{|r|}{{See next page}} \\ \hline
\hline \multicolumn{2}{|r|}{{Voir page suivante}} \\ \hline
\endfoot
\endlastfoot
%\hline NULL & null value \\
\hline NULL & valeur null \\
%\hline sqrt(\textit{a}) & square root \\
\hline sqrt(\textit{a}) & racine carrée \\
%\hline sin(\textit{a}) & sinus of \textit{a} \\
\hline sin(\textit{a}) & sinus de \textit{a} \\
%\hline cos(\textit{a}) & cosinus of \textit{b} \\
\hline cos(\textit{a}) & cosinus de \textit{b} \\
%\hline tan(\textit{a}) & tangens of \textit{a} \\
\hline tan(\textit{a}) & tangeant de \textit{a} \\
%\hline asin(\textit{a}) & arcussinus of \textit{a} \\
\hline asin(\textit{a}) & arcsinus de \textit{a} \\
%\hline acos(\textit{a}) & arcuscosinus of \textit{a} \\
\hline acos(\textit{a}) & arccosinus de \textit{a} \\
%\hline atan(\textit{a}) & arcustangens of \textit{a} \\
\hline atan(\textit{a}) & arctangeant de \textit{a} \\
%\hline to int(\textit{a}) & convert string \textit{a} to integer \\
\hline to int(\textit{a}) & convertie une chaîne \textit{a} en entier \\
%\hline to real(\textit{a}) & convert string \textit{a} to real \\
\hline to real(\textit{a}) & convertie une chaîne \textit{a} en réel \\
%\hline to string(\textit{a}) & convert number \textit{a} to string \\
\hline to string(\textit{a}) & convertie un nombre \textit{a} en chaîne \\
%\hline lower(\textit{a}) & convert string \textit{a} to lower case \\
\hline lower(\textit{a}) & convertie une chaîne \textit{a} en minuscule \\
%\hline upper(\textit{a}) & convert string \textit{a} to upper case \\
\hline upper(\textit{a}) & convertie une chaîne \textit{a} en majuscule \\
%\hline length(\textit{a}) & length of string \textit{a} \\
\hline length(\textit{a}) & longueur d'une chaîne \textit{a} \\
%\hline atan2(y,x) & arcustangens of y/x using the signs of the two arguments 
%to determine the quadrant of the result. \\
\hline atan2(y,x) & arctangeant de y/x en utilisant les signes des deux arguments 
pour déterminer le quadrant du résultat. \\
%\hline replace(\textit{a}, replacethis, withthat) & replace \textit{replacethis} 
%with \textit{withthat} in string \textit{a} \\
\hline replace(\textit{a}, cela, parcela) & remplace \textit{cela} par 
\textit{parcela} dans la chaîne \textit{a} \\
%\hline substr(\textit{a},from,len) & len characters of string \textit{a} 
%starting from from (first character index is 1) \\
\hline substr(\textit{a},de,longueur) & longueur caractères de la chaîne \textit{a} 
en commençant par 'de' (le premier caractère a pour index 1) \\
%\hline \textit{a} || \textit{b} & concatenate strings \textit{a} and \textit{b} \\
\hline \textit{a} || \textit{b} & concatène les chaînes \textit{a} et \textit{b} \\
%\hline \$rownum & number current row \\
\hline \$rownum & numéro de ligne \\
%\hline \$area & area of polygon \\
\hline \$area & surface du polygone \\
%\hline \$perimeter & perimeter of polygon \\
\hline \$perimeter & périmètre du polygone \\
%\hline \$length & area of line \\
\hline \$length & longueur de la ligne \\
%\hline \$id & feature id \\
\hline \$id & id de l'entité \\
%\hline \$x & x coordinate of point \\
\hline \$x & coordonnée x du point \\
%\hline \$y & y coordinate of point \\
\hline \$y & coordonnée y du point \\
%\hline \textit{a} $\wedge$ \textit{b} & \textit{a} raised to the power of \textit{b} \\
\hline \textit{a} $\wedge$ \textit{b} & \textit{a} à la puissance de \textit{b} \\
%\hline \textit{a} * \textit{b} & \textit{a} multiplied by \textit{b} \\
\hline \textit{a} * \textit{b} & \textit{a} multiplié par \textit{b} \\
%\hline \textit{a} / \textit{b} & \textit{a} divided by \textit{b} \\
\hline \textit{a} / \textit{b} & \textit{a} divisé par \textit{b} \\
%\hline \textit{a} + \textit{b} & \textit{a} plus \textit{b} \\
\hline \textit{a} + \textit{b} & \textit{a} plus \textit{b} \\
%\hline \textit{a} - \textit{b} & \textit{a} minus \textit{b} \\
\hline \textit{a} - \textit{b} & \textit{a} moins \textit{b} \\
%\hline + \textit{a} & positive sign \\
\hline + \textit{a} & signe plus \\
%\hline - \textit{a} & negative value of \textit{a} \\
\hline - \textit{a} & valeur négative de \textit{a} \\
\hline 
%\caption{List of operators for the field calculator}\\
\caption{Liste des opérateurs pour le calculateur de champ}\\
\end{longtable}}
\end{center}

\FloatBarrier
