% vim: set textwidth=78 autoindent:

\subsection{GPS Plugin}\label{label_plugingps}

% when the revision of a section has been finalized, 
% comment out the following line:
\updatedisclaimer

\subsubsection{What is GPS?}\label{whatsgps}

GPS, the Global Positioning System, is a satellite-based system that allows anyone with a GPS receiver to find their exact position anywhere in the world.
It is used as an aid in navigation, for example in airplanes, in boats and by hikers.
The GPS receiver uses the signals from the satellites to calculate its latitude, longitude and (sometimes) elevation.
Most receivers also have the capability to store locations (known as \emph{waypoints}), sequences of locations that make up a planned \emph{route} and a tracklog or \emph{track} of the receivers movement over time.
Waypoints, routes and tracks are the three basic feature types in GPS data.
QGIS displays waypoints in point layers while routes and tracks are displayed in linestring layers.

\subsubsection{Loading GPS data from a file}\label{label_loadgps}

There are dozens of different file formats for storing GPS data.
The format that QGIS uses is called GPX (GPS eXchange format), which is a standard interchange format that can contain any number of waypoints, routes and tracks in the same file.

To load a GPX file you first need to load the plugin.
\mainmenuopt{Plugins} > \dropmenuopttwo{mActionShowPluginManager}{Plugin Manager...} > \checkbox{GPS Tools}

When this plugin is loaded a button with a small handheld GPS device will show up in the toolbar.
Clicking on this button \toolbtntwo{gps_importer}{GPS Tools} will open the GPS Tools dialog (see figure \ref{gpxloader}).

%\begin{figure}[ht]
%   \begin{center}
%\caption{\label{gpxloader}The \emph{GPS Tools} dialog window}
%\includegraphics[clip=true, width=12cm]{loadgpx}
%\end{center}
%\end{figure}

Use the browse button \browsebutton to select the GPX file, then use the checkboxes to select the feature types you want to load from that GPX file.
Each feature type will be loaded in a separate layer when you click \button{OK}.

\subsubsection{GPSBabel}

Since QGIS uses GPX files you need a way to convert other GPS file formats to GPX.
This can be done for many formats using the free program GPSBabel, which is available at \url{http://www.gpsbabel.org}.
This program can also transfer GPS data between your computer and a GPS device.
QGIS uses GPSBabel to do these things, so it is recommended that you install it.
However, if you just want to load GPS data from GPX files you will not need it.
Version 1.2.3 of GPSBabel is known to work with QGIS, but you should be able to use later versions without
any problems.


\subsubsection{Importing GPS data}

To import GPS data from a file that is not a GPX file, you use the tool \tab{Import other file} in the GPS Tools dialog.
Here you select the file that you want to import, which feature type you want to import from it, where you want to store the converted GPX file and what the name of the new layer should be.

When you select the file to import you must also select the format of that file by using the menu in the file selection dialog (see figure \ref{figure importdialog}).
All formats do not support all three feature types, so for many formats you will only be able to choose between one or two types.

%\begin{figure}[ht]
%   \begin{center}
%\caption{\label{figure importdialog}File selection dialog for the import
%tool}
%\includegraphics[clip=true, width=12cm]{importdialog}
%   \end{center}
%\end{figure}

\subsubsection{Downloading GPS data from a device}

QGIS can use GPSBabel to download data from a GPS device directly into vector layers.
For this you use the tool \tab{Download from GPS} (see Figure \ref{figure_download}), where you select your type
of GPS device, the port that it is connected to, the feature type that you want to download, the GPX file where the data should be stored, and the name of the new layer.

%\begin{figure}[ht]
%   \begin{center}
%\caption{\label{figure_download}The download tool}
%\includegraphics[clip=true, width=12cm]{download}
%   \end{center}
%\end{figure}


The device type you select in the GPS device menu determines how GPSBabel tries to communicate with the device.
If none of the types works with your GPS device you can create a new type (see section \ref{sec:Defining-new-device}).

The port is a file name or some other name that your operating system uses as a reference to the physical port in your computer that the GPS device is connected to.
\nix On Linux this is something like /dev/ttyS0 or /dev/ttyS1 and on \win Windows it's COM1 or COM2.

When you click \button{OK} the data will be downloaded from the device and appear as a layer in QGIS.

\subsubsection{Uploading GPS data to a device}

You can also upload data directly from a vector layer in QGIS to a GPS device, using the tool \tab{Upload to GPS}.
The layer must be a GPX layer.
To do this you simply select the layer that you want to upload, the type of your GPS device and the port that it's connected to.
Just as with the download tool you can specify new device types if your device isn't in the list.

This tool is very useful together with the vector editing capabilities of QGIS.
You can load a map, create some waypoints and routes and then upload them and use them in your GPS device.

\subsubsection{\label{sec:Defining-new-device}Defining new device types}

There are lots of different types of GPS devices.
The QGIS developers can't test all of them, so if you have one that does not work with any of the device types listed in the \tab{Download from GPS} and \tab{Upload to GPS} tools you can define your own device type for it.
You do this by using the GPS device editor, which you start by clicking the \button{Edit devices} button in the download or the upload window.

To define a new device you simply click the \button{New device} button, enter a name, a download command and an upload command for your device, and click the \button{Update device} button.
The name will be listed in the device menus in the upload and download windows, and can be any string.
The download command is the command that is used to download data from the device to a GPX file.
This will probably be a GPSBabel command, but you can use any other command line program that can create a GPX file.
QGIS will replace the keywords \usertext{\%type}, \usertext{\%in}, and \usertext{\%out} when it runs the command.

\usertext{\%type} will be replaced by {}``\usertext{-w}'' if you are downloading waypoints, {}``\usertext{-r}'' if you are downloading routes and {}``\usertext{-t}'' if you are downloading tracks.
These are command line options that tell GPSBabel which feature type to download.

\usertext{\%in} will be replaced by the port name that you choose in the download window and \usertext{\%out} will be replaced by the name you choose for the GPX file that the downloaded data should be stored in.
So if you create a device type with the download command {}``\usertext{gpsbabel \%type -i garmin -o gpx \%in \%out}'' (this is actually the download command for the predefined device type \selectstring{GPS device:}{Garmin serial})and then use it to download waypoints from port {}``\usertext{/dev/ttyS0}'' to the file {}``\usertext{output.gpx}'', QGIS will replace the keywords and run the command {}``\usertext{gpsbabel -w -i garmin -o gpx /dev/ttyS0 output.gpx}''.

The upload command is the command that is used to upload data to the device.
The same keywords are used, but \usertext{\%in} is now replaced by the name of the GPX file for the layer that is being uploaded, and \usertext{\%out} is replaced by the port name.

You can learn more about GPSBabel and it's available command line options at \url{http://www.gpsbabel.org}

Once you have created a new device type it will appear in the device lists for the download and upload tools.
