% vim:autoindent:set textwidth=78:

\section{Print Composer}\label{label_printcomposer}

% when the revision of a section has been finalized, 
% comment out the following line:
% \updatedisclaimer

The print composer provides growing layout and printing
capabilities. It allows you to add elements such as the QGIS map canvas, 
legend, scalebar, images, and text labels. You can size, group 
align and position each element and adjust the properties to create your
layout. The layout can be printed (also to Postscript and PDF), exported to
image formats or to SVG \footnote{Export to SVG supported, but it
is not working properly with some recent QT4 versions. You should try and
check individual on your system} and you can save the layout as template and
load it again in another session. See a list of tools in
table~\ref{tab:printcomposer_tools}:

\begin{table}[h]\index{Print composer!tools}
\centering
\caption{Print Composer Tools}\label{tab:printcomposer_tools}\medskip
 \begin{tabular}{|l|p{6.9cm}|l|p{6.9cm}|}
 \hline \textbf{Icon} & \textbf{Purpose} & \textbf{Icon} &
 \textbf{Purpose} \\
 \hline \includegraphics[width=0.7cm]{mActionFolder}
 & Load from template &
 \includegraphics[width=0.7cm]{mActionFileSaveAs} & Save as template \\
 \hline \includegraphics[width=0.7cm]{mActionExportMapServer}
 & Export to an image format & 
 \includegraphics[width=0.7cm]{mActionSaveAsSVG} & Export print composition 
 to SVG \\
 \hline \includegraphics[width=0.7cm]{mActionFilePrint} & Print or 
 export as PDF or Postscript &
 \includegraphics[width=0.7cm]{mActionZoomFullExtent} & Zoom to
 full extend \\
 \hline \includegraphics[width=0.7cm]{mActionZoomIn} & Zoom in &
 \includegraphics[width=0.7cm]{mActionZoomOut} & Zoom out \\
 \hline \includegraphics[width=0.7cm]{mActionDraw} & Refresh 
 view &
 \includegraphics[width=0.7cm]{mActionAddRasterLayer} & Add 
 new map from QGIS map canvas \\
 \hline \includegraphics[width=0.7cm]{mActionSaveMapAsImage} & Add Image to 
 print composition &
 \includegraphics[width=0.7cm]{mActionLabel} & Add label to print composition \\
 \hline \includegraphics[width=0.7cm]{mActionAddLegend} & Add new legend to 
 print composition & 
 \includegraphics[width=0.7cm]{mActionScaleBar} & Add new scalebar to print
 composition\\
 \hline \includegraphics[width=0.7cm]{mActionSelectPan} & Select/Move item in 
 print composition &
 \includegraphics[width=0.7cm]{mActionMoveItemContent} & Move content within
 an item \\
 \hline \includegraphics[width=0.7cm]{mActionGroupItems} & Group items of 
 print composition & 
 \includegraphics[width=0.7cm]{mActionUngroupItems} & Ungroup items of print 
 composition \\
 \hline \includegraphics[width=0.7cm]{mActionRaiseItems} & Raise selected
 items  &
 \includegraphics[width=0.7cm]{mActionLowerItems} & Lower selected items \\
 \hline \includegraphics[width=0.7cm]{mActionMoveItemsToTop} & Move selected
 items to top & 
 \includegraphics[width=0.7cm]{mActionMoveItemsToBottom} & Move selected
 items to bottom \\
 \hline \includegraphics[width=0.7cm]{mActionAlignLeft} & Align selected 
 items left &
 \includegraphics[width=0.7cm]{mActionAlignRight} & Align selected items 
 right \\
 \hline \includegraphics[width=0.7cm]{mActionAlignHCenter} & Align selected 
 items center &
 \includegraphics[width=0.7cm]{mActionAlignVCenter} & Align selected items
 center vertical \\
 \hline \includegraphics[width=0.7cm]{mActionAlignTop} & Align selected
 items top &
 \includegraphics[width=0.7cm]{mActionAlignBottom} & Align selected
 items bottom \\
\hline
\end{tabular}
\end{table}

To access the print composer, click on the \toolbtntwo{mActionFilePrint}{Print}
button in the toolbar or choose \mainmenuopt{File} > \dropmenuopttwo{mActionFilePrint}{Print Composer}.

\subsection{Using Print Composer}\label{label_useprintcomposer} 

Before you start to work with the print composer, you need to load some 
raster and vector layers in the QGIS map canvas and adapt their properties 
to suite your own convinience. After everything is rendered and symbolized to 
your liking you click the \toolbtntwo{mActionFilePrint}{Print Composer} icon.

\begin{figure}[ht]
   \begin{center}
   \caption{Print Composer \nixcaption}\label{fig:print_composer_blank}\smallskip
   \includegraphics[clip=true, width=\textwidth]{print_composer_blank}
\end{center}  
\end{figure}

Opening the print composer provides you with a blank canvas to which you can 
add the current QGIS map canvas, legend, scalebar, images and text. Figure
\ref{fig:print_composer_blank} shows the initial view of the print composer 
with an activated \checkbox{Snap to grid} modus but before any elements are
added. The print composer provides two tabs:

\begin{itemize}
\item The \tab{General} tab allows you to set paper size, orientation, the
print quality for the output file in dpi and to activate snapping to a grid
of a defined resolution. Please note, the \checkbox{Snap to grid} feature
only works, if you define a grid resolution > 0. Furthermore you can also
activate the \checkbox{Print as raster} checkbox. This means all elements
will be rastered before printing or saving as Postscript of PDF.
\item The \tab{Item} tab displays the properties for the selected map element. 
Click the \toolbtntwo{mActionSelectPan}{Select/Move item} 
icon to select an element (e.g. legend, scalebar or label) on the canvas. 
Then click the Item tab and customize the settings for the selected 
element.
\end{itemize}

You can add multiple elements to the composer. It is also possible to have 
more than one map view or legend or scalebar in the print composer canvas. 
Each element has its own properties and in the case of the map, its own 
extent.

\subsubsection{Adding a current QGIS map canvas to the Print Composer}

To add the QGIS map canvas, click on the \toolbtntwo{mActionAddRasterLayer}{Add new map 
from QGIS map canvas} button in the print composer toolbar and drag a 
rectangle on the composer canvas with the left mouse button to add the map. 
You will see an empty box with a \textit{"Map will be printed here"} message.
To display the current map, you can choose between three different modes in
the map \tab{Item} tab:

\begin{itemize}
\item \selectstring{Preview}{Rectangle} is the default setting. It only
displays an empty box with a message \textit{"Map will be printed here"}. 
\item \selectstring{Preview}{Cache} renders the map in the current screen
resolution. If case you zoom in or out the composer window, the map is not
rendered again but the image will be scaled.
\item \selectstring{Preview}{Render} means, that if you zoom in or out the
composer window, the map will be rendered again, but for space reasons, only
up to a maximum resolution.
\end{itemize}

\begin{figure}[ht]
\centering
\caption{Print Composer map item tab content \nixcaption}\label{fig:print_composer_map_item}
   \subfigure[Width, height and extend dialog] {\label{subfig:print_composer_map_item1}\includegraphics[clip=true, width=0.4\textwidth]{print_composer_map_item1}}\goodgap
   \subfigure[Properties dialog] {\label{subfig:print_composer_map_item2}\includegraphics[clip=true, width=0.4\textwidth]{print_composer_map_item2}}
\end{figure}

You can resize the map later by clicking on the \toolbtntwo{mActionSelectPan}{Select/Move item} 
button, selecting the element, and dragging one of the blue handles in the corner of the map. With the 
map selected, you can now adapt more properties in the map \tab{Item} tab. Resize the map 
item specifying the width and height or the scale. Define the map extend using Y and 
X min/max values or clicking the \button{set to map canvas extend} button. Update the 
map preview and select, whether to see a preview from cache or an empty rectangle with 
a \textit{"Map will be printed here"} message. Define colors and outline width for the 
element frame, set a background color and opacity for the map canvas. And you can also 
select or unselect to display an element frame with the \checkbox{frame} checkbox 
(see Figure~\ref{fig:print_composer_map_item}). If you change the view on the QGIS 
map canvas by zooming or panning or changing vector or raster properties, you can 
update the print composer view selecting the map element in the print composer and clicking 
the \button{Update Preview} button in the map \tab{Item} tab 
(see Figure~\ref{fig:print_composer_map_item}). 

To move layers within the map element select the map element, click 
the \toolbtntwo{mActionMoveItemContent}{Move item content} icon 
and move the layers within the map element frame with the left mouse button.

\subsubsection{Navigation tools}

For map navigation the print composer provides 4 general tools:

\begin{itemize}
\item \toolbtntwo{mActionZoomOut}{Zoom in},
\item \toolbtntwo{mActionZoomOut}{Zoom out},
\item \toolbtntwo{mActionZoomFullExtent}{Zoom to full extend} and
\item \toolbtntwo{mActionDraw}{Refresh the view}, if you find the view in an
inconsistent state.
\end{itemize}


\subsubsection{Adding other elements to the Print Composer} 

Besides adding a current QGIS map canvas to the Print Composer, it is also possible 
to add, position, move and customize legend, scalebar, images and label elements.

\minisec{Label and images}

To add a label or an image, click the \toolbtntwo{mActionLabel}{Add label} or 
\toolbtntwo{mActionSaveMapAsImage}{Add image} icon, place the element with
the left mouse button on the print composer canvas and position and customize
their appearance in the \tab{Item} tab. 

\begin{figure}[ht]
\centering
\caption{Customize print composer label and images \nixcaption}\label{fig:print_composer_tab2}
   \subfigure[label item tab] {\label{subfig:print_composer_label_item}\includegraphics[clip=true, width=0.4\textwidth]{print_composer_label_item}}\goodgap
   \subfigure[image item tab] {\label{subfig:print_composer_image_item}\includegraphics[clip=true, width=0.4\textwidth]{print_composer_image_item}}
\end{figure}

\minisec{Legend and scalebar}

To add a map legend or a scalebar, click the \toolbtntwo{mActionAddLegend}{Add new legend} or 
\toolbtntwo{mActionScaleBar}{Add new scalebar} icon, place the element with the left 
mouse button on the print composer canvas and position and customize their appearance in the \tab{Item} tab.

\begin{figure}[ht]
\centering
\caption{Customize print composer legend and scalebar \nixcaption}\label{fig:print_composer_tab1}
   \subfigure[legend item tab] {\label{subfig:print_composer_legend_item}\includegraphics[clip=true, width=0.4\textwidth]{print_composer_legend_item}}\goodgap
   \subfigure[scalebar item tab] {\label{subfig:print_composer_scalebar_item}\includegraphics[clip=true, width=0.4\textwidth]{print_composer_scalebar_item}}
\end{figure}

\subsubsection{Raise, lower and align elements}

Raise or lower functionalities for elements are inside the
\toolbtntwo{mActionRaiseItems}{Raise selected items} pulldown menu. Choose an
element on the print composer canvas and select the matching functionality to
raise or lower the selected element compared to the other elements (see
table~\ref{tab:printcomposer_tools}). 

There are several alignment functionalities available within the
\toolbtntwo{mActionAlignLeft}{Align selected items} pulldown menu (see
table~\ref{tab:printcomposer_tools}). To use an alignment functionality , you
first select some elements and then click on the matching alignment icon. All
selected will then be aligned within to their common bounding box.       

\subsubsection{Creating Output}

Figure \ref{fig:print_composer_complete} shows the print composer with an example 
print layout including each type of map element described in the sections above.

\begin{figure}[h]
   \begin{center}
   \caption{Print Composer with map view, legend, scalebar, and text added \nixcaption}
   \label{fig:print_composer_complete}\smallskip
   \includegraphics[clip=true, width=\textwidth]{print_composer_complete}
\end{center}  
\end{figure}

The print composer allows you to create several output formats and it is possible to 
define the resolution (print quality) and paper size:

\begin{itemize}
\item The \toolbtntwo{mActionFilePrint}{Print} icon allows to print the layout 
to a connected printer or as PDF or Postscript file depending on installed printer 
drivers.
\item The \toolbtntwo{mActionExportMapServer}{Export as image} icon exports the 
composer canvas in several image formats such as PNG, BPM, TIF, JPG, \dots
\item The \toolbtntwo{mActionSaveAsSVG}{Export as SVG} icon saves the print 
composer canvas as a SVG (Scalable Vector Graphic). \textbf{Note:} Currently the 
SVG output is very basic. This is not a QGIS problem, but a problem of the underlaying 
Qt library. This will hopefully be sorted out in future versions.
\end{itemize}

\subsubsection{Saving and loading a print composer layout}

With the \toolbtntwo{mActionFileSaveAs}{Save as template} and
\toolbtntwo{mActionFolder}{Load from template} icons you can save the current
state of a print composer session as a  *.qpt template and load the template
again in another session.

