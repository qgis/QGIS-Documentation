% vim:autoindent:set textwidth=78:

\section{Print Composer}\label{label_printcomposer}

% when the revision of a section has been finalized, 
% comment out the following line:
\updatedisclaimer

The print composer provides growing layout and printing
capabilities. It allows you to add elements such as the QGIS map canvas, 
legend, scalebar, images, and text labels. You can size, group 
and position each element and adjust the properties to create your layout. 
The result can be printed (also to Postscript and PDF), exported as an image, 
or exported to SVG.\footnote{Export to SVG is currently not supported, 
because this functionality is not working with recent QT4 versions.} See a
list of all buttons in table~\ref{tab:printcomposer_tools} below:

\begin{table}[h]\index{Print composer!tools}
\centering
\caption{Print Composer Tools}\label{tab:printcomposer_tools}\medskip
 \begin{tabular}{|l|p{6.9cm}|l|p{6.9cm}|}
 \hline \textbf{Icon} & \textbf{Purpose} & \textbf{Icon} &
 \textbf{Purpose} \\

 \hline \includegraphics[width=0.7cm]{mActionExportMapServer}
 & Export to an image format & 
 \includegraphics[width=0.7cm]{mActionSaveAsSVG} & Export print composition 
 to SVG \\
 \hline \includegraphics[width=0.7cm]{mActionFilePrint} & Print or 
 export as PDF or Postscript &
 \includegraphics[width=0.7cm]{mActionZoomFullExtent} & Zoom to
 full extend \\
 \hline \includegraphics[width=0.7cm]{mActionZoomIn} & Zoom in &
 \includegraphics[width=0.7cm]{mActionZoomOut} & Zoom out \\
 \hline \includegraphics[width=0.7cm]{mActionDraw} & Refresh 
 view &
 \includegraphics[width=0.7cm]{mActionAddRasterLayer} & Add 
 new map from QGIS map canvas \\
 \hline \includegraphics[width=0.7cm]{mActionSaveMapAsImage} & Add Image to 
 print composition &
 \includegraphics[width=0.7cm]{mActionLabel} & Add label to print composition \\
 \hline \includegraphics[width=0.7cm]{mActionAddLegend} & Add new legend to 
 print composition & 
 \includegraphics[width=0.7cm]{mActionScaleBar} & Add new scalebar to print
 composition\\
 \hline \includegraphics[width=0.7cm]{mActionSelectPan} & Select/Move item in 
 print composition &
 \includegraphics[width=0.7cm]{mActionMoveItemContent} & Move content within
 an item \\
 \hline \includegraphics[width=0.7cm]{mActionGroupItems} & Group items of 
 print composition & 
 \includegraphics[width=0.7cm]{mActionUngroupItems} & Ungroup items of print 
 composition \\
 \hline \includegraphics[width=0.7cm]{mActionRaiseItems} & Raise selected
 items in print composition &
 \includegraphics[width=0.7cm]{mActionLowerItems} & Lower selected items 
 in print composition \\
 \hline \includegraphics[width=0.7cm]{mActionMoveItemsToTop} & Move selected
 items to top & 
 \includegraphics[width=0.7cm]{mActionMoveItemsToBottom} & Move selected
 items to bottom \\
\hline
\end{tabular}
\end{table}

To access the print composer, click on the \toolbtntwo{mActionFilePrint}{Print}
button in the toolbar or choose \mainmenuopt{File} > \dropmenuopttwo{mActionFilePrint}{Print}.

\subsection{Using Print Composer}\label{label_useprintcomposer} 

To use the print composer, first add the layers you
want to print to QGIS. The layers should be rendered and symbolized to your
liking prior to composing the map (see example in Figure \ref{fig:print_composer_complete}). 

\begin{figure}[ht]
   \begin{center}
   \caption{Print Composer}\label{fig:print_composer_blank}\smallskip
   \includegraphics[clip=true, width=\textwidth]{print_composer_blank}
\end{center}  
\end{figure}

Opening the print composer provides you with a blank canvas to which you can add
the current map view, legend, scalebar, and text. Figure
\ref{fig:print_composer_blank} shows the initial view of the print composer before
any elements are added.

The print composer has two tabs: \tab{General} and \tab{Item}. The General tab
allows you to set the paper size, orientation, and resolution for the map.
The Item tab displays the properties for the currently selected map element.
By selecting an element on the map (eg. legend, scalebar, text, etc.) and
clicking on the \tab{Item} tab, you can customize the settings.

You can add multiple elements to the composer. This allows you to have more
than one map view and legend in the composer. Each element has its own
properties and in the case of the map, its own extent.

\subsubsection{Adding a map layout to the Print Composer}

To add the QGIS map canvas to the print composer, click on the
%FIXME \toolbtntwo{composer_add_image}{Add a new map} 
button in toolbar. Drag a rectangle on the composer canvas to add the
map. You can resize the map later by clicking on the \button{Select/move item}
button, clicking on the map, and dragging one of the handles in the corner of
the map. With the map selected, you can also resize the map by specifying the
width and height on the Item properties tab.

The map is linked to the QGIS map canvas. If you change the view on the map
canvas by zooming or panning, you can update the print composer view by
selecting the map in the print composer and clicking on the \button{Set Extent} 
button. You can also change the print composer view by specifying a map scale. 
To set the view to a specific scale:

\begin{enumerate}
\item Choose \selectstring{Set}{Scale (calculate extent)}
\item Enter the scale denominator in the scale box
\item Press Enter
\end{enumerate} 

\subsubsection{Adding other Elements to the Print Composer} 
 
Already existing QGIS templates can be used to easily load and adapt print
layouts. To open an existing template, click on the
%FIXME \toolbtntwo{composer_open_template}{Open Template} 
button. Choose a template and
customize its appearance. 

To add a logo, north arrow or any  kind of image to the composer, click on
the 
%FIXME \toolbtntwo{composer_add_image}{Add Image} 
button. The image will 
be placed on the composer canvas and you can move it where you like. 

A legend can be added to the composer canvas and customized to show only the
desired layers. To add a legend, click on the
%FIXME \toolbtntwo{composer_add_legend08}{Add Vector Legend} 
button. The legend will be
placed on the composer canvas and you can move it where you like. Click on
the \tab{Items} tab to customize the appearance of the legend, including
which layers are shown.

To add a scalebar to the composer, click on the
%FIXME \toolbtntwo{composer_add_scalebar08}{Add Scalebar} 
button. Use the \tab{Item}
tab to customize the segment size, number of segments, scalebar units, size,
and font for the scalebar.

You can add text labels to the composer by clicking on the
%FIXME \toolbtntwo{composer_add_label}{Add New Label} 
button. Use the \tab{Item} tab
while the text is selected to customize the settings or change the default text.

Figure \ref{fig:print_composer_complete} shows the print composer after adding
each type of map element.
\begin{figure}[h]
   \begin{center}
   \caption{Print Composer with map view, legend, scalebar, and text added}
   \label{fig:print_composer_complete}\smallskip
   \includegraphics[clip=true, width=\textwidth]{print_composer_complete}
\end{center}  
\end{figure}

\subsubsection{Other Features}

The print composer has navigation tools to zoom in and out. To zoom in, click
the \button{Zoom in} tool. The print composer canvas will be scaled by a factor to 2. Use
the scrollbars to adjust the view to the area of interest. Zooming out works
in a similar fashion.

If you find the view in an inconsistent state, you can use the \button{Refresh} button
to redraw the print composer canvas.

\subsubsection{Creating Output}

The print composer allows you to print the layout to a printer, export to a PNG or
export to SVG. Each of these functions is available from the composer toolbar.

To save the composer canvas as a templates, click on the
%\toolbtntwo{composer_save_template}{Save Template As} 
button. Browse to the directory 
you like and save a template to use it again for another map canvas.

It is possible to export the result as an image by clicking on the
%FIXME \toolbtntwo{composer_export_image}{Export as image} 
button. 

To export the composer canvas as an  SVG (Scalable Vector Graphic), click on
the 
%FIXME \toolbtntwo{composer_export_svg}{Export as SVG} 
button. \textbf{Note:}
Currently the SVG output is very basic. This is not a QGIS problem, but a
problem of the underlaying Qt library. This will be sorted out in future versions.
 
