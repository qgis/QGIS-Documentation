%  !TeX  root  =  user_guide.tex 

\chapter{Using external QGIS Python Plugins}\label{sec:external_plugins}\index{plugins}

% when the revision of a section has been finalized, 
% comment out the following line:
% \updatedisclaimer

External QGIS plugins are written in Python. They are stored in either 
the 'Official' or 'User contributed' QGIS Repostories, or in various other external 
repositories maintained by individual authors. 
Table \ref{tab:external_plugins} shows a list of currently available 'Official' 
plugins, with a short description.
Detailed documentation about the usage, minimum QGIS version, homepage, authors, 
and other important information are provided with the external plugins themselves 
and is not included in this manual.
\footnote{Updates of core plugins may be 
available in this repository as external overlays.} 
\footnote{fTools, Mapserver Export, and the Plugin Installer are Python plugins, 
but they are also part of the QGIS sources, and are automatically loaded and 
enabled inside the QGIS Plugin Manager (see Section~\ref{sec:load_external_plugin}).}

You will find an up-to-date list of 'Official' plugins in the Official QGIS 
Repository at \url{http://qgis.osgeo.org/download/plugins.html}. This list is 
also available automatically from the \filename{Plugins installer} 
via \dropmenuopttwo{plugin_installer}{Fetch Python Plugins...}.

\begin{table}[H]
\centering
 \begin{tabular}{|l|l|p{8cm}|}
\hline \textbf{Icône} & \textbf{Extension externe} & \textbf{Description}\\
\hline
\includegraphics[width=0.7cm]{zoom2point_icon}
 & Zoom To Point \index{plugins!Zoom To Point} & Zooms to a coordinate 
  specified in the input dialog. You can specify the zoom level as well to 
  control the view extent.\\
\hline
\includegraphics[width=0.7cm]{plugin_installer}
 & Plugin Installer \index{plugins!Plugin Installer} & The most recent Python Plugin Installer.\\
\hline
\end{tabular}
\caption{Current moderated external QGIS Plugins}\label{tab:external_plugins}
\end{table}

A detailed description of the installation procedure for external python 
plugins can be found in Section \ref{sec:load_external_plugin}.

\begin{Tip} \caption{\textsc{Add more repositories}}
To add the 'User contributed' repository and/or several external author repositories, open the 
Plugin Installer (\mainmenuopt{Plugins} > \dropmenuopttwo{plugin_installer}{Fetch Python Plugins...}),
go to the \tab{Repositories} tab, and click \button{Add 3rd party repositories}. 
If you do not want one or more of the added repositories, they can be disabled via the 
\button{Edit...} button, or completely removed with the \button{Delete} button.
\end{Tip}
\FloatBarrier
