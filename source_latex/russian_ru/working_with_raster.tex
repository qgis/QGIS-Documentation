\chapter{Работа с растровыми данными}\label{label_raster}
\index{растровые слои|(}

% when the revision of a section has been finalized,
% comment out the following line:
% \updatedisclaimer

Из этой главы вы узнаете, как как вывести растровый слой с различными параметрами.

Реализация работы с растрами в \qg основана на библиотеке GDAL
\footnote{работа с растровыми данными GRASS реализована через отдельный
провайдер}, что позволяет открывать данные в форматах Arc/Info Binary Grid\index{Arc/Info Binary Grid},
Arc/Info ASCII Grid\index{Arc/Info ASCII Grid},GeoTIFF\index{GeoTIFF},
Erdas Imagine\index{Erdas Img.} и многих других.

На момент написания руководства, библиотека GDAL подерживала свыше 100
различных форматов растровых данных \cite{GDALweb}. С полным списком можно
ознакомиться по адресу \\
\url{http://www.gdal.org/formats_list.html}.

\textbf{Примечание}: по различным причинам, не все из поддерживаемых
форматов могут работать в QGIS. Например, некоторые требуют наличия
внешних коммерческих библиотек или библиотеки GDAL/OGR в вашей операционной
системе (ОС) скомпилированы без поддержки формата, который вы хотите
использовать. При загрузке в QGIS данных векторных или растровых форматов
в списке типов файлов будут отображаться только те форматы, которые были
проверены. Остальные (непроверенные) форматы могут быть загружены, если
выбрать <<*.*>> в выпадающем списке <<Типы файлов>>.

Работа с растровыми данными GRASS описана в разделе~\ref{sec:grass}.

\section{Что такое растровые данные?}\label{label_whatsraster}
\index{растровые слои!определение}

Растровые данные в ГИС представляют из себя матрицы, каждая ячейка которых
передаёт значение некого параметра поверхности. Каждая ячейка в
растровой сетке имеет определенный размер. Как правило, ячейки имеют
прямоугольную форму (в QGIS они всегда прямоугольные). Типичный набор
растровых данных включает в себя данные дистанционного зондирования,
такие как аэрофотосъемка, спутниковые снимки или смоделированные данные,
например, матрицу высот.

В отличии от векторных данных, у растров, как правило, нет присоединенных к каждой
ячейке табличных данных. Они геокодируются размещением пикселей относительно
координат углового пикселя растрового слоя, что позволяет корректно размещать
такие данные на картах в \qg.

Для правильного отображения данных \qg использует информацию о привязке,
находящуюся внутри растрового слоя (например, GeoTiff) или в соответствующем файле
привязки.\index{растровые слои!привязка}

\section{Загрузка растровых данных в QGIS}\label{label_loadraster}

Растровые слои загружаются нажатием на кнопку
\toolbtntwo{mActionAddRasterLayer}{Загрузить растр} или выбором меню
\mainmenuopt{Слой} \arrow
\dropmenuopttwo{mActionAddRasterLayer}{Добавить растроый слой}.
Несколько слоёв можно загрузить, удерживая клавишу \keystroke{Control}
или \keystroke{Shift} в диалоге \dialog{Добавить растровый слой}.\index{растровые слои!загрузка}

Когда растровый слой появится в панели <<Слои>>, нажмите на нем правой
кнопкой мыши для вызова контекстного меню.

\minisec{Контекстное меню для растровых слоев}

\begin{itemize}[label=--]
\item \dropmenuopt{Увеличить до границ слоя}
\item \dropmenuopt{Увеличить до наилучшего масштаба (100\%)}
\item \dropmenuopt{Показать в обзоре}
\item \dropmenuopt{Удалить}
\item \dropmenuopt{Изменить систему координат}
\item \dropmenuopt{Выбрать систему координат слоя для проекта}
\item \dropmenuopt{Свойства}
\item \dropmenuopt{Переименовать}
\item \dropmenuopt{Добавить группу}
\item \dropmenuopt{Развернуть все}
\item \dropmenuopt{Свернуть все}
\end{itemize}

\section{Свойства растра}\label{label_rasterprop}

Чтобы открыть и установить свойства растрового слоя, необходимо два раза
кликнуть на нем мышкой в панели <<Слои>> или нажать на растре правой
кнопкой мыши и выбрать \dropmenuopt{Свойства} из контекстного
меню\index{растровые слои!контекстное меню}. На рисунке~\ref{fig:raster_properties}
показано диалоговое окно \dialog{Свойства слоя}. Оно состоит из
вкладок:

\begin{itemize}[label=--]
 \item \tab{Стиль}
 \item \tab{Прозрачность}
 \item \tab{Цветовая карта}
 \item \tab{Общие}
 \item \tab{Метаданные}
 \item \tab{Пирамиды}
 \item \tab{Гистограмма}
\end{itemize}

\begin{figure}[h]
  \centering
   \includegraphics[clip=true, width=14cm]{rasterPropertiesDialog}
   \caption{Свойства растрового слоя \wincaption}\label{fig:raster_properties}
\end{figure}

\subsection{Стиль}\label{label_symbology}

В QGIS есть два способа отображения растрового слоя:\index{растровые слои!каналы}

\begin{itemize}[label=--]
\item Одноканальное серое "--- изображение будет выведено в оттенках серого,
в псевдоцветном или кислотном режиме.
\item Трехканальное цветное "--- растр отображается в виде трех каналов:
красный, зелёный и синий, которые используются для создания цветного
изображения.
\end{itemize}

В обоих типах отображения можно инвертировать цвета, используя флажок
\checkbox{Обратить цветовую карту}.

\minisec{Одноканальное отображение}

Этот режим включает два основных параметра. Во-первых, вы можете выбрать канал,
который нужно отобразить (если данные состоят из нескольких каналов).

Во-вторых, вы можете выбрать для отображения одну из имеющихся цветовых схем.

В выпадающем списке выбора цветовой карты доступны следующие пункты:
\selectstring{{}Цветовая карта}{Градации серого} "--- по умолчанию выбраны
градации серого. Также доступны режимы:
\begin{itemize}[label=--]
\item Псевдоцвет
\item Кислотная
\item Цветовая карта
\end{itemize}

При выборе режима \selectstring{{}Цветовая карта}{Цветовая карта}, становится
доступной вкладка \tab{Цветовая карта}. Цветовая карта подробно
рассматривается в разделе~\ref{label_colormaptab}.

QGIS может скрывать пиксели, значения которых находятся вне заданного
интервала стандартного отклонения от среднего по слою.\index{растровые слои!стандартное отклонение}
Эта функция применяется, когда в растровом слое присутствуют несколько ячеек
с ошибочно завышенными значениями, которые негативно влияют на отображение растра.
Данный параметр доступен только для изображений, выводимых в псевдоцвете
или в кислотной палитре.

\minisec{Трёхканальное отображение}

Этот режим позволяет более гибко изменять внешний вид растрового слоя.
Например, можно изменить порядок цветовых каналов со стандартной RGB-схемы на
какой-нибудь другой.

Для цветовых каналов допускается масштабирование значений.

\begin{Tip}\caption{\textsc{Просмотр одного канала многоканального растра}}
Для того, чтобы отобразить только один канал (например, красный) в
многокальном изображении, можно задать каналы зелёного и синего в значение
<<Не задано>>, но это не совсем корректно. Для отображения красного
канала нужно задать тип отображения в <<Градации серого>>, а затем
выбрать красный как основной канал для серого.
\end{Tip}

\subsection{Прозрачность} \label{rastertab:transparency}

QGIS поддерживает отображение растровых слоёв с разной степенью
прозрачности.\index{растровые слои!прозрачность} Для этого используется
ползунок прозрачности, с помощью которого можно указать, до какой
степени слой может быть прозрачным, чтобы увидеть слои, находящиеся под
ним. Это очень удобно, когда загружено множество растровых слоев.
Например, когда загружен растр рельефа и основной растр, что делает
карту более <<рельефной>>.

Также, можно ввести величину растра, которая будет рассматриваться
как значение {\em <<НЕТ ДАННЫХ>>}. Сделать это можно как вручную, так и
воспользовавшись кнопкой
\toolbtntwo{mActionContextHelp}{Добавить значения с экрана}.

Более гибко степень прозрачности можно настроить в панели
\guiheading{Параметры прозрачности}, которая позволяет указать индивидуальную
прозрачность каждого пикселя.

Например, нужно установить прозрачность воды в растре
\filename{landcover.tif} 20\%. Для этого нужно:
\begin{enumerate}
 \item Загрузить растр \filename{landcover}
 \item Открыть \dialog{Свойства} растра двойным щелчком на имени растра в легенде
 или щёлкнув на нём правой кнопкой мыши и выбрать \dropmenuopt{Свойства}
 из контекстного меню.
 \item Перейти на вкладку \tab{Прозрачность}
 \item \label{enum:add} Нажать кнопку
 \toolbtntwo{mActionNewAttribute}{Добавить значения вручную}. Появится
 новая строка в перечне прозрачных пикселей
 \item \label{enum:transp} Ввести значение растра (например, 0) и
 установить значение прозрачности в 20\%
 \item Нажать \button{Применить} и посмотреть результат на карте
\end{enumerate}

Можно повторить шаги \ref{enum:add} и \ref{enum:transp} чтобы добавить
больше значений для задания прозрачности.

Очевидно, что настройка прозрачности растра "--- простая, но требующая времени
процедура. Для экономии времени в дальнейшем можно воспользоваться кнопкой
\toolbtntwo{mActionFileSave}{Экспорт в файл}, которая позволяет сохраненить заданные
параметры. Кнопка \toolbtntwo{mActionFolder}{Импорт из файла}
загружает сохранённые ранее параметры прозрачности и применяет их к
выбранному растровому слою.

\subsection{Цветовая карта} \label{label_colormaptab}

Вкладка \tab{Цветовая карта} доступна при выборе одноканального режима
отображения растра во вкладке \tab{Стиль} (см. главу~\ref{label_symbology}).

Доступны три вида интерполяции цветов:
\begin{itemize}[label=--]
\item Дискретная
\item Линейная
\item Точная
\end{itemize}

Кнопка \button{Добавить значение} добавляет цвет в пользовательскую таблицу
цветов. Кнопка \button{Удалить значение} удаляет цвет из пользовательской
таблицы цветов, а кнопка \button{Сортировать} сортирует таблицу цветов
по значениям из колонки <<Значение>>. Двойной щелчок мыши на поле <<Значение>>
позволяет задать конкрентую величину, соответствующую данному цвету. Двойной
щелчок мыши на поле <<Цвет>> открывает диалоговое окно \dialog{Выбор цвета},
в котором можно выбрать цвет для данной величины. Здесь же можно назначить
каждому цвету свою метку, но эти метки не будут отображаться при использовании
инструмента определения.

Кнопка \toolbtntwo{mActionNewAttribute}{Загрузить цветовую карту из канала},
позволяет загрузить таблицу цветов из канала (если она в нём присутствует).

Блок \guiheading{Создать новую цветовую карту} позволяет создавать
новые категории цветовой карты. Для этого задается нужное
\selectnumber{{}количество значений}{15} и нажимается кнопка
\button{Классифицировать}. В настоящее время поддерживается только
\selectstring{{}Режим классификации}{Равные интервалы}\index{растровые слои!классификация}.

\subsection{Общие}\label{label_generaltab}

Вкладка \tab{Общие} содержит основную информацию о выбранном растре,
в том числе источник слоя и его имя в легенде (которое можно изменить).
Также в этой вкладке отображается образец слоя, его легенда и палитра.
\index{растровые слои!свойства}

Кроме того, здесь можно установить видимость слоя в пределах
масштаба. Для этого нужно установить флажок в соответствующем поле и
задать масштаб, в пределах которого данный слой будет отображаться на
карте.

Здесь же показана система координат в виде строки PROJ.4. Для её изменения
нажмите кнопку \button{Выбрать\ldots}.

\subsection{Метаданные}\label{label_metatab}

Вкладка \tab{Метаданные} содержит полные данные о растровом слое,
включая статистику о каждом канале загруженного растра. Статистические
данные собираются по принципу <<нужно знать>>, так что, возможно,
статистика по слоям может быть не доступной.\index{растровые слои!метаданные}

В основном, эта вкладка используется для просмотра информации. В ней
нельзя изменить какие-либо значения. Для обновления статистики нужно
перейти на вкладку \tab{Гистограмма} и нажать кнопку \button{Обновить}.
Подробнее о гистограммах говорится в разделе~\ref{label_histogram}.

\subsection{Пирамиды}\label{raster_pyramids}

Большие растры высокого разрешения могут замедлить работу в \qg. Для повышения
скорости при работе с такими растрами предусмотрена функция создания
копий данных низкого разрешения (пирамид). При работе с пирамидами, \qg
автоматически выбираем оптимальное разрешение в зависимости от текущего масштаба.
\index{растровые слои!пирамиды}
\index{растровые слои!разрешение пирамид}

Для сохранения пирамид необходимы права на запись в каталог, в котором
хранятся оригинальные данные. \\
Для построения пирамид используются два метода интерполяции:
\begin{itemize}[label=--]
\item Среднее значение
\item Ближайший сосед
\end{itemize}

Если активен флажок
\checkbox{Создавать встроенные пирамиды, если возможно} \qg будет
пытаться создать внутренние пирамиды.

Обратите внимание, что операция построения встроенных пирамид может
изменить оригинальный файл данных и их невозможно будет удалить после
создания. Поэтому перед построением пирамид рекомендуется сделать резервную копию
растровых данных.

\subsection{Гистограмма}\label{label_histogram}

Вкладка \tab{Гистограмма} позволяет просмотреть распределение \index{растровые слои!гистограмма}
каналов или цветов в растре. Гистрограмма формируется автоматически при
открытии вкладки.
%You can choose which bands to display by selecting them in the list box
%at the bottom left of the tab.

%% FIXME not supported at the moment
%Two different chart types are allowed:
%
%\begin{itemize}[label=--]
%\item Столбчатая
%\item Линейная
%\end{itemize}
%
%Также можно задать количество столбцов диаграммы и режим отображения:
%\checkbox{Разрешить аппроксимацию} или \checkbox{Разрешить значения вне диапозона}.
%После просмотра гистограммы можно заметить, что поля статистики каналов
%на вкладке \tab{Метаданные} заполнены.\index{растровые слои!метаданные}

\begin{Tip}\caption{\textsc{Сбор статистики растра}}
Для сбора статистики по слою, выберите псевдоцветное преобразование и
нажмите \button{Применить}. Сбор статистики может занять продолжительное
время. Дождитесь, пока \qg обработает ваши данные!\index{растровые слои!статистика}
\end{Tip}

\section{Калькулятор растров}\label{sec:raster_calc}
\index{растр!калькулятор растров}
\index{калькулятор растров}

\dialog{Калькулятор растров} доступный из меню \mainmenuopt{Растр} позволяет
выполнять различные вычисления на основе значений пикселей. Результат
вычислений сохраняется как изображение в GDAL-совместимом формате.

\begin{figure}[ht]
  \centering
    \includegraphics[clip=true, width=11.5cm]{raster_calculator}
    \caption{Калькулятор растров \wincaption}\label{fig:raster_calculator}
\end{figure}

В списке \textbf{Каналы растров} перечисленны доступные растровые слои.
Добавить растр в выражение можно двойным щелчком по его имени в списке.
При построении выражения можно использовать кнопки операторов или просто
набирать его в соответствующем поле.

В группе \textbf{Результаты} расположены настройки результирующего слоя.
Здесь можно задать охват области вычислений по охвату исходного растра или
введя координаты X, Y и желаемое количество строк и столбцов, чтобы получить
необходимое разрешение итогового слоя. Если исходный слой имеет другое
разрешение, величины будут пересчитанны по алгоритму <<ближайший сосед>>.

В разделе \textbf{Операторы} перечисленны все доступные операторы. Добавить
оператор в поле выражения можно нажав соответсвующую кнопку. Доступны
математические операторы ( + , - , * \dots) и тригонометрические функции
( sin, cos, tan, \dots). Планируется добавление дополнительных операторов.

При установленном флажке \checkbox{Добавить результат в проект} итоговый слой
будет автоматически добавлен к списку слоёв карты.

\section{Анализ растровых данных}\label{sec:raster_analysis}
\index{Raster!raster analysis}
\index{Raster analysis}

Помимо Калькулятора растров в \qg \CURRENT присутствуют и другие инструменты
для работы с растрами, а именно модуль ядра GDALTools. Подробное описание
этого модуля можно найти в разделе~\ref{label_plugingdaltools}.

\FloatBarrier
