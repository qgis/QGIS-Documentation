% vim: set textwidth=78 autoindent:
% !TeX root = user_guide.tex

\section{Getrennter Text Plugin}
\label{label_dltext}
\index{Plugins!Layer aus Textdatei laden}    

% when the revision of a chapter has been finalized, 
% comment out the following line:
% \updatedisclaimer

Mit dem Getrennter Text Plugin k�nnen ASCII-Texttabellen, die u.a zwei
Spalten f�r X- und Y-Koordinaten enthalten, als ein Layer in QGIS geladen
werden.

\minisec{Anforderungen}

Um Datenspalten aus einer Textdatei in QGIS zu laden, muss diese Textdatei
bestimmte Eigenschaften aufweisen:

\begin{enumerate}
\item Eine Kopfzeile mit den Spaltennamen. Diese Kopfzeile muss die erste
Zeile der Datei sein
\item Die Textdatei muss mindestens eine Spalte mit X- und eine mit
Y-Koordinaten enthalten.
Die Bezeichnungen in der Kopfzeile f�r diese Spalten k�nnen beliebig sein.
\item Die X- und Y-Koordinaten m�ssen als Zahlen angegeben sein. Das
Koordinatensystem spielt keine Rolle.
\end{enumerate}

Als Beispiel f�r einen Textdatei importieren wir die Datei
\filename{elevp.csv} aus dem QGIS Beispieldatensatz \ref{label_sampledata}:

\begin{verbatim} 
X;Y;ELEV
-300120;7689960;13
-654360;7562040;52
1640;7512840;3
[...]
\end{verbatim}

Einige weitere Anmerkungen zu Textdateien:

\begin{enumerate}
\item  Die Beispieldatei verwendet \mbox{$;$} als Trennzeichen. Es k�nnen
auch andere Zeichen zum Trennen der Spalten verwendet werden.
\item Die erste Zeile ist die Kopfzeile. Sie enth�lt die Spaltennamen name,
latdec, longdec und cell
\item Anf�hrungszeichen ({\tt{}"{}}) d�rfen nicht als Trennzeichen benutzt
werden
\item Die X-Koordinaten sind in der Spalte {\em longdec} enthalten
\item Die Y-Koordinaten sind in der Spalte {\em latdec} enthalten
\end{enumerate}

\minisec{Das Plugin verwenden}

Um das Plugin zu verwenden, m�ssen Sie es zuerst laden, wie in
Kapitel~\ref{sec:managing_plugins} beschrieben. 

Klicken Sie danach auf das Icon \toolbtntwo{delimited_text}{Getrennter Text} in
der Werkzeugleiste, um den Dialog zu �ffnen (siehe
Abbildung~\ref{fig:delim_text_plugin_dialog}).

\begin{figure}[ht]
   \begin{center}
   \caption{Plugin Getrennter Text \nixcaption}\label{fig:delim_text_plugin_dialog}\smallskip
   \includegraphics[clip=true, width=11cm]{delimited_text_dialog}
   \end{center}  
\end{figure}

Zuerst m�ssen Sie ein Trennzeichen f�r die Textspalten angeben. In diesem
Beispiel ist es das Trennzeichen ein Semikolon (\mbox{$;$}). Danach w�hlen
Sie mit den Knopf \button{Suchen} z.B. die Datei
\filename{qgis\_sample\_data/csv/elevp.csv} aus dem QGIS Beispieldatensatz. 

Damit die Textdatei richtig analysiert werden kann, ist es wichtig, das
richtige Trennzeichen zu w�hlen. Im unteren Feld wird der Inhalt korrekt 
in die vorkommenden Spalten unterteilt dargestellt. 

W�hlen Sie nun die Spalten f�r die X- und Y-Koordinaten aus, tragen einen
Namen ein unter dem die Daten in QGIS angezeigt werden sollen und wenn 
vorhanden auch die WKT-Spalte mit den KBS Informationen. Um die Daten zu 
sehen, klicken Sie auf \button{Hinzuf�gen}. Der Layer verh�lt sich nun wie
jeder andere Vektorlayer in QGIS.

\FloatBarrier

