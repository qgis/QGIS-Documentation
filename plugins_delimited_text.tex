\subsection{Delimited Text Plugin}\label{label_dltext}    

% when the revision of a section has been finalized, 
% comment out the following line:
\updatedisclaimer

The Delimited Text plugin allows you to load a delimited text file as a layer in QGIS. 
    
\subsubsection{Requirements}

To view a delimited text file as layer, the text file must contain:

\begin{enumerate}      
\item A delimited header row of field names. This must be the first line in the text file.
\item The header row must contain an X and Y field. These fields can have any name.
\item The x and y coordinates must be specified as a number. The coordinate system is not important.
\end{enumerate}

An example of a valid text file might look like this:

\begin{verbatim} 
name|latdec|longdec|cell|
196 mile creek|61.89806|-150.0775|tyonek d-1 ne|
197 1/2 mile creek|61.89472|-150.09972|tyonek d-1 ne|
a b mountain|59.52889|-135.28333|skagway c-1 sw|
apw dam number 2|60.53|-145.75167|cordova c-5 sw|
apw reservoir|60.53167|-145.75333|cordova c-5 sw|
apw reservoir|60.53|-145.75167|cordova c-5 sw|
aaron creek|56.37861|-131.96556|bradfield canal b-6|
aaron island|58.43778|-134.81944|juneau b-3 ne|
aats bay|55.905|-134.24639|craig d-7|
\end{verbatim}


Some items of note about the text file are:

\begin{enumerate}        
\item  The example text file uses \mbox{$|$} as delimiter. Any character can be used to delimit the fields.
\item The first row is the header row. It contains the fields name, latdec, longdec and cell.
\item No quotes ({\tt{}"{}}) are used to delimit text fields.
\item The x coordinates are contained in the {\em longdec} field.
\item The y coordinates are contained in the {\em latdec} field.
\end{enumerate}

\subsubsection{Using the Plugin}
To use the plugin you must have QGIS running and use the Plugin Manager to load the plugin:

Start QGIS, then open the Plugin Manager by choosing \mainmenuopt{Plugins} > \dropmenuopttwo{mActionShowPluginManager}{Plugin Manager...}
\index{plugins!manager}
The Plugin Manager displays a list of available plugins.
Those that are already loaded have a check mark to the left of their name.
Click on the checkbox to the left of the \checkbox{Add Delimited Text Layer} plugin and click \button{OK} to load it as described in Section \ref{sec:managing_plugins}.


A new toolbar icon is now present:
\includegraphics[width=0.7cm]{delimited_text}
Click on the icon \toolbtntwo{delimited_text}{Add Delimited Text Layer} to open the Delimited Text dialog as shown in Figure
\ref{fig:delim_text_plugin_dialog}.

%\begin{figure}[ht]
%   \begin{center}
%   \caption{Delimited Text
%Dialog}\label{fig:delim_text_plugin_dialog}\smallskip
%\includegraphics[clip=true, width=8cm]{dialog}            
%   \end{center}  
%\end{figure}

First select the file to import by clicking on the \browsebutton .
Select the desired text file from the file dialog.
Once the file is selected, the plugin attempts to parse the file using the last used delimiter, in this case \mbox{$|$} (see Figure \ref{fig:delim_text_file_selected}).

%\begin{figure}[ht]
%   \begin{center}
%   \caption{File Selected}\label{fig:delim_text_file_selected}\smallskip
%\includegraphics[clip=true, width=8cm]{file_selected}   
%   \end{center}  
%\end{figure}
  
In this case the delimiter \mbox{$|$} is not correct for the file.
The file is actually tab delimited.
Note that the X and Y field drop down boxes do not contain valid field names.

%\begin{figure}[ht]
%   \begin{center}
%   \caption{Fields Parsed from Text
%File}\label{fig:delim_text_file_selected2}\smallskip  
%\includegraphics[clip=true, width=8cm]{file_selected2}
%   \end{center}  
%\end{figure}

To properly parse the file, change the delimiter to tab using \mbox{$\backslash$}t (this is a regular expression for the tab character).
After changing the delimiter, click \button{Parse}.
The drop down boxes now contain the fields properly parsed as shown in Figure \ref{fig:delim_text_file_selected2}.

%\begin{figure}[ht]
%   \begin{center}
%   \caption{Selecting the X and Y
%Fields}\label{fig:delim_text_file_selected3}\smallskip
%\includegraphics[clip=true, width=8cm]{file_selected3}
%   \end{center}  
%\end{figure}

Choose the X and Y fields from the drop down boxes and enter a Layer name as shown in Figure \ref{fig:delim_text_file_selected3}.
To add the layer to the map, click \button{Add Layer}.
The delimited text file now behaves as any other map layer in QGIS.
