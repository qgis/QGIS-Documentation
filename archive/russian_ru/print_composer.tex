%  !TeX  root  =  user_guide.tex

\chapter{Компоновщик карты}\label{label_printcomposer}

% when the revision of a section has been finalized,
% comment out the following line:
% \updatedisclaimer

Компоновщик карты обеспечивает широкие возможности для подготовки
макета карты и его печати. Он позволяет добавлять следующие элементы: карта QGIS, легенда,
масштабная линейка, изображения, фигуры, стрелки и текстовые блоки. При
создании макета доступно изменение размеров, группировка, выравнивание и
изменение положения каждого элемента, а также настройка их свойств.
Готовый макет можно распечатать или экспортировать в растровое изображение,
форматы Postscript, PDF или SVG \footnote{Экспорт в SVG поддерживается,
но может работать некорректно с некоторыми последними версиями Qt4.
Необходимо самостоятельно проверить это на своей системе}, кроме того,
макет можно сохранить как шаблон и использовать его повторно в другой
сессии. Полный перечень инструментов Компоновщика приведен в
таблице~\ref{tab:printcomposer_tools}.

\begin{table}[h]\index{компоновщик карты!инструменты}
\centering\small
\renewcommand{\arraystretch}{2}
 \begin{tabular}{|m{1cm}|m{5.4cm}|m{1cm}|m{5.4cm}|}
 \hline \textbf{Иконка} & \textbf{Описание} & \textbf{Иконка} &
 \textbf{Описание} \\
 \hline \includegraphics[width=0.7cm]{mActionFolder}
 & Загрузить из шаблона &
 \includegraphics[width=0.7cm]{mActionFileSaveAs} & Сохранить как шаблон \\
 \hline \includegraphics[width=0.7cm]{mActionExportMapServer}
 & Экспорт в изображение &
 \includegraphics[width=0.7cm]{mActionSaveAsPDF} & Экспорт в PDF \\
 \hline \includegraphics[width=0.7cm]{mActionSaveAsSVG} & Экспорт в SVG
 & \includegraphics[width=0.7cm]{mActionFilePrint}
 & Печать \\
 \hline \includegraphics[width=0.7cm]{mActionZoomFullExtent} & Полный
 охват & \includegraphics[width=0.7cm]{mActionZoomIn} & Увеличить \\
 \hline \includegraphics[width=0.7cm]{mActionZoomOut} & Уменьшить &
 \includegraphics[width=0.7cm]{mActionDraw} & Обновить \\
 \hline \includegraphics[width=0.6cm]{mActionUndo} & Отменить последнее изменение &
 \includegraphics[width=0.6cm]{mActionRedo} & Вернуть отменённое действие \\
 \hline \includegraphics[width=0.7cm]{mActionAddMap} & Добавить
 карту & \includegraphics[width=0.7cm]{mActionSaveMapAsImage}
 & Добавить изображение \\
 \hline \includegraphics[width=0.7cm]{mActionLabel} & Добавить текст
 & \includegraphics[width=0.7cm]{mActionAddLegend} & Добавить
 легенду \\
 \hline \includegraphics[width=0.7cm]{mActionScaleBar} & Добавить
 масштабную линейку & \includegraphics[width=0.7cm]{mActionAddBasicShape}
 & Добавить фигуру \\
 \hline \includegraphics[width=0.7cm]{mActionAddArrow} & Добавить
 стрелку & \includegraphics[width=0.7cm]{mActionOpenTable} & Добавить
 таблицу \\
 \hline \includegraphics[width=0.7cm]{mActionSelectPan} & Выбрать/переместить
 элемент &
 \includegraphics[width=0.7cm]{mActionMoveItemContent} & Переместить
 содержимое элемента \\
 \hline \includegraphics[width=0.7cm]{mActionGroupItems} & Сгруппировать &
 \includegraphics[width=0.7cm]{mActionUngroupItems} & Разгруппировать \\
 \hline \includegraphics[width=0.7cm]{mActionRaiseItems} & Поднять &
 \includegraphics[width=0.7cm]{mActionLowerItems} & Опустить \\
 \hline \includegraphics[width=0.7cm]{mActionMoveItemsToTop} & На передний
 план &
 \includegraphics[width=0.7cm]{mActionMoveItemsToBottom} & На задний
 план \\
 \hline \includegraphics[width=0.7cm]{mActionAlignLeft} & Выровнять по
 левым краям &
 \includegraphics[width=0.7cm]{mActionAlignRight} & Выровнять по правым
 краям \\
 \hline \includegraphics[width=0.7cm]{mActionAlignHCenter} & Центрировать &
 \includegraphics[width=0.7cm]{mActionAlignVCenter} & Центрировать по
 вертикали \\
 \hline \includegraphics[width=0.7cm]{mActionAlignTop} & Выровнять по верхним
 краям &
 \includegraphics[width=0.7cm]{mActionAlignBottom} & Выровнять по нижним
 краям \\
\hline
\end{tabular}
\caption{Инструменты Компоновщика карты}\label{tab:printcomposer_tools}
\end{table}

Все инструменты Компоновщика карты доступны через меню и кнопки на
панели инструментов. Панель инструментов можно скрыть или отобразить,
наведя мышку на панель и нажав правую кнопку.

\section{Открытие новой компоновки}\label{composertemplates}

Прежде чем начать работать с компоновкой карты, необходимо загрузить
несколько растровых или векторных слоёв в QGIS и настроить их свойства
удобным для себя образом. После того, как все отрисовывается и выглядит
так, как требуется, нажмите на кнопку
\toolbtntwo{mActionNewComposer}{Создать компоновку карты} на панели
инструментов или выберите \mainmenuopt{Файл} \arrow
\dropmenuopttwo{mActionNewComposer}{Создать компоновку карты}.

\section{Использование компоновщика карт}\label{label_useprintcomposer}

\begin{figure}[ht]
   \centering
   \includegraphics[clip=true, width=\textwidth]{print_composer_blank}
   \caption{Компоновщик карт \wincaption}\label{fig:print_composer_blank}
\end{figure}

Открыв компоновку, вы увидите пустой лист, на который можно добавить
загруженную в QGIS карту, легенду, масштабную линейку, изображения,
фигуры, стрелки и текст. На рисунке~\ref{fig:print_composer_blank}
показан начальный вид компоновщика с включенным режимом
\checkbox{Прилипать к сетке} но без каких-либо элементов. В окне
компоновщика есть две вкладки:

\begin{itemize}[label=--]
\item На вкладке \tab{Общие} можно настроить размер и ориентацию бумаги,
задать качество печати в dpi и активировать прилипание к сетке с заданным
шагом. Обратите внимание, что функция \checkbox{Прилипать к сетке}
работает только тогда, когда шаг сетки~>~0. Здесь же можно активировать
опцию \checkbox{Печатать как растр}. Это значит, что все элементы будут
растеризованы перед печатью или при сохранении в Postscript или PDF.
\item Вкладка \tab{Элемент} служит для отображения свойств выделенного
элемента. Для выделения элемента (например, легенды, масштабной линейки
или текста) нажмите кнопку \toolbtntwo{mActionSelectPan}{Выбрать/переместить
элемент}. Затем перейдите на вкладку \tab{Элемент} и настройте свойства
выделенного элемента.
\item Вкладка \tab{История команд} отображает историю всех изменений, сделаных
в макете. Здесь можно как отменить сделанные изменения, так и повторить
ранее отмененные действия.
\end{itemize}

На компоновку можно добавить несколько элементов. Также, в пределах
одной компоновки, можно иметь более одной карты, легенды или масштабной
линейки. Каждый элемент имеет свои настройки и, в случае карты, свой
охват. Удалять элементы компоновки можно при помощи клавиш \keystroke{delete}
и \keystroke{backspace}.

\section{Добавление карты QGIS на компоновку}

Для добавления карты QGIS, нажмите на кнопку
\toolbtntwo{mActionAddMap}{Добавить карту} в панели инструментов
компоновщика и, зажав левую кнопку мыши, протяните курсор, нарисовав
прямоугольник на листе компоновки. Добавленная карта может отображаться
в одном из трех режимов, выбрать которые можно на вкладке \tab{Элемент}
при выделенной карте:

\begin{itemize}[label=--]
\item \selectstring{{}Предпросмотр}{Прямоугольник} является режимом по
умолчанию. Отображается пустой прямоугольник с текстом
\textit{"Место изображения карты"}.
\item \selectstring{{}Предпросмотр}{Кэш} отрисовывает карту в текущем
разрешении экрана. При выполнении масштабирования в окне компоновщика,
карта не перерисовывается, но само изображение масштабируется.
\item \selectstring{{}Предпросмотр}{Отрисовка} выбор этого режима
означает, что при выполнении масштабирования в окне компоновщика карта
будет перерисовываться, но с целью экономии места только до
максимального разрешения.
\end{itemize}

\textbf{Кэш} является режимом по умолчанию для всех только что
добавленных карт.

Изменить размер карты можно, выделив ее при помощи инструмента
\toolbtntwo{mActionSelectPan}{Выбрать/переместить элемент},
и переместив один из голубых маркеров, находящихся в углах. Изменить
другие свойства выделенной карты можно на вкладке \tab{Элемент}.

Для перемещения слоев внутри карты выделите её, затем нажмите на кнопку \\
\toolbtntwo{mActionMoveItemContent}{Переместить содержимое элемента} и
перемещайте слои внутри объекта, зажав левую кнопку мыши.
После того, как элемент расположен в нужном месте, можно зафиксировать
его положение на листе компоновки. Выделите элемент и нажмите правую
кнопку мыши, чтобы \toolbtntwo{mIconLock}{заблокировать} положение
элемента, повторное нажатие разблокирует элемент. Кроме того, можно
заблокировать элементы внутри самой карты активировав настройку
\checkbox{Заблокировать слои для этой карты} в диалоге Карта вкладки
Элемент.

\textbf{Примечание:} QGIS \CURRENT может отображать в компоновке подписи,
созданные новым модулем подписывания, но они некорректно масштабируются.
Поэтому иногда требуется переключаться на стандартный режим подписывания
объектов.

\subsection{Свойства карты "--- диалоги Карта и Границы}

\begin{figure}[ht]
  \centering
  \subfloat[Диалог Карта]{\label{subfig:map_dialog1}\includegraphics[clip=true, width=0.4\textwidth]{print_composer_map1}}
    \hspace{1cm}
  \subfloat[Диалог Границы]{\label{subfig:map_dialog2}\includegraphics[clip=true, width=0.4\textwidth]{print_composer_map2}}
  \caption{Свойства карты "--- диалоги Карта и Границы \wincaption}\label{fig:mapdialog}
\end{figure}

\minisec{Диалог Карта}

Диалог \textbf{Карта} состоит из следующих разделов
(см. Рисунок~\ref{fig:mapdialog}a):

\begin{itemize}[label=--]
\item В разделе \textbf{Предпросмотр} установливаются режимы
предпросмотра Прямоугольник, Кэш и Отрисовка, как описано выше. Для
применения изменений необходимо нажать кнопку \button{Обновить}.
\item В разделе \textbf{Карта} можно изменять размер элемента Карта
путём редактирования ширины и высоты или масштаба. Поле
\selectstring{{}Вращение}{0} позволяет поворачивать содержимое карты
по часовой стрелке, значения угла задаются в градусах. Обратите
внимение, что фреймы с системой координат по умолчанию добавляются со
значением 0. Здесь же можно активировать флажки
\checkbox{Заблокировать слои для этой карты} и \checkbox{Включить экранные
элементы оформления карты}.
\end{itemize}

Если внешний вид карты в главном окне QGIS был изменён в результате
масштабирования или перемещения, либо из-за изменения свойств векторных
или растровых слоёв, обновить карту в окне компоновки можно, выделив её
и нажав на кнопку \button{Обновить}.

\minisec{Диалог Границы}

В диалоге \textbf{Границы} есть разделы:
(смотри Рисунок \ref{fig:mapdialog}b):

\begin{itemize}[label=--]
\item Раздел \textbf{Границы карты} позволяет указать границы карты,
задавая максимальное и минимальное значения для Y и X или нажав кнопку
\button{Взять с экрана}.
\end{itemize}

Если внешний вид карты в главном окне QGIS был изменён в результате
масштабирования или перемещения, либо из-за изменения свойств векторных
или растровых слоёв, обновить карту в окне компоновки можно, выделив ее
и нажав на кнопку \button{Обновить} на вкладке \tab{Элемент} (см.
Рисунок~\ref{fig:mapdialog}a).

\subsection{Свойства карты "--- диалоги Сетка и Общие параметры}

\begin{figure}[ht]
\centering
   \subfloat[Диалог Сетка]{\label{subfig:map_dialog3}\includegraphics[clip=true, width=0.4\textwidth]{print_composer_map3}}
   \hspace{1cm}
   \subfloat[Диалог Общие параметры]{\label{subfig:map_dialog4}\includegraphics[clip=true, width=0.4\textwidth]{print_composer_map4}}
   \caption{Свойства карты "--- диалоги Сетка и Общие параметры \wincaption}\label{fig:sec_map_dialog}
\end{figure}

\minisec{Диалог Сетка}

Диалог \textbf{Сетка} предназначен для настройки координатной сетки
(см. Рисунок \ref{fig:sec_map_dialog}a):

\begin{itemize}[label=--]
\item Флажок \checkbox{Включить сетку?} позволяет наложить сетку на
карту. Сетка может быть в виде линий или в виде перекрестий. Также
можно задать интервал сетки по X и по Y, смещение по X и по Y, размер
перекрестия или толщину линии.
\item Активация флажка \checkbox{Включить аннотацию} добавит
координаты к рамке карты. Аннотация может выводиться за рамкой карты или
внутри нее. Выводить аннотации можно горизонтально, вертикально,
горизонтально и вертикально или по направлению рамки. И, наконец, можно
задать цвет сетки, шрифт для аннотации, отступ аннотации от рамки и
желаемую точность выводимых координат.
\end{itemize}

\minisec{Диалог Общие параметры}

Диалог \textbf{Общие параметры} используется для настройки внешнего
вида элемента (см. Рисунок~\ref{fig:sec_map_dialog}b):

\begin{itemize}[label=--]
\item Здесь можно задать цвет и толщину обводки элемента, установить цвет
фона и степень непрозрачности карты. Кнопка \button{Положение} открывает
диалог \dialog{Положение элемента}, где можно задать положение карты
используя точки привязки или координаты. Здесь же можно включить или
выключить отображение рамки элемента при помощи флажка
\checkbox{Включить рамку}.
\end{itemize}

\section{Добавление других элементов к компоновке}

Кроме добавления карты QGIS на компоновку можно добавлять, размещать,
передвигать и настраивать легенду, масштабную линейку, изображения и
текст.

\subsection{Свойства текста "--- диалоги Текст и Общие параметры}

Для добавления текста нажмите на кнопку
\toolbtntwo{mActionLabel}{Добавить текст}, поместите указатель мыши в
нужное место компоновки и нажмите левую кнопку мыши. Изменить свойства
текстового блока можно на вкладке \tab{Элемент}.

\begin{figure}[ht]
\centering
   \subfloat[Диалог Текст]{\label{subfig:labeloptions1}\includegraphics[clip=true, width=0.4\textwidth]{print_composer_label1}}
   \hspace{1cm}
   \subfloat[Диалог Общие параметры]{\label{subfig:labeloptions2}\includegraphics[clip=true, width=0.4\textwidth]{print_composer_label2}}
   \caption{Свойства текста "--- диалоги Текст и Общие параметры \wincaption}\label{fig:label_option}
\end{figure}

\minisec{Диалог Текст}

Диалог \textbf{Текст} предназначен для управления свойствами текстовых
подписей (см. Рисунок~\ref{fig:label_option}a):

\begin{itemize}[label=--]
\item диалог \textbf{Текст} позволяет добавить текстовые метки к
компоновке. Здесь можно задать выравнивание по горизонтали и вертикали,
указать используемый шрифт и его цвет, а также задать размер полей в мм
\end{itemize}

\minisec{Диалог Общие параметры}

Диалог \textbf{Общие параметры} поможет вам если надо
(см. Рисунок~\ref{fig:label_option}b):

\begin{itemize}[label=--]
\item настроить цвет и толщину рамки элемента, задать
цвет фона и степень непрозрачности. Нажатием на кнопку \button{Положение}
вызывается окно \dialog{Положение элемента}, в котором настраивается
положение текста по точкам привязки или по координатам. Здесь же можно
включить или выключить отображение рамки элемента при помощи флажка
\checkbox{Включить рамку}.
\end{itemize}

\subsection{Свойства изображения "--- диалоги Параметры изображения и Общие параметры}

Для добавления изображения нажмите на кнопку
\toolbtntwo{mActionSaveMapAsImage}{Добавить изображение}, поместите
курсор в нужное место компоновки и нажмите левую кнопку мыши, при
необходимости настройте внешний вид на вкладке \tab{Элемент}.

\begin{figure}[ht]
\centering
   \subfloat[Диалог Параметры изображения]{\label{subfig:print_composer_image1}\includegraphics[clip=true, width=0.30\textwidth]{print_composer_image1}}
     \hspace{1cm}
   \subfloat[Диалог Общие параметры]{\label{subfig:print_composer_image2}\includegraphics[clip=true, width=0.4\textwidth]{print_composer_image2}}
   \caption{Свойства изображения "--- диалоги Параметры изображения и Общие параметры \wincaption}\label{fig:imageoptions}
\end{figure}

\minisec{Диалог Параметры изображения}

Диалог \textbf{Параметры изображения} состоит из следующих
разделов (см. Рисунок~\ref{fig:imageoptions}a):

\begin{itemize}[label=--]
\item В разделе \textbf{Искать в каталогах} добавляются и удаляются
каталоги с изображениями в формате SVG.
\item В поле \textbf{Предпросмотр} показаны все изображения, найденные
в указанных каталогах.
\item Раздел \textbf{Параметры} показывает текущее изображение и
позволяет задать его ширину, высоту и угол поворота по часовой стрелке.
Также можно указать свой путь к файлам SVG. Установка флажка
\checkbox{Синхронизировать с картой} синхронизирует поворот изображения
на карте QGIS (например, повёрнутый указатель севера) с соответствующим
изображением в компоновке.
\end{itemize}

\minisec{Диалог Общие параметры}

Используя диалог \textbf{Общие параметры} вы можете
(см. Рисунок~\ref{fig:imageoptions}b):

\begin{itemize}[label=--]
\item настроить цвет и толщину рамки элемента, задать
цвет фона и степень непрозрачности. Нажатием на кнопку \button{Положение}
вызывается диалог \dialog{Положение элемента}, который позволяет
настроить положение изображения, используя точки привязки или координаты.
Здесь же можно включить отображение рамки элемента при помощи флажка
\checkbox{Включить рамку}.
\end{itemize}

\subsection{Свойства легенды "--- диалоги Общие, Элементы легенды и Общие параметры}

Для добавления легенды нажмите кнопку
\toolbtntwo{mActionAddLegend}{Добавить легенду}, поместите указатель
мыши в нужное место компоновки и нажмите левую кнопку мыши. Настроить
внешний вид нового элемента можно на вкладке \tab{Элемент}.

\begin{figure}[h]
\centering
   \subfloat[Диалог Общие]{\label{subfig:print_composer_legend1}\includegraphics[clip=true, width=0.3\textwidth]{print_composer_legend1}}
   \hspace{1cm}
   \subfloat[Диалог Элементы легенды]{\label{subfig:print_composer_legend2}\includegraphics[clip=true, width=0.3\textwidth]{print_composer_legend2}}
   \hspace{1cm}
   \subfloat[Диалог Общие параметры]{\label{subfig:print_composer_legend3}\includegraphics[clip=true, width=0.3\textwidth]{print_composer_legend3}}
   \caption{Свойства легенды "--- диалоги Общие, Элементы легенды и Общие параметры \wincaption}\label{fig:legendoptions}
\end{figure}

\minisec{Диалог Общие}

Диалог \textbf{Общие} используется для настройки внешнего вида
легенды (см. Рисунок~\ref{fig:legendoptions}a):

\begin{itemize}[label=--]
\item Здесь можно изменить заголовок легенды. Доступно изменение шрифта
заголовка, группы и слоя. Пользователь может изменять ширину и высоту
знаков, добавлять группы, знаки, подписи и изменять отступы элементов.
\end{itemize}

\minisec{Диалог Элементы легенды}

Вид отдельного элемента легенды настраивается в диалоге
\textbf{Элементы легенды} (см. Рисунок~\ref{fig:legendoptions}b):

\begin{itemize}[label=--]
\item В этом окне перечислены все элементы легенды и здесь можно
изменять их порядок, редактировать имена слоев, удалять и восстанавливать
элементы списка. Нажатие на кнопку \button{Update} после изменения
символики в главном окне QGIS применит эти изменения к элементам
легенды в окне компоновщика. Порядок элементов может быть изменен
кнопками Вверх и Вниз или путём перетаскивания элементов в списке.
\end{itemize}

\minisec{Диалог Общие параметры}

Настройки в диалоге \textbf{Общие параметры} задают общий вид
элемента компоновки (см. Рисунок~\ref{fig:legendoptions}c):

\begin{itemize}[label=--]
\item Здесь можно настроить цвет и толщину рамки элемента, задать
цвет фона и степень непрозрачности. Нажатием на кнопку
\button{Положение} вызывается диалог \dialog{Положение элемента},
который позволяет настроить положение легенды, используя точки привязки
или координаты. Здесь же можно включить отображение рамки элемента при
помощи флажка \checkbox{Включить рамку}.
\end{itemize}

\subsection{Свойства масштабной линейки "--- диалоги Масштабная линейка и Общие параметры}

Для добавления масштабной линейки нажмите кнопку
\toolbtntwo{mActionAddLegend}{Добавить масштабную линейку}, поставьте указатель
мыши в нужное место компоновки и нажмите левую кнопку мыши. Настроить
внешний вид нового элемента можно на вкладке \tab{Элемент}.

\begin{figure}[ht]
\centering
\subfloat[Диалог Масштабная линейка]{\label{subfig:scalebaroptions1}\includegraphics[clip=true, width=0.35\textwidth]{print_composer_scalebar1}}
\hspace{1cm}
\subfloat[Диалог Общие параметры]{\label{subfig:scalebaroptions2}\includegraphics[clip=true, width=0.4\textwidth]{print_composer_scalebar2}}
\caption{Свойства масштабной линейки "--- диалоги Масштабная линейка и Общие параметры \wincaption}\label{fig:scalebaroptions}
\end{figure}

\minisec{Диалог Масштабная линейка}

Используя диалог \textbf{Масштабная линейка}, можно
(см. Рисунок~\ref{fig:scalebaroptions}a):

\begin{itemize}[label=--]
\item При помощи этого окна можно задать размер сегмента масштабной
линейки в единицах карты, количество единиц карты в одном делении
линейки; указать, сколько сегментов должно отображаться слева и справа
от 0.
\item Установить стиль масштабной линейки. Доступны следующие стили:
одинарная и двойная рамка, штрих вверх-вниз, штрих вверх, штрих вниз и
числовой стиль.
\item Кроме того, можно задать высоту, толщину линии, подпись и отступы
для масштабной линейки. Добавить подпись с единицами измерения, настроить
шрифт и цвет.
\end{itemize}

\minisec{Диалог Общие параметры}

Диалог \textbf{Общие параметры} поможет (см. Рисунок~\ref{fig:scalebaroptions}b):

\begin{itemize}[label=--]
\item настроить цвет и толщину рамки элемента, задать цвет фона и
степень непрозрачности. Нажатием на кнопку \button{Положение}
вызывается диалог \dialog{Положение элемента}, который позволяет
настроить положение линейки, используя точки привязки или координаты.
Здесь же можно включить отображение рамки элемента при помощи флажка
\checkbox{Включить рамку}.
\end{itemize}

\section{Инструменты навигации}

Для перемещения по компоновке существует 4 основных инструмента:

\begin{itemize}[label=--]
\item \toolbtntwo{mActionZoomIn}{Увеличить},
\item \toolbtntwo{mActionZoomOut}{Уменьшить},
\item \toolbtntwo{mActionZoomFullExtent}{Полный охват} и
\item \toolbtntwo{mActionDraw}{Обновить}, если изображение находится в
несогласованном состоянии.
\end{itemize}

\section{Инструменты отмены и возврата}

В процессе работы над макетом можно отменять и возвращать сделанные изменения.
Для этого служат инструменты:

\begin{itemize}[label=--]
\item \toolbtntwo{mActionUndo}{Отменить последнее изменение}
\item \toolbtntwo{mActionRedo}{Вернуть отменённое действие}
\end{itemize}

Того же эффекта можно добиться выделив нужное действие на вкладке \tab{История команд}
(см. Рисунок~\ref{fig:commandhist}).

\begin{figure}[ht]
   \centering
   \includegraphics[clip=true, width=14cm]{command_hist}
   \caption{История команд в Компоновщике карт \wincaption}\label
   {fig:commandhist}
\end{figure}

\section{Добавление фигуры и стрелки}

К компоновке можно добавлять фигуры (эллипс, прямоугольник, треугольник)
и стрелки.

\begin{figure}[ht]
\centering
\subfloat[Диалог Фигура]{\label{subfig:shapedialog}\includegraphics[clip=true, width=0.4\textwidth]{print_composer_shape}}
\hspace{1cm}
\subfloat[Диалог Стрелка]{\label{subfig:arrowdialog}\includegraphics[clip=true, width=0.4\textwidth]{print_composer_arrow}}
\caption{Свойства фигур и стрелок "--- диалоги Фигура и Стрелка \wincaption}\label{fig:shapearrow}
\end{figure}

\begin{itemize}[label=--]
\item Диалог \textbf{Фигура} позволяет нарисовать на компоновке эллипс,
прямоугольник или треугольник. Можно настроить цвет обводки и заливки,
толщину обводки и угол поворота по часовой стрелке.
\item Диалог \textbf{Стрелка} предназначен для рисования стрелок на
компоновке. Доступна настройка цвета, толщины линии и размера маркера.
Есть возможность использовать маркер по умолчанию, отказаться от маркера
или загрузить его из файла SVG. При использовании маркеров в формате SVG
можно задать отдельно маркер конца и маркер начала.
\end{itemize}

\section{Добавление значений из таблицы атрибутов}

Возможно добавление на компоновку части атрибутивной таблицы векторного
слоя.

\begin{figure}[ht]
\centering
\subfloat[Диалог Таблица]{\label{subfig:tabledialog1}\includegraphics[clip=true, width=0.38\textwidth]{print_composer_attribute1}}
\hspace{1cm}
\subfloat[Диалог Общие параметры]{\label{subfig:tabledialog2}\includegraphics[clip=true, width=0.38\textwidth]{print_composer_attribute2}}
\caption{Свойства таблицы атрибутов "--- диалоги Таблица и Общие параметры \wincaption}\label{fig:attrcomp}
\end{figure}

\minisec{Диалог Таблица}

Диалог \textbf{Таблица} предоставляет следующий функционал
(см. Рисунок~\ref{fig:attrcomp}a):

\begin{itemize}[label=--]
\item В диалоге \textbf{Таблица} выбирается векторный слой и столбцы
атрибутивной таблицы. Содержимое колонок можно отсортировать по возрастанию
или по убыванию.
\item Можно указать максимальное количество видимых записей или включить
отображение атрибутов только видимых на компоновке объектов.
\item Кроме того, предоставляется возможность настроить отображение сетки
таблицы и задать шрифт для заголовков и содержимого.
\end{itemize}

\minisec{Диалог Общие параметры}

Диалог \textbf{Общие параметры} используется, когда необходимо
(см. Рисунок~\ref{fig:attrcomp}b):

\begin{itemize}[label=--]
\item настроить цвет и толщину рамки элемента, задать
цвет фона и степень непрозрачности. Нажатием на кнопку \button{Положение}
вызывается диалог \dialog{Положение элемента}, который позволяет
настроить положение таблицы, используя точки привязки или координаты.
Здесь же можно включить отображение рамки элемента при помощи флажка
\checkbox{Включить рамку}.
\end{itemize}

\section{Сортировка и выравнивание элементов}

Функции сортировки элементов находятся в выпадающем меню
\toolbtntwo{mActionRaiseItems}{Поднять выбранные элементы}. Выделите
элемент компоновки и выберите необходимое действие, чтобы расположить
выделенный элемент выше или ниже других (см. таблицу~\ref{tab:printcomposer_tools}).

Инструменты выравнивания доступны через выпадающее меню \\
\toolbtntwo{mActionAlignLeft}{Выровнять выбранные элементы по левым краям}
(см. таблицу~\ref{tab:printcomposer_tools}). Перед использованием
инструмента выравнивания необходимо выделить несколько элементов, а
затем нажать кнопку соответствующего инструмента. Все выделенные объекты
будут выровнены в пределах их общих границ.

\section{Создание вывода}

На Рисунке~\ref{fig:print_composer_complete} показан пример компоновки,
которая содержит все описанные выше элементы.

\begin{figure}[h]
   \centering
   \includegraphics[clip=true, width=\textwidth]{print_composer_complete}
   \caption{Компоновка с добавленными картой, легендой, масштабной линейкой, координатами и текстом \wincaption} \label{fig:print_composer_complete}
\end{figure}

Компоновщик печати позволяет экспортировать результат в несколько
форматов, при этом можно задавать разрешение (качество печати) и размер
бумаги:

\begin{itemize}[label=--]
\item Кнопка \toolbtntwo{mActionFilePrint}{Печать} предназначена для
печати компоновки на подключенный принтер или в Postscript-файл, в
зависимости от установленных драйверов принтера.
\item Нажатием на кнопку
\toolbtntwo{mActionExportMapServer}{Экспорт в изображение}
компоновку можно экспортировать в один из графических форматов: PNG,
BPM, TIF, JPG \ldots
\item Нажав на кнопку \toolbtntwo{mActionSaveAsPDF}{Экспорт в PDF}, вы
сохраните компоновку в формате PDF.
\item Кнопка \toolbtntwo{mActionSaveAsSVG}{Экспорт в SVG} создаст из
компоновки файл формата SVG (Scalable Vector Graphic).
\textbf{Примечание:} Сейчас сохранение в SVG работает на базовом уровне.
Это не проблема QGIS, а недостаток нижележащих библиотек Qt. Вероятно,
в будущем эти проблемы будут решены.
\end{itemize}

\section{Сохранение и загрузка шаблона}

При помощи кнопок \toolbtntwo{mActionFileSaveAs}{Сохранить как шаблон}
и \toolbtntwo{mActionFolder}{Загрузить из шаблона} состояние открытой
компоновки можно сохранить как *.qpt шаблон и загрузить шаблон в другой
сессии.

Кнопка \toolbtntwo{mActionComposerManager}{Управление компоновками}
на панели инструментов и пункт меню \\
\mainmenuopt{Файл} \arrow \dropmenuopttwo{mActionComposerManager}{Управление компоновками}
позволяют добавлять новые компоновки и управлять существующими.

\begin{figure}[h]
   \centering
   \includegraphics[clip=true, width=8cm]{print_composer_manager}
   \caption{Управление компоновками \wincaption}
   \label{fig:print_composer_manager}
\end{figure}

\FloatBarrier
