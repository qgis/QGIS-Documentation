\section{Working with Raster Data}\label{label_raster}
\index{raster layers|(}

% when the revision of a section has been finalized, 
% comment out the following line:
\updatedisclaimer

QGIS supports a number of raster data formats. This section describes how to
work with raster data in QGIS.

\subsection{What is raster data?}\label{label_whatsraster}
\index{raster layers!definition}

Raster data in GIS are matrices of discrete cells that represent features on,
above or below the earth's surface. Each cell in the raster grid is the same
size, and cells are usually rectangular (in QGIS they will always be
rectangular). Typical raster datasets include remote sensing data such as
aerial photography or satellite imagery and modelled data such as an elevation
matrix.

Unlike vector data, raster data typically do not have an associated database
record for each cell.

In GIS, a raster layer would have georeferencing data associated with it which
will allow it to be positioned correctly in the map display to allow other
vector and raster data to be overlaid with it. QGIS makes use of georeferenced
rasters to properly display the data.\index{raster layers!georeferenced}
	
\subsection{Raster formats supported in QGIS}\label{label_rastformats}
QGIS supports a number of different raster formats. Currently tested formats
include:\index{raster layers!data formats}

\begin{itemize}
\item Arc/Info Binary Grid
\item Arc/Info ASCII Grid
\item GRASS Raster
\item GeoTIFF
\item JPEG
\item Spatial Data Transfer Standard Grids (with some limitations)
\item USGS ASCII DEM
\item Erdas Imagine
\end{itemize}

Because the raster implementation in QGIS is based on the GDAL library, other
raster formats implemented in GDAL are also likely to work, but have not yet
been tested. See Appendix \ref{appdx_gdal} for more
details.\index{raster layers!GDAL implementation}
	
\subsection{Loading raster data in QGIS}\label{label_loadraster}

Raster layers are loaded either by clicking on the 
\toolbtntwo{mActionAddRasterLayer}{Load Raster} icon or by
selecting the \mainmenuopt{View}>\dropmenuopttwo{mActionAddRasterLayer}{Add Raster Layer} menu option. More than one 
layer can be loaded at the same time by holding down the
\keystroke{Control} or \keystroke{Shift} key
and clicking on multiple items in the dialog \dialog{Open a GDAL Supported
Raster Data Source}.\index{raster layers!loading}

Please refer to section \ref{sec:load_grassdata} if you intend to load GRASS rasterdata.
	
\subsection{Raster Properties Dialog}\label{label_rasterprop}

To view and set the \dropmenuopt{properties} for a raster layer, right click on the layer
name. This displays the raster layer context menu that includes a number of
items that allow you to:\index{raster layers!context menu}

\begin{figure}[ht]
 \begin{center}
   \caption{Raster context menu}\label{fig:raster_contextmenu}\smallskip
   \includegraphics[clip=true, width=5cm]{rasterContext}
 \end{center}  
\end{figure}

\begin{itemize}
\item \toolbtntwo{mActionZoomFullExtent}{Zoom to the full extent} of the raster
\item Zoom to the best scale of the raster
\item Show the raster in the map overview window
\item \toolbtntwo{mActionRemoveLayer}{Remove layer} from the map
\item Open the raster layers properties
\item Rename the layer
\item Add a layer group
\item \toolbtntwo{mActionExpandTree}{Expand legend tree view}
\item \toolbtntwo{mActionCollapseTree}{Collapse legend tree view}
\item Show file groups
\end{itemize}

Choose \dropmenuopt{Properties} from the context menu to open the
\button{Raster Layer Properties}
dialog for the layer.\index{raster layers!properties}

Figure \ref{fig:raster_properties} shows the \dialog{Raster Layer
Properties} dialog. There are several
tabs on the dialog: 
\begin{itemize}
 \item \tab{Symbology}, 
 \item \tab{Transparency},
 \item \tab{Colormap},
 \item \tab{General}, 
 \item \tab{Metadata}, 
 \item \tab{Pyramids} 
 \item and \tab{Histogram}.
\end{itemize}

\begin{figure}[h]
  \begin{center}
   \caption{Raster Layers Properties Dialog}\label{fig:raster_properties}\smallskip
   \includegraphics[clip=true, width=14cm]{rasterPropertiesDialog}
\end{center}  
\end{figure}

\subsubsection{Symbology Tab}\label{label_sombology}

QGIS can render two different types of rasterlayers:\index{raster layers!supported channels}

\begin{itemize}
\item Single band gray
\item Three band color
\end{itemize}

Within both rendertypes you can invert the color using the checkbox
\checkbox{Invert color map}.

\minisec{Single band gray}

This selection offers you two possibilites to choose. At first you can
select which band you like to use for rendering. The second option offers
a selection of available colortables for rendering.

The following settings are available through the dropdownbox
\selectstring{color map}{Grayscale}, where grayscale is the default
setting.
Also available are
\begin{itemize}
\item Pseudocolor
\item Freak Out
\item Colormap
\end{itemize}

When selecting the entry \selectstring{color map}{Colormap}, the tab
\tab{Colormap} becomes available. See more on that at chapter
\ref{label_colormaptab}.

QGIS can restrict the data displayed to only show cells whose values are
within a given number of standard deviations of the mean for the
layer.\index{raster layers!standard deviation} This is useful when you have one or
two cells with abnormally high values in a raster grid that are having a
negative impact on the rendering of the raster. This option is only available
for pseudocolor images.


\minisec{Three band color}

This selection offers you a wide range of options to modify the appereance
of your rasterlayer. For example you could switch color-bands from the
standard RGB-order to something else.

Also scaling of colors are available.
Note that the scaling of you custom min and max values for each band is
only available if your dataset has three bands to do so.

% SH: Not applyable anymore
%%%QGIS supports three forms of raster layers:\index{raster layers!supported channels}
%%%
%%%\begin{itemize}
%%%\item Single Band Grayscale Rasters
%%%\item Palette Based RGB Rasters
%%%\item Multiband RGB Rasters
%%%\end{itemize}
%%%
%%%From these three basic layer types, eight forms of symbolised raster display
%%%can be used:\index{raster layers!rendering interpretation}
%%%
%%%\begin{itemize}
%%%\item Single Band Grayscale
%%%\item Single Band Pseudocolor
%%%\item Paletted Grayscale (where only the red, green or blue component of the
%%%image is displayed)
%%%\item Paletted Pseudocolor (where only the red, green or blue component of the
%%%image is displayed, but using a pseudocolor algorithm)
%%%\item Paletted RGB
%%%\item Multiband Grayscale (using only one of the bands to display the image)
%%%\item Multiband Pseudocolor (using only one of the bands shown in
%%%pseudocolor)
%%%\item Multiband RGB (using any combination of three bands)
%%%\end{itemize}

\smallskip

QGIS can invert the colors in a given layer so that light colors become dark
(and dark colors become light). Use the \checkbox{Invert Color Map} checkbox to
enable / disable this behavior.\index{raster layers!icolor map inversion}

\begin{Tip}\caption{\textsc{Viewing a Single Band of a Multiband Raster}}
\qgistip{If you want to view a single band (for example Red) of a multiband
image, you might think you would set the Green and Blue bands to ``Not
Set''. But this is not the correct way. To display the Red band, 
set the image type to grayscale, then select Red as the band to use for Gray.
}
\end{Tip} 

\subsubsection{Transparency Tab} \label{rastertab:transparency}

QGIS has the ability to display each raster layer at varying transparency
levels.\index{raster layers!transparency} Use the transparency slider to indicate to
what extent the underlying layers (if any) should be visible though the
current raster layer. 
This is very usefull, if
you like to overlay more than one rasterlayer, e.g. a shaded relief-map
overlayed by a classified rastermap. This will make the look of the map
more plastique.

Additionally you can enter a rastervalue, which should be treated as
{\em NODATA}.

An even more flexible way to customize the transparency can be done in the
\guiheading{Custom transparency options} section.
The transparency of every pixel can be set in this block.

As an example we want to set the water of our example rasterfile
\filename{landcover.tif} to a transparency of 20\%. The following steps
are neccessary:
\begin{enumerate}
 \item  Load the rasterfile \filename{landcover}
 \item Open the \dialog{properties} dialog by double-clicking on the
 rasterfile-name in the legend or by right-clicking and choosing
 \dropmenuopt{Properties} from the popup meun.
 \item select the \tab{Transparency} tab
 \item \label{enum:add} Click the \toolbtntwo{mActionNewAttribute}{Add values manually}
 button. A new row will appear in the pixel-list.
 \item \label{enum:transp} enter the the raster-value (we use 0 here) and adjust the
 transparency to 20\%
 \item press the \button{Apply} button and have a look at the map
\end{enumerate}

You can repeat the steps \ref{enum:add} and \ref{enum:transp} to adjust
more values with custom transparency.

As you can see this is quite easy set custom transparency, but it can be
quite a lot of work. Therefor you can use the button
\toolbtntwo{mActionFileSave}{Export to file} to save your
transparency-list to a file. The button
\toolbtntwo{mActionAddRasterLayer}{Import from file} loads your
transparency-settings and applies them to the current rasterlayer.

\subsubsection{Colormap} \label{label_colormaptab}
% FIXME: Write me

The \tab{Colormap} tab is only available, when you have selected a
single-band-rendering within the tab \tab{Symbology} (see chapt. \ref{label_sombology}).

Three ways of color interpolation are available:
\begin{itemize}
\item Discrete
\item Linear 
\item Exact
\end{itemize}

The button \button{Add Entry} adds a color to the individual color-table.
Double-Clicking on the value-column lets you inserting a specific value.
Double clicking on the color-column opens the dialog \dialog{Select
color} where you can select a color to apply on that value.

Alternativly you can click on the button
\toolbtntwo{mActionNewAttribute}{Load colormap from Band}
, which tries to
load the table from the band (if it has any).

The block \guiheading{Generate new color map} allows you to create newly
categorized colormaps. You only need to select the \selectnumber{number of
classes}{15} you need and press the button \button{Classify}. Currently
only one \selectstring{Classification mode}{Equal Interval} is
supported\index{raster layer!classify}.

\subsubsection{General Tab}\label{label_generaltab}

The \tab{General} tab displays basic information about the selected raster,
including the layer source and  display name in the legend (which can be
modified). This tab also shows a thumbnail of the layer, its legend symbol,
and the palette.\index{raster layers!properties}

Additionally scale-dependent visability can be set in this tab. You need to
check the checkbox and set an appropriate scale where your data will be
displayed in the map canvas.

Also the spatial reference system is printed here as a PROJ.4-string. 
This can be modified by hitting the \button{Change} button.

\subsubsection{Metadata Tab}\label{label_metatab}

The \tab{Metadata} tab displays a wealth of information about the raster layer,
including statistics about each band in the current raster layer. Statistics
are gathered on a 'need to know' basis, so it may well be that a given layers
statistics have not yet been collected.\index{raster layers!metadata}

This tab is mainly for information. You cannot change any values printed
inside this tab. To update the statistics you need to change to tab
\tab{Histogram} and press the button \button{Refresh} on the bottom right,
see ch. \ref{label_histogram}.

\subsubsection{Pyramids Tab}\label{raster_pyramids}

Large resolution raster layers can slow navigation in QGIS. By creating lower
resolution copies of the data (pyramids), performance can be considerably
improved as QGIS selects the most suitable resolution to use depending on the
level of zoom.
\index{raster layers!pyramids}
\index{raster layers!resolution pyramids}

You must have write access in the directory where the original data is stored
to build pyramids. \\
Several resampling methods can be used to calculate the pyramides:
\begin{itemize}
\item Average
\item Nearest Neighbour
\end{itemize}

When checking the checkbox \checkbox{Build pyramids internally if
possible} QGIS tries to build pyramids internally.

Please note that building pyramids may alter the original data file and once
created they cannot be removed. If you wish to preserve a 'non-pyramided'
version of your raster, make a backup copy prior to building pyramids.

\subsubsection{Histogram Tab}\label{label_histogram}

The histogram tab allows you to view the distribution\index{raster layers!histogram} 
of the bands or colors in your raster. You must first generate the raster statistics 
by clicking the \button{Refresh} button. You can choose which bands to display by 
selecting them in the list box at the bottom left of the tab. Two different
chart types are allowed: 
\begin{itemize}
\item Barcharts and 
\item Linegraphs.
\end{itemize}

Once you view the histogram, you'll notice that the band statistics have been
populated on the \tab{metadata} tab.\index{raster layers!metadata)}

\begin{Tip}\caption{\textsc{Gathering Raster Statistics}}
\qgistip{To gather statistics for a layer, select pseudocolor rendering and
click the \button{Apply} button. Gathering statistics for a layer can be time
consuming. Please be patient while QGIS examines your
data!\index{raster layers!statistics}
}
\end{Tip}

